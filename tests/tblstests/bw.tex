\subsection{Simple boxes}

\lipsum[1]

\tblsbwboxlight{This box is light gray}{
\lipsum[2]
}

\lipsum[14]
 
\tblsbwboxdark{This box is dark gray}{
\lipsum[3]
}
 
\subsection{Lined boxes}

\lipsum[15]

\tblsbwthinsandwich{This box has thin lines around it}{
\lipsum[4]
}

\lipsum[5]

\tblsbwthicksandwich{This box has thick lines around it}{
\lipsum[6-7]
}


\subsection{Floating boxes}
\lipsum[8-9]

\tblsbwboxlight{This box is inline}{
\lipsum[8]
}

\begin{figure}
\color{red} This dummy figure is used to check whether the floating box interferes with figure numbering
 \caption{Dummy figure for counting}
\label{fig:dummycountfiguretbls}
\end{figure}


\begin{figure}
\tblsbwboxlight{This box floats to the top}{
\lipsum[9]
}
\end{figure}

\lipsum[11-12]

\begin{figure}
\color{red}The caption of this figure is only 1 higher than \figref{fig:dummycountfiguretbls}
 \caption{Dummy figure for counting. The uncaptioned figure with the floating box is not counted.}
\label{fig:dummycountfiguretbls2}
\end{figure}


\subsection{Boxes with icons}
\lipsum[13-14]


\tblsbwboxlight[bulb]{This light box has a bulb icon}{
\lipsum[15]
}

\lipsum[16]
 
\tblsbwboxdark[glass]{This dark box has a looking glass icon}{
\lipsum[17]
}

\subsection{Multipage boxes}


\lipsum[18]

\tblsbwthinsandwich{This box has thin lines around it. The lines repeat on page breaks}{
\lipsum[19-23]
}

\lipsum[24-25]

\tblsbwthicksandwich{This box has thick lines around it. The lines repeat on page breaks}{
\lipsum[26-30]
}

\lipsum[31]