%% -*- coding:utf-8 -*-
\chapter{Style rules for authors}

\inlinetodostefan{How to submit a book}

%\section{How to submit a book}

%OMP account

%Contact the series editor informally.



The following sections describe the layout of various items that play a role in typesetting. 

\section{Headings}

Please provide the headings of chapters and sections in normal spelling. If you are writing English,
please do not capitalize content words unless capitalization is required by orthographical rules. 

Your document may use structures up to six levels, that is there may be a section with
the number 1.2.3.4.5.6.\footnote{
  See page~\pageref{sec-Chinese} for an actual use of subsubsections.%
} However, such elaborated structures may be difficult for the readers, so
there should be a good motivation for going beyond four levels.

%\subsubsubsubsection{Test}

\section{Emphasizing}

If you want to \emph{emphazize} terms, please use \emph{italics}. Bold face should be avoided if
possible.
% Small Caps

\section{Glossed examples}

Please gloss all examples and provide them with translations. The glossing should be done according
to the Leipzig Glossing Rules. If you need special abbreviations that are not defined by the Leipzig
Glossing Rules\footnote{
\url{http://www.eva.mpg.de/lingua/resources/glossing-rules.php}. 27.10.2013.
}
\todo{Martin: provide an example}, put them in a table immedeately before the first chapter of a
monograph. In case of edited volumes the tables with abbreviations should be placed immedeately
before the references.

The formating of linguistic examples in typological series follows the format that is used by the
World Atlas of Language Structures \citep{DM2013a-ed}: If there is just one example, the language name and references
follow the example number as in (\mex{1}):
{\def\exfont{\normalsize\itshape}
\ea\label{ex-typology}
{\rm Mising\il{Mising} \citep[69]{Prasad91a}}\\
\gll azɔ́në dɔ́luŋ\\
     small village\\ 
\glt `a small village' 
\z
If two examples with different numbers belong to the same language, the language name is provided in
both examples.\footnote{
  See examples (5) and (6) in the WALS online at \url{http://wals.info/chapter/87} for an example.
}
If an example consists of several subexamples, the language and references follow the letters as in
(\mex{1}):
\eal
\ex {\rm Apatani\il{Apanti} \citep[23]{Abraham85a}}\\
\gll aki atu\\ 
     dog small\\ 
\glt ‘the small dog’ 
\ex {\rm Temiar\il{Temiar} \citep[155]{Benjamin76a}}\\ 
\gll dēk mənūʔ\\
     house big\\
\glt ‘big house’ 
\zl
}

\section{Figures and tables}

Figures and tables should come with a caption. Captions are set below figures and above tables. The caption should be in
normal spelling, that is without capitalization of content words. Please number figures and
tables. The number should consist of the chapter number and a number that starts with one for every
new chapter. There has to be one counter for figures and another one for
tables. Figure~\vref{fig-example-fig-the-dog-barks} is an example of a figure and
Table~\vref{tab-example-croft} is an example of a table.

\begin{figure}[htbp]
\centerline{%
\begin{tikzpicture}
\tikzset{level 1+/.style={level distance=2\baselineskip}}
\tikzset{frontier/.style={distance from root=6\baselineskip}}
\Tree[.S
       [.NP 
         [.Det the ]
         [.N   dog ] ]
       [.VP barks ] ]
\end{tikzpicture}
}
\caption{\label{fig-example-fig-the-dog-barks}An example of a figure: Analysis of the sentence \emph{The dog barks.}}
\end{figure}

\begin{table}[htbp]
\caption{\label{tab-example-croft}An example of a table taken from \citew[214]{Croft2003a}}
\centerline{
\begin{tabularx}{\textwidth}{llX}\hline\hline
          & Low categoriality unit      & Unit with wich it clusters\\\hline
`Noun'    & low referentiality NP       & forgrounded verb\\
          & attached body part noun     & forgrounded verb\\
          & anaphoric NP                & forgrounded verb, emphasized element\\
`Verb'    & tense/aspect/mood auxiliary & forgrounded verb\\ 
\hline\hline
\end{tabularx}
}
\end{table}

\section{Footnotes}

Please use footnotes rather than endnotes. Footnotes go to the end of the clause after punctuation
unless they refer to a specific word or phrase.\footnote{
  This is an example of a footnote that refers to the whole clause.
}

Please do not use footnotes in tables or figures\footnote{
  This is a footnote that refers to the word \emph{figures}. If only there was something interesting
  to say about figures apart from the fact that they are floating objects.
} but attach them to the text preceeding or following them.


\section{Quotations}

If long passages are quoted, they should be indented and the quote should be followed by the exact reference:
\begin{quotation}
Precisely constructed models for linguistic structure can play an
important role, both negative and positive, in the process of discovery 
itself. By pushing a precise but inadequate formulation to
an unacceptable conclusion, we can often expose the exact source
of this inadequacy and, consequently, gain a deeper understanding
of the linguistic data. More positively, a formalized theory may 
automatically provide solutions for many problems other than those
for which it was explicitly designed. Obscure and intuition-bound
notions can neither lead to absurd conclusions nor provide new and
correct ones, and hence they fail to be useful in two important respects. 
I think that some of those linguists who have questioned
the value of precise and technical development of linguistic theory
have failed to recognize the productive potential in the method
of rigorously stating a proposed theory and applying it strictly to
linguistic material with no attempt to avoid unacceptable conclusions 
by ad hoc adjustments or loose formulation.
\citep[5]{Chomsky57a}
\end{quotation}
%
Short passages should be quoted inline using quotes: \citet[5]{Chomsky57a} stated that ``[o]bscure
  and intuition-bound notions can neither lead to absurd conclusions nor provide new and
correct ones''.

If you quote text that is not in the language of the book provide a translation. Short quotes should
be translated inline, long quotes should be translated in a footnote.

\section{Crossreferences in the text}

Please use the crossreferenceing mechanisms of your text editing/type setting software. Using such
crossreferencing mechanisms is less error-prone when you shift text blocks around and in addition
all these crossreferences will be turned into hyperlinks between document parts, which makes the
final documents much more useful.

If you have numbered example sentence, please start with (1) for every new chapter.

Please use capitals if you refer to numbered chapters, sections, tables, figure, or footnotes: \emph{As we have shown in
  Section~3.1}, \emph{As Figure~3.5 shows}. Do not capitalize without a number: \emph{In the
  following section we will discuss}.
Depending on the series and the langauge the book is published in authors may also use the § sign
instead of the word \emph{Section}. So the above sentence would read: \emph{As we have shown in
  §3.1}.

\section{References}
\label{sec-references-authors}

If books or larger articles are cited, exact page numbers should be provided. This is both good for
authors since it helps them to keep track of their source and enables them to find and reread the
referenced passages and it is a good service to the readers.

We use the \emph{Unified Style Sheet for Linguistics}, which is described here:
\url{http://celxj.org/downloads/UnifiedStyleSheet.pdf}. The \bibtex file is contained in the \latex
classes that are used for typesetting \lsp books. Please deliver a \bibtex file with all your
references together with your submissions. \bibtex can be exported from all common bibliography
tools (We recommend BibDesk for the Mac and JabRef for all other platforms). Please make sure that
all \bibtex fields are complete. Please provide all first and last names of all authros and
editors. It is always possible to let a \bibtex style insert ``et.\,al.'', but of course it is
impossible to have the \bibtex style guess what the names of the editors are if this information is
not provided. For names like ``von Stechow'', ``Van Eynde'', and ``de Hoop'' make sure that these
family names are contained in curly brackets. These authors will then be cited as
\citet{VanEynde2006a} and \citet{vonStechow84a}.

The references in your \bibtex file will be typeset correctly automatically. So, provided the
\bibtex file is correct, authors do not have to worry about this. But there are some things to
observe in the main text. Please cite as shown in Table~\ref{tab-citation}.

\begin{table}[htbp]
\caption{\label{tab-citation}Citation style for \lsp}
\centerline{%
\begin{tabular}{lp{9cm}}\hline\hline
citation type & example\\\hline
author & As \citet[215]{MZ85a} have shown\\
       & As \citet[215]{MZ85a} and \citet{Bloomfield33a} have shown\\
work   & As was shown in \citew[215]{Saussure16a}, this is a problem for theories that \ldots\\
work   & This is not true \citep{Saussure16a,Bloomfield33a}.\\\hline\hline
\end{tabular}
}
\end{table}
\nocite{Bresnan82b}% something with an editor.

%% Table~\ref{tab-various-publication-types} provides examples for different publication types (book,
%% journal article, paper in an edited volume, and so on). Please refer to the bibliography at the end
%% of this book to see how the respective items are formated.

If you have an enummeration of references in the text as in \emph{As X, Y, and Z have shown}, please use
the normal punctuation of the respective language rather then special markup like `;'.

If you refer to regions in a text, for instance 111--112, please do not use 111f.\ or 111ff.\ but provide the
full information. 

\noindent
\inlinetodostefan{Say something about decapitalization. \url{http://tex.stackexchange.com/a/140071/12092}}

\section{Special terms}

If you refer to special terms, please use italics as in (\mex{1}):
\ea
I use the term \emph{nominative} for \ldots
\z

\section{Interpunction}

Please use interpunction consistently. If you use initial adverbial clauses, please use commas:
\emph{When referring to such nominatives, I use \ldots}.


\section{Academic \emph{we}}

Monographs and articles that are authored by a single author should use the pronoun \emph{I} rather
than \emph{we} as in \emph{As we showed in Section~3}.

\section{Edited volumes}

Papers in edited volumes should start with an abstract.


\section{Checklist}

The following is a general checklist for authors. Author who use \latex should also consult the
checklist for advanced authors/typesetters in Section~\ref{sec-check-typesetters}.



















%      <!-- Local IspellDict: en_US-w_accents -->
