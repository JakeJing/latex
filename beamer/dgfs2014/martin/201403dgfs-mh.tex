\documentclass[10pt]{beamer}
\usepackage{textcomp}
\useoutertheme{lsp}
\usepackage{lsptitle}  


\usepackage[utf8x]{inputenc}
\usepackage{textcomp}
%% -*- coding:utf-8 -*-

\usepackage[utf8x]{inputenc}

% \usepackage{german}
% \selectlanguage{USenglish}



% DD.MM.YY
\def\two@digits#1{\ifnum#1<10 0\fi\number#1}
\def\mytoday{\two@digits{\number\day}.\two@digits{\number\month}.\number\year}


\usepackage{xspace,multicol}
\newcommand{\latex}{\LaTeX\xspace}


\newcounter{lastpagemainpart}
\footnotesep0pt
\renewcommand{\footnoterule}{}
\usefootnotetemplate{
  \noindent
  \insertfootnotemark\insertfootnotetext}

\let\beamerfn=\footnote
\renewcommand{\footnote}[1]{%
\let\oldfnsize=\footnotesize%
\let\footnotesize=\tiny%
\beamerfn<\thebeamerpauses->{#1}%
\let\footnotesize=\oldfnsize}


\date{05.03.2014}

\usepackage{eurosym} 
% % \usepackage{ogonek}  % Dabrowska

% Irgendein Font definiert mir das \dh wieder über.
\renewcommand{\dh}{d.\,h.\xspace}        % was macht dh sonst??
\renewcommand{\centerline}[1]{\hfill#1\hfill\hfill\mbox{}}


 
\author[M. Haspelmath]{Martin Haspelmath}
\title{Warum die Linguistik Open-Access-Bücher braucht}
\date{05.03.2014}

\begin{document}

\lspbeamertitle 
 
\section{Einleitung} 
 
\frame{ 
\frametitle{Linguistik braucht Bücher} 
\begin{itemize}
 \item Wir machen nicht nur Aufsätze, sondern ganze Bücher
 \item  Wir machen auch gerne Sammelbände
 \item Nichtquantitative Wissenschaft braucht eben mehr Platz für Reflexion und Argumentation
\end{itemize} 
}

\section{Open Access}
  
  
\frame{ 
\frametitle{Linguistik-Bücher sollten open-access sein (1)} 
\begin{itemize}
 \item (ebenso wie auch Zeitschriften)
 \item  weil das Reader-pays-Modell für hochspezialisierte Inhalte ineffizient ist
 \item  Bibliotheken können kaum vorhersagen, ob sie ein Buch brauchen
 \item Individuen, die ein Buch brauchen, können es nicht einfach kaufen
 \item Auflagen sind in den letzten zwei Jahrzehnten von einst 500 auf 120 gesunken
 \item Preise für Bücher sind massiv gestiegen, die Verlagskonzentration nimmt ständig zu, die Profite der kommerziellen Verlage ebenfalls
\end{itemize} 
}

  
\frame{ 
\frametitle{Linguistik-Bücher sollten open-access sein (2)} 
\begin{itemize}
 \item Verbreitung ist durch das Internet superleicht geworden
 \item  Aufsätze und Bücher werden oft in der freien Manuskriptversion gelesen, oder bei Google Books
 \item Die gedruckte Version dient oft vor allem als Scanvorlage: E-Version des Autors -- Papierversion der Bibliothek -- E-Version für die Seminarteilnehmer
 \item offensichtlich ein absurdes System
 \item  warum hält sich das System so lange? 
\end{itemize} 
}

\section{Verlage}  
  
\frame{ 
\frametitle{\parbox{5cm}{\vspace{1em}Der Mehrwert eines\\  wissenschaftlichen\- Verlags}} 
\begin{itemize}
 \item im 19. und 20. Jahrhundert:
\begin{itemize}
 \item Produktion
 \item Verbreitung
 \item Reklame
 \item Reputation
\end{itemize}
 \item im 21. Jahrhundert
\begin{itemize}
 
 \item praktisch nur: Reputation
 \item denn Produktion, Verbreitung und Reklame kosten nichts oder wenig
\end{itemize}    
\end{itemize}    
  
}

\frame{ 
\frametitle{Warum hält sich das System?} 
\begin{itemize}
 \item Reputation ist der entscheidende Faktor
 \item aus Reputation entstehen Karrieren, und ohne Karrieren gäbe es keine Wissenschaft
 \item ein renommierter Verlag ist wichtiger als viele Leser
 \item ein renommierter Verlag sorgt für Leser, selbst wenn fast niemand das Buch kauft
 \item die traditionellen Verlage gehen nur sehr langsam Richtung open-access (momentan eigentlich nur: De Gruyter Open) 
\end{itemize} 
}
  
\frame{ 
\frametitle{\parbox{5cm}{\vspace{1em}Warum haben renommierte Verlage Renommee?}}
\begin{itemize}
 \item Weil sie so guten Service liefern?
 \begin{itemize}
  \item Nein, weil sie prominente Werke veröffentlichen 
 \item (ohne Chomskys "Syntactic Structures" gäbe es heute den Verlag De Gruyter Mouton nicht)
 \end{itemize}
 \item Renommee entsteht durch exklusive Assoziation mit anderen renommierten Inhalten
 \item Also können renommierte Wissenschaftler relativ leicht neue renommierte Marken schaffen, auch ohne traditionelle Verlage
\end{itemize} 
}
   
\subsection{Finanzierungsmodelle}   
\frame{ 
\frametitle{Finanzierungsmodelle (1)} 
\begin{itemize}
 \item Author-pays-Modell: Der Autor trägt die "Druckkosten" komplett (ca. EUR 15.000 bei SpringerOpen, ca. EUR 10.000 bei De Gruyter Open)
 \item Freemium-Modell: Die Bücher sind im Prinzip frei verfügbar, aber für spezielle Services (z.B. EPUB-Version) muss man zahlen
 \item Publisher-pays-Modell
\begin{itemize}
\item  
 Der Verlag trägt die Publikationskosten, mit dem Ziel, durch gute Bücher sein Renommee zu steigern 
 \item vielleicht "Marburg University Press"?
 \item Publikationskosten werden als Teil der Forschungskosten angesehen
 

\end{itemize} 
\end{itemize} 
} 

\frame{ 
\frametitle{Finanzierungsmodelle (2)} 
\begin{itemize}

 \item Nachteile von Author-pays:
\begin{itemize} \item Die Reputation des Verlags bestimmt die Höhe der Kosten; wenn ich ein sehr gutes Werk bei einem kommerziellen Verlag veröffentlichen, steigere ich dessen Reputation und zahle beim nächsten Mal mehr.
 \end{itemize} 
\item Vorteile von Publisher-pays:
\begin{itemize} \item Die begrenzten Mittel können nur für eine begrenzte Menge von Publikationen verwendet werden; dadurch entsteht automatisch Exklusivität und Renommee.
\end{itemize} \end{itemize} 
} 
 
\end{document}s