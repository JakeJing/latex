%% -*- coding:utf-8 -*-
\chapter{General information on Language Science Press}


\section{Background and motivation}

Language Science Press is a book imprint that publishes high-quality books in the field of academic linguistics. It was founded in 2013, growing out of the initiative ``Open-Access Books for Linguistics'' (OALI) that was started by Stefan Müller (and other linguists at FU Berlin) and joined by Martin Haspelmath. After its first launch in August 2012, it quickly found over 100 prominent supporters from various subfields of linguistics and a range of different countries.

The problem to which this initiative responded was the increasing cost of linguistics books, which has come to contrast more and more with the ease with which files can be shared \citep{MuellerOA}. Increasingly, it seems that most of what the traditional publishers add to the scientists' work is the prestige of an imprint label \citep{Haspelmath2012a}, but this is something that is ultimately created by the scientists as well.

Thus, we decided to found a new imprint (\lsp) dedicated to publishing high-quality books which exist primarily in electronic form. Printed copies will be available through print-on-demand services. This imprint will be owned and run by scholars, and neither authors nor readers will be charged. The required work (reviewing, proofreading, typesetting) will be organized and carried out by the scholars themselves.

Language Science Press is associated with the FU Berlin and is coordinated by Stefan Müller and Martin Haspelmath.


%% \section{Set Up and Responsibilities}

%% Language Science Press works with Open Monograph Press’s (OMP) management software.
%% (more details will follow later)

\section{Responsibilities}

All books published by Language Science Press appear in book series, which are managed by a Series
Editor (or a team of Editors). The Series Editors are in charge of the reviewing and production of
the books in their series. The overall coordination of the Press is in the hands of the Press
Coordinators Stefan Müller and Martin Haspelmath.

\subsection{Advisory board}



\subsection{Series and editorial boards}

\subsection{Open Monograph Press and ZEDAT/CEDIS}

\subsection{The library of the Freie Universität Berlin}


\section{Open access and licence}

All Language Science Press books are published with open access, i.\,e.\ they can be downloaded free of
charge. All rights (copyrights, translation rights) remain with the author. 

By default, \lsp books are published with a Creative Commons CC-BY licence\footnote{ 
  Currently \url{http://creativecommons.org/licenses/by/4.0/}, 16.02.2014.
} (see \citew{Shieber2012a} for details
of what this means and why it is the preferred licence for scientific papers and books). The CC-BY
license allows for free reuse of the material in the book, including commercial uses as for instance
edited volumes that contain parts of the book licensed under CC-BY. The only condition is that the
work is properly attributed to the author/authors. The CC-BY license guarantees maximal distribution
of the material.

In certain situations, a CC-BY license is not possible. For instance if \lsp publishes a translation of a
book that already appeared with another publisher. In such situations the books will be published
under the more restrictive CC-BY-ND license\footnote{
Currently \url{http://creativecommons.org/licenses/by-nd/4.0/}, 16.02.2014.
}, which forbids to change the material (NoDerivatives) and hence guarantees that
the rights of the original publisher are not violated by somebody translating the work back into the
original language and distributing the book commercially or non-commercially.


(more details will follow later)


%      <!-- Local IspellDict: en_US-w_accents -->
