\chapter{Introduction}\label{introduction}

%%%%%%%%%%%%%%%%%%%%%%%%%%
%%%%%% SECTION 1.1 %%%%%%%
%%%%%%%%%%%%%%%%%%%%%%%%%%

\section{Marked"=S coding}\label{coding}\is{alignment|(}

Syntactic typology traditionally distinguishes languages by the way they encode the single argument of an intransitive verb (S) compared to the more agent-like (A) and the more patient-like arguments (P) of a monotransitive verb \citep{Comrie:1978,Dixon:1979,Dixon:2010-1}. 
On this basis, the two main alignment types, namely \textsc{nominative"=accusative} and \textsc{ergative"=absolutive} (Figure~\ref{Alignment}), are dis\-tin\-guish\-ed. 
While\is{alignment!through case-marking|(}  the nominative"=accusative\is{alignment!nominative-accusative|(} pattern employs the same form for S and A (nominative case\is{case!individual forms!nominative}), the P argument receives a special form of encoding (so-called accusative\is{case!individual forms!accusative} case)\is{alignment!nominative-accusative|)}.  
In an ergative"=absolutive\is{alignment!ergative-absolutive|(} system,  S and P are coded alike (absolutive\is{case!individual forms!absolutive} case) while the A argument is in a special form (ergative\is{case!individual forms!ergative} case)\is{alignment!ergative-absolutive|)}.

\begin{figure}[ht] \centering \fbox{
\begin{picture}(265,95)
\put(10,75){\makebox(100,15)[l]{nominative"=accusative:}}

\put(45,45){\makebox(20,20){\textbf{S}}}
\put(15,15){\makebox(20,20){\textbf{A}}}
\put(75,15){\makebox(20,20){\textbf{P}}}

\put(85,25){\circle{20}}
 
\closecurve(40,60, 65,65, 60,40, 40,20, 15,15, 20,40)

\put(150,75){\makebox(100,15)[l]{ergative"=absolutive:}}

\put(185,45){\makebox(20,20){\textbf{S}}}
\put(155,15){\makebox(20,20){\textbf{A}}}
\put(215,15){\makebox(20,20){\textbf{P}}}

\put(165,25){\circle{20}}
 
\closecurve(190,40, 185,65, 210,60, 230,40, 235,15, 210,20)
\end{picture}}

\caption{Nominative"=accusative vs. ergative"=absolutive alignment}\label{Alignment} \end{figure}\is{alignment|)}\is{alignment!through case-marking|)} 

In\is{marked-S languages!formal definition|(} most languages, overt formal marking is employed for the P argument in nominative"=accusative\is{alignment!nominative-accusative} languages and the A argument in ergative"=absolutive\is{alignment!ergative-absolutive} languages, while the relation including the S argument (i.e. S+A for the former and S+P for the latter type) is typically left zero-coded\is{markedness!overt versus zero-coding}.\footnote{The label `unmarked' is often used for the case-form lacking overt morphological marking. 
However, this terminology is problematic because the term `unmarked' is used for a variety of concepts in linguistics \citep{Haspelmath.mark:2006}. 
I follow Haspelmath's proposal to use the term `zero-coded' for the formal manifestation of unmarkedness.}  
This tendency was prominently phrased in \citeauthor{Greenberg:1963}'s Universal 38:

\begin{quote} Where there is a case system, the only case which ever has only zero allomorphs is the one which includes among its meanings that of subject of the intransitive verb.
\citep[75]{Greenberg:1963}
\end{quote}  

However, there are clear exceptions to this generalization. 
I will use the term \textsc{marked"=S language} in the following to refer to those exceptional languages. 
More precisely, there are two types of marked"=S languages: \textsc{marked"=nominative} and \textsc{marked-absolutive}. 
Both have in common the property that the single argument of intransitive verbs is overtly-coded while one of the transitive arguments (A or P) receives zero-coding.\footnote{No examples are known to me of a language using zero-coding for both A and P but overt coding for the S argument -- a pattern which would by my definition be included in the group of marked"=S languages. 
Given the overall rarity of horizontal alignment (i.e. in which A and P are treated alike but differently from S) and the rarity of marked"=S systems, it does not come as a surprise that the combination of both rarities is not (yet) found.} 
%\textcolor{red}{Koenig:2008 p.9 refers to Donohue & Brown 1999 and Plank's Rarit\"atenkabinet for the claim that Chukchi absolutive singular nouns (encoding S) are the morphologicaly most complex, I have not found this citation yet, still needs to be checked}
This study presents an in-depth survey of marked"=S languages.\is{marked-S languages!formal definition|)}

I will begin this first chapter by defining marked"=S languages, which constitute a rare and somewhat unexpected type of encoding grammatical relations (Section~\ref{definition}). 
Following this introduction, a brief digression will be made to discuss the phenomenon known as grammatical markedness and the different usages of the term (Section~\ref{markedness}). 
Then I will address the issue of terminology used in describing the case-forms of a nominal in a language of the marked"=S type. 
Since a wide range of different case-terms\is{caseterminology} are used in the descriptions of marked"=S languages. 
In order to assure consistency within this work, I will employ a uniform set of case-terms. 
In addition, I will propose a new terminology to be employed when comparing\is{case!comparative concepts} marked"=S languages with each other (Section~\ref{label}). 
Following that, I will discuss the explanations and types of explanations given to justify the existence of this rare type of case-system (Section~\ref{explain}). 
These explanations will be grouped into two types: historical and functional explanations.
The subsequent section discusses marked"=S systems from the point of view of formal linguistic theories. 
Not only do marked"=S languages constitute a typological exception, they also pose a serious problem for various formal theories of case-marking, as will be demonstrated using the example of Lexical Decomposition Grammar (Section~\ref{theoretical}).
Alignment is most prominently associated with nominal case-marking, and this study is likewise restricted to marked"=S alignment as it is found in this domain. 
However, the term `alignment' is also used to refer to verbal agreement and word order, as well as with reference to behavioral properties of nominals. 
Marked"=S coding in those other domains, or the domain-specific counterparts thereof, will briefly
be discussed in Section~\ref{domains}.\enlargethispage{\baselineskip}
Finally, I will give an outline of the remainder of this study in Section~\ref{outline}.

%%%%%%%%%%%%%%%%%%%%%%%%%%%%%%%%%%%
%%%%%%%%%% SECTION 1.2 %%%%%%%%%%%%
%%%%%%%%%%%%%%%%%%%%%%%%%%%%%%%%%%%

\section{Definition}\label{definition}

Marked"=S languages are more traditionally known by the name \textsc{marked"=nominative} languages\is{alignment!marked-nominative}. 
I have chosen this new term in order to allow the inclusion of data from a related, yet not so widely recognized, phenomenon, namely \textsc{marked"=absolutive} languages. 
The term \textit{marked"=S} also nicely summarizes the central characteristic of this type of language, namely the overt marking found on the single argument of intransitive verbs (S) combined with a zero-coded A or P argument. 

The marked"=nominative\is{alignment!marked-nominative|(}  type is the most frequent manifestation of the marked"=S coding type. 
As a subtype of the nominative"=accusative alignment system, marked"=nominative languages exhibit the basic pattern exemplified for this system in Figure~\ref{Alignment}. 
In marked"=nominative systems -- like in standard nominative"=accusative -- S is aligned with A and opposed to P. 
Unlike in the standard system, the P relation is left without any formal encoding of its case relation. 
It is in the zero-coded form, while the S+A relation (the nominative) has an overt morphological marker (cf. Figure~\ref{MN alignment}). 
In contrast, the standard -- `unmarked' -- nominative"=accusative system uses overt marking either for both S+A (nominative) and P (accusative) or restricts it to P-marking. 

\begin{figure}[htbp]\centering\fbox{
\begin{picture}(225,70)
%\put(10,72){\makebox(90,15)[l]{Marked nominative:}}

\put(95,40){\makebox(20,20){\textbf{S}}}
\put(65,10){\makebox(20,20){\textbf{A}}}
\put(125,10){\makebox(20,20){\textbf{P}}}
 
\closecurve(90,55, 115,60, 110,35, 90,15, 65,10, 70,35)

\put(3,30){\makebox(65,15){overt coding}}
\put(65,37){\vector(1,0){25}}

\put(150,30){\makebox(55,15){zero-coding}}
\put(168,33){\vector(-2,-1){28}}

\end{picture}}
\caption{Marked"=nominative coding}\label{MN alignment}
\end{figure}

The two types of coding are illustrated here by some examples.
The Turkish\il{Turkish} data in (\ref{Turkish}) exemplify the standard nominative"=accusative system with a zero-coded Nominative\footnote{Throughout the text, I follow the convention of capitalizing case-labels, when referring to a specific case-form in an individual language.}  case and an overtly coded Accusative\is{case!individual forms!accusative} case-marked by the suffix \emph{-\i}. 
\enlargethispage{2\baselineskip}

\begin{exe}\ex\label{Turkish}\langinfo{Turkish}{Altaic; Turkey}{own knowledge}
\begin{xlist}
\ex\label{TurkishS}\gll Adam gel-di\\
man.{\nom{}} come-\pst{}\\
\glt `The man came.'
\ex\label{TurkishAP}\gll Ö\v{g}retmen adam-\textbf{\i}{} gör-dü\\
teacher man-\acc{} see-\pst{}\\
\glt `The teacher saw the man.'
\end{xlist}
\end{exe}

The marked"=nominative type is illustrated by examples from Cocopa\il{Cocopa}. 
The S and A relation in (\ref{Cocopa}a, b) is marked with the overt Nominative suffix \emph{-c} while the P relation in (\ref{CocopaAP}) is left zero-coded.

\begin{exe}\ex\label{Cocopa}\langinfo{Cocopa}{Yuman, Hokan; Mexico}{\citealp[183, 186]{Crawford:1966}}
\begin{xlist}
\ex\label{CocopaS}\gll\textipa{ap\'a\textbf{-c}} \textipa{aw-y\'a}\\
man-{\nom{}} 3-know\\
`The man knows.'
\ex\label{CocopaAP}\gll \textipa{ap\'a\textbf{-c}} \textipa{k\super{w}\'ak} \textipa{pa:-\textsubdot{t}\'Im}\\
man-{\nom{}} deer.{\acc{}} 3>3-shoot\\
`The man shot the deer.'
\end{xlist}
\end{exe}\is{alignment!marked-nominative|)} 

\is{alignment!marked-absolutive|(}A parallel system to the marked"=nominative type exists for ergative"=absolutive languages -- the marked"=absolutive type. 
Again, the marking relations are reversed from the standard type. 
The standard ergative"=absolutive system overtly codes the A relation (ergative\is{case!individual forms!ergative}) and leaves the S+P relation (absolutive\is{case!individual forms!absolutive}) zero-coded. 
Conversely, in marked-absolutive languages one finds overtly coded S+P and zero-coded A. The marked-absolutive system is illustrated in Figure~\ref{MA-align}. 
Just like the marked"=nominative, the marked-absolutive contradicts Greenberg's universal quoted above since it has overt marking of the S argument while zero-coding one of the transitive core relations. 

\begin{figure}[ht]
\centering
\fbox{
\begin{picture}(225,70)
%\put(10,72){\makebox(90,15)[l]{Marked absolutive:}}

\put(95,40){\makebox(20,20){\textbf{S}}}
\put(65,10){\makebox(20,20){\textbf{A}}}
\put(125,10){\makebox(20,20){\textbf{P}}}
 
\closecurve(100,35, 95,60, 120,55, 140,35, 145,10, 120,15)

\put(3,35){\makebox(65,15){zero-coding}}
\put(53,38){\vector(1,-1){17}}

\put(153,27){\makebox(55,15){overt coding}}
\put(150,35){\vector(-1,0){30}}
\end{picture}}
\caption{Marked"=absolutive coding}\label{MA-align}
\end{figure}

To illustrate the two patterns, I provide examples for ergative"=absolutive and marked-absolutive coding below.
In Chechen\il{Chechen}, the S and P relation is zero coded (\ref{Chechen}a, b, c) while the A relation in (\ref{Chechen}b, c) is overtly coded by the Ergative\is{case!individual forms!ergative} suffix \emph{-(a)s}.
In addition to the ergative"=absolutive case-marking, verbal indexing\is{verbal indexing} also has an ergative"=absolutive basis. 
The verb agrees in gender with the S and P argument. 

\pagebreak
\begin{exe}\ex\label{Chechen}\langinfo{Chechen}{Nakh-Daghestanian; Chechen Republic}{Zarina Molochieva, p.c.}
\begin{xlist}\ex\label{ChechenS}\gll naana baazar j-ax-na\\
mother(\fem{}) market.ADV \fem{}-go-\prf{}\\
\glt `The mother went to the market.'

\ex\gll k'ant-\textbf{as} naana lie-j-i-na\\
boy(\mas{})-{\erg{}} mother(\fem{}) lie-\fem{}-make-\prf{}\\
\glt `The boy has lied to the mother.'

\ex\gll naana-\textbf{s} k'ant liicna-v-i-na\\
mother(\fem{})-{\erg{}} boy(\mas{}) wash-\mas{}-make-\prf{}\\
\glt `The mother has bathed the boy.'
\end{xlist}
  \end{exe}

The only known straightforward example of a marked"=absolutive case-system so far is attested in Nias\il{Nias}.\footnote{\citet{Crysmann:2009}, using a quite different definition of marked"=absolutive language, argues against classifying Nias\il{Nias} as such. 
In his approach, alignment is not considered to be construction specific as it is in my account and many other recent works that study alignment from a cross-linguistic perspective \citep[cf. among others][]{Bickel.align}. 
I will discuss this the notion in more detail in Section~\ref{domains} and Chapter~\ref{method}.} 
While the S argument in (\ref{NiasS}) is in the overtly marked Absolutive\is{case!individual forms!absolutive} case, the A argument is in the zero-coded Ergative\is{case!individual forms!ergative} form of a noun (\ref{NiasAP}). 
As illustrated in (\ref{NiasP}), the P relation is also encoded in the overtly coded Absolutive\is{case!individual forms!absolutive} form.\footnote{The distinction between the Absolutive and Ergative case-form is not always as straightforward as in example (\ref{Nias}a, c). 
In most cases the difference is marked by initial consonant mutation, see Section~\ref{morphophon} for a brief discussion of the morphophonemics of nominal mutation\is{case-marking!via nominal mutation} in Nias\il{Nias}.}
  
\begin{exe}\ex\label{Nias}\langinfo{Nias}{Sundic, Western Malayo Polynesian, Austronesian; Sumatra, Indonesia}{\citealp[343]{Brown:2001}, Lea Brown, unpublished fieldwork data from 2003,  \citealp[346]{Brown:2001}}
\begin{xlist} 
\ex\label{NiasS}\gll aukhu \textbf{n-}idan\"o\\
{\stat}.hot {\abs}-water\\
\glt `The water is hot.'
\ex\label{NiasAP}\gll i-f-o-houu	defao 	\textbf{idan\"o} 	nasi\\
3\sg{}-\caus{}-have-rust	iron.\abs{}	water.\erg{}	sea.\abs{}\\
\glt `The seawater rusted the iron'
\ex\label{NiasP}\gll  la-bunu \textbf{m-}ba\ss i\\
3\pl{}.{\rls}-kill \abs{}-pig\\
\glt `They killed a pig.'
\end{xlist}
\end{exe}
\is{alignment!marked-absolutive|)}

In\is{marked-S languages!functional definition|(} addition to this definition of marked"=S coding, which is purely based on the absence versus presence of overt case-marking\is{marked-S languages!formal definition}, there is a second definition of marked"=S languages. \citet{Koenig:2006} distinguishes between Type 1 and Type 2 marked"=nominative languages. 
Type 1 languages are classified by the criterion I have discussed above, namely the overt coding of the nominative case-form and the zero-coding of the accusative\is{case!individual forms!accusative}. 
Type 2 languages have overtly coded case-forms for both nominative and accusative. 
However, the accusative\is{case!individual forms!accusative} has a wider range of functions; it is for example the form of a noun used in citation. 
I will discuss this second definition of marked"=S coding in Section~\ref{functional}.\is{marked-S languages!functional definition|)}

%%%%%%%%%%%%%%%%%%%%%%%%%%
%%%%%% SECTION 1.3 %%%%%%%
%%%%%%%%%%%%%%%%%%%%%%%%%%

\section{Markedness in grammar}\label{markedness}\is{markedness|(}

The term \textit{markedness}, and more often the statement that a certain linguistic feature is \textit{marked}, are often employed in grammatical descriptions.
Markedness, in its most basic meaning, is universally associated with the presence of overt material, such as overt case morphology for example. 
Other aspects that are often also called \textit{markedness} frequently correlate with formal markedness (i.e. the presence of overt material). 
These other factors include (often interrelated) phenomena such as restriction in use, late and/or cumbersome acquisition of a structure, specialization in meaning, or a low usage frequency of an item. 
Even if not correlated with formal markedness in any sense, structures that meet any of these additional criteria may be referred to as marked by many linguists. 
\citet{Haspelmath.mark:2006} discusses the different meanings of the term markedness and distinguishes between twelve basic senses. 
These twelve senses are roughly grouped into four types that view markedness as either complexity, difficulty, abnormality or a multidimensional correlation. 

The notion of markedness has a prominent position in both functionally- and formally-oriented linguistic traditions. 
In the functional tradition, this notion goes back to the work of Roman Jakobson\aimention{Jakobson, Roman} and other members of the Prague School, while in the formal tradition the notion was made prominent by Noam Chomsky\aimention{Chomsky, Noam}.
In both cases the concept of markedness appears to have been inspired by phonological work.
\citet{Battistella:1996} provides a detailed discussion of the understanding and evolution of the term markedness in the Jakobsoninan and Chomskyan traditions. 

With respect to marked"=S languages, the term \textit{marked} is principally understood as overt coding. 
As mentioned at the end of the previous section, an extension of the term marked to other criteria can also be seen for marked"=nominative languages, namely in the definition of Type 2 marked"=nominative languages proposed by \citet{Koenig:2006}. 
In the following, I will attempt to make clear which one of the different definitions of markedness is presently being discussed.
When referring to the form-based definition of markedness, I will use the terms \textsc{overtly coded} or respectively \textsc{zero-coded}. 
Only when directly quoting the work of other authors, the terms `markedness', `marked' and  unmarked' may be used for describing forms that differ in term of more versus less overt material (i.e. the form based criterion)\is{markedness|)}. 


%%%%%%%%%%%%%%%%%%%%%%%%%%%%%%%%%%
%%%%%%%%%% SECTION 1.4%%%%%%%%%%%%
%%%%%%%%%%%%%%%%%%%%%%%%%%%%%%%%%%

\section{Case-labels}\label{label}\is{case!terminology|(}

When describing the case-system of a given language, linguists often apply traditional case-terminology familiar from Latin. 
If a case does not resemble any case in the Latin case-system, linguistic theory today provides a huge arsenal of Latinate case-labels to be employed \citep[cf.][]{Haspelmath:2009}. 
This practice of reusing terms is not uncontroversial, since the range of functions or meanings of case-forms will virtually never coincide between any two languages. 
Marked"=S languages are a prime example of this variation of functions. 
Their nominative/absolutive and ergative/accusative forms show properties quite unlike the properties of case-forms that are known by the same name in standard nominative"=accusative or ergative"=absolutive languages.
This has led many scholars working on marked"=S languages to abandon the traditional labels.
Yet many of the alternate labels they came up with are equally inappropriate. 
In this section, I will discuss the different approaches for labeling cases in marked"=S languages.
Furthermore, I will introduce the case-terminology to be employed in the remainder of this book.

As described in the definition of marked"=S coding above (Section~\ref{definition}), the special property of the marked"=S system is that it combines the standard alignment of nominative"=accusative or ergative"=absolutive languages with an unexpected pattern of overt/ non-overt marking of the nominal cases. 
The decision between choosing a label according to the function of a form or according to the overt marking relations is also the main problem when it comes to finding appropriate names for the individual cases defining this alignment system. 
   
One possibility is to simply use the labels from the standard nominative"=accusative and ergative"=absolutive systems.
So\is{case!individual forms!nominative|(}, the term `nominative' is used for the S+A relation in standard nominative"=accusative as well as in marked"=nominative systems. 
Similarly, the label `absolutive\is{case!individual forms!absolutive}' refers to the S+P relation in standard ergative"=absolutive and in marked-absolutive systems. 
However, the more problematic issue is how to label the zero-coded case in the marked"=S systems.  
Not only do the terms `accusative\is{case!individual forms!accusative}' and `ergative\is{case!individual forms!ergative}' suggest overt marking of the case relation \citep[56--57]{Dixon:1994}, the uses of the zero-coded case-forms also go beyond those of the accusative or ergative as found in the standard versions of these alignment systems. 
Nonetheless the  label `accusative\is{case!individual forms!accusative}' is used in the description of a number of marked"=nominative languages, namely Maa\il{Maa} \citepalias{Tucker:1955}, Murle\il{Murle} \citep{Arensen:1982} and K'abeena\il{K'abeena} \citep[85--86]{Crass:2005}. 
The labels of `nominative' and `accusative\is{case!individual forms!accusative}' are also employed by \citet{Koenig:2006,Koenig:2008} in her overview of marked"=nominative languages in Africa.

Differently, \citet{Dixon:1979} proposed the term `extended ergative' for the marked nominative case to reflect its overt marking, which is parallel to the mostly overtly marked ergative\is{case!individual forms!ergative} in ergative"=absolutive systems. 
The label `extended accusative' for the marked-absolutive would be analogous to this label, though he does not propose this term since he disputes the existence of marked-absolutive systems altogether. 
The `extended ergative', however, did not make it into the terminology of grammar writers.\footnote{In theoretical typology, the term has done a little better. 
\citet{Plank:1985} uses the terms `extended ergative/restricted absolutive' to refer to the marked"=nominative system and also `extended accusative/restricted nominative' for the marked absolutive pattern.} 
If discussed at all, it is merely mentioned as a possible alternative label, for example by \citet[133]{Wegener:2008}, who irrespective of this calls the Savosavo\il{Savosavo} subject-marker `Nominative'. 
Dixon appears to have had a change of mind on the appropriate terminology for marked"=nominative systems, as he concludes in his \citeyear{Dixon:1994} work that ``it seems wisest to maintain the standard use of ergative to refer to marking just of A function'' \citep[64]{Dixon:1994} and proposes to stick to the term marked"=nominative after all.  

One strategy that avoids the terminological problem altogether is not to use any of the traditional Latinate case names at all. 
Most often this results in labels such as `subject-case' and `object-case' (or just `subject-marker'). 
This approach is chosen by the grammar writers of some African languages, e.g. for Borana Oromo \citep[34]{Stroomer:1995} or Gamo\il{Gamo} \citep[364]{Hompo:1990}. 
Most commonly it has been applied to languages of the North-American West-Coast -- especially in the 60s and 70s of the last century. 
This is the case for the Yuman languages Cocopa\il{Cocopa} \citep[104]{Crawford:1966}, Diegue\~no\il{Diegue\~no (Mesa Grande)} \citep[151]{Langdon:1970}, Mojave\il{Mojave} \citep[18]{Munro:1976}, Yavapai\il{Yavapai} \citep[68]{Kendall:1976} and Hualapai \citep[38]{Watahomigie:2001}. 
The same is true for the non-related language Wappo\il{Wappo} \citep*[90]{Lietal:1977}. 
However, in their grammar of Wappo\il{Wappo} -- published about thirty years later -- the same authors have switched from the term `subject-marker' to the use of \textit{nominative} \citep*{Thompsonetal:2006}. 
The labels `subject-case' and `object-case' have the disadvantage that they carry theoretical connotations unrelated to marked"=S case-marking. 
Not all nouns bearing the overt marking of subjects might be subjects in a syntactic sense, and not all syntactic subjects might have the subject case-marker in a certain language (unless one establishes overt case-marking as the only defining property of subjects in that language). 

Drawing very much on the Latin grammar tradition, \citet{Melcuk:1997} makes an idiosyncratic proposal for labeling compared to the current practice in descriptive linguistics. 
For Maa\il{Maa}, he suggests to stick to the literal translation of the term `nominative' as `the naming case' and therefore proposes to use it for the form used in naming a nominal (i.e. the citation form\is{citation form}). 
Apart from the use as citation form, this case also encodes the P argument of a transitive verb (`Accusative'\is{case!individual forms!accusative} case in other descriptions of Maa\il{Maa}). 
The subject-marking case he relabels as `Oblique' in turn. 
This usage of the term Nominative may be well motivated from a historical and etymological perspective as \citet[453]{Creissels:2009} points out. 
Still, if used for the case-form encoding transitive objects, the term Nominative is bound to give rise to confusion. 
Beyond being extremely confusing, there is -- in my opinion -- a major problem with this approach, % to take the literal translation of the case name as the basis for picking the case-form to stick the lable nominative on. 
namely, that the etymological meaning is not the meaning most prominently associated with the nominative. 
It is rather its function as the `subject-case' that comes to mind first, maybe along with the function as `default' or `elsewhere' case for some linguists.\footnote{The notion of default case is discussed in more detail in Section~\ref{theoretical}.} 
Both functions are fulfilled by \citeauthor{Melcuk:1997}'s `Oblique' case and not his `Nominative'.
%\citet{Creissels:2009} makes a similar proposal to the labeling of cases

The terminologies traditionally employed for the marked"=nominative languages of Eastern Africa are far less confusing. 
`Nominative' is used in the traditional way as referring to the S+A relation.\is{case!individual forms!nominative|)} 
To\is{case!individual forms!absolute|(} account for the special status of the form used for the P relation, this form is not referred to as `accusative\is{case!individual forms!accusative}' but as the `absolute' case -- e.g. in Turkana\il{Turkana} \citep{Dimmendaal:1982} or Datooga\il{Datooga} \citep{Kiessling:2007}. 
The same terminology of `Absolute' and `Nominative' case was also introduced in an early description of the Yuman language Yuma \citep[210]{HalpernYuma3}, although it did not catch on in the terminology of this genus, as noted above. 
Also \citet[456]{Creissels:2009} proposes `absolute' as a label for nouns in extra-syntactic function such as citation forms, which tend to be zero-coded morphologically (all these function are covered by the typical East-African absolute).
Some linguists also use the term `absolutive' (rather than absolute). 
This is attested, for example, in the description of Harar\il{Oromo (Harar)} Oromo by \citet{Owens:1985}. 
This, however, might lead to confusion with the S+P relation in ergative"=absolutive languages. 
Therefore absolute should be preferred as a label. 
Finally, \citet[24]{Koenig:2008}, in her discussion of the marked"=nominative case-terminology, notes that the term `absolute' might also lead to confusion since it is used for the zero-coded Nominative\is{case!individual forms!nominative} case-form in Turkish\il{Turkish}. 
This leads her to use the term `accusative\is{case!individual forms!accusative}' also for marked"=nominative languages.\is{case!individual forms!absolute|)} 

Given all these different traditions and approaches to naming cases in marked"=S languages, one is bound to get mixed up in the terminology when comparing data from different languages. 
In order to spare the reader from confusion by changing glosses from one example to the next, I decided not to stick to the case-labels chosen by the linguists working on the individual languages. 
Instead, I will change the glosses with the goal of achieving maximum transparency. 
All examples from marked"=nominative languages are uniformly named and glossed as \textsc{nominative}\is{case!individual forms!nominative} and \textsc{accusative}\is{case!individual forms!accusative}. 
I have chosen this convention for the following reasons.
There seems to be a certain trend towards recognizing the overt subject-marker as parallel to the nominative case-marker in any standard nominative"=accusative language. 
This trend is indicated by \citeauthor{Thompsonetal:2006}'s (\citeyear{Thompsonetal:2006}) change of terminology as well as Dixon's change of mind concerning the `extended ergative' vs. `marked"=nominative' terminology. 
Moreover, this proposal involves the least amount of relabeling of case-forms and thus makes going back to the original sources less prone to requiring terminological adjustments. 

`Accusative\is{case!individual forms!accusative}' and `absolute\is{case!individual forms!absolute}' both appear to be good choices for the non-no\-mi\-na\-tive case in marked"=S languages. 
The encoding of transitive P arguments is just one of many functions the zero-coded case-form fulfills in marked"=nominative languages, as I will demonstrate in the following (Chapters~\ref{nompred}--\ref{extrasyn}). 
The label `accusative\is{case!individual forms!accusative}' is traditionally associated with a case-form with the main function of encoding P arguments. 
Using this label for a case-form that has a variety of additional functions may lead to mild confusion on the first encounter with examples in which the accusative argument is clearly not an object\is{grammatical relation!object} of any kind. 
This study aims at exploring the functions of the different case-forms in marked"=S languages, and thus wants to draw attention of those functions of the object-case in marked"=nominative languages that are unusual compared with standard nominative"=accusative languages. 
In contrast, the label `absolute' is less familiar and might be mixed up with `absolutive' and thus lead to the wrong impression that one is dealing with an ergative"=absolutive language. 
In the remainder of this study I will use the term `accusative\is{case!individual forms!accusative}' to refer to the case-form that (among other functions) encodes transitive P arguments in marked"=nominative languages. First, I do so because of the greater familiarity of the term `accusative' over `absolute'. 
Also, as noted above, the range of uses of the object case-form in marked"=nominative languages is a central aspect of this study. 
Therefore, unexpected occurrences of accusative\is{case!individual forms!accusative} case, from the standpoint of the more widely known standard nominative"=accusative system, are meant to be highlighted here.

For the marked"=absolutive system, we are not confronted with such hard decisions about terminology, since one is not faced with any differing traditions of labeling. 
\citet{Brown:2001} uses the terms \textsc{Unmutated} and \textsc{Mutated} form of the noun for Nias\il{Nias}. 
These labels refer to the morphophonemic shape of the nouns in question. 
The Mutated form covers the absolutive\is{case!individual forms!absolutive} (i.e. S+P) function, while the Unmutated form is used for A arguments among others. 
Since Nias\il{Nias} is the only language with a marked-absolutive system in my study, I will adopt the language-specific terms Mutated and Unmutated form referring to the S+P relation (the absolutive\is{case!individual forms!absolutive}) and A relation (the ergative\is{case!individual forms!ergative}), respectively. 

The previous discussion has dealt with the issue of labeling cases in individual languages.
In addition, terminology is needed to make general statements about the overtly coded and zero-coded forms in both marked"=nominative and marked-absolutive languages. 
For this purpose I propose the following terminology, which will be employed throughout the study whenever making comparative statements.  
When referring to marked"=S languages I use the terms \textsc{S-case}\is{case!comparative concepts!S-case} and \textsc{zero-case}\is{case!comparative concepts!zero-case} form. 
The label S-case refers to the case which includes among its functions that of encoding the single argument of an intransitive verb (overtly coded in all languages under investigation by definition). 
This is the nominative (S+A) in marked"=nominative languages and the absolutive\is{case!individual forms!absolutive} (S+P) in Nias\il{Nias}. 
The zero-case on the other hand refers to the P argument in marked"=nominative languages and the A argument in marked-absolutive languages. 
This case-form is expressed by zero-morphology in the overwhelming majority of marked"=S languages.\is{case!terminology|)}  
%I could have used these terms on all levels of description (the language level, and the intermediate levels generalizing over a subtype of marked"=S language). However, for the benefit of readability I stick to the more common and transparent terms. 
      
%The marked nominative and marked absolutive types of language pose a problem to the descriptive linguist, when it comes to assigning a label to a case. Aside of being the case associated with S+A marking (or the `subject' case) the term nominative  has some additional connotations. Taking literally, the Latin Nominative is the `naming case', and does hence imply the use as a citation form. Also the nominative is considered the `default' or `elsewhere' case by many theories, which implies a wide range of uses.  


%%%%%%%%%%%%%%%%%%%%%%%%%%
%%%%%% SECTION 1.5 %%%%%%%
%%%%%%%%%%%%%%%%%%%%%%%%%%


\section[Explaining the existence of marked"=S]{Explaining the existence of marked"=S}\label{explain}

\subsection{Rare and geographically skewed}
Languages of the marked"=S type are a typological exception. 
Their occurrence is unexpected, a view expressed for example by Greenberg's (1963) Universal 38 cited above. Not surprisingly, languages of this type are extremely rare and their occurrence is geographically highly skewed. 
The main locus for marked"=S languages is in North-Eastern Africa \citep[138]{Koenig:2008}. 
Apart from the cluster in Africa, the pattern is also found in the Yuman genus of southwestern North America and a few other languages of that region, as well as in some languages of the Pacific region. 

To account for the existence of this unexpected case-system, two types of explanations have been put forward, a historical one and a functional one. 
The historical explanation describes how the marked"=S system can evolve from one of the more widespread systems. 
This explanation has been put forward for languages of the marked"=nominative type, and considers them to be an extended variant of ergative"=absolutive systems with overtly coded ergative case-marking, which has been extended to cover the S relation. 
Apart from ergative"=absolutive systems a variety of other sources have been suggested for marked"=nominative languages. 
What all of these historical scenarios have in common is that they propose an origin for the nominative from a category which is expected to be overtly coded from a cross-linguistic perspective.  
The second type of explanation draws on the number of other functions the case-forms in the marked"=S system cover beyond S, A and P marking. 
This theory predicts that the overall distribution of the zero-coded form will be broader than the distribution of the overtly coded form if one takes into consideration other functions, such as marking of attributive possessors, marking of predicate nominals etc.
Of course, these two types of explanations are more different points of view than mutually exclusive approaches.\footnote{I have to thank Eitan Grossman for reminding me of this.} 
Historical change in a language can certainly be explained by functional motivations in many cases.
Some scholars would even argue that functional motivations are the ultimate explanation for all language change \citep{Keller:1994,Dubois:1987,Croft:2000}.

While both types of explanations try to account for the fact that marked"=S languages occur in the first place, the two approaches fall short of explaining the rarity of the phenomenon.
In this section, the two lines of argumentation will be discussed in more detail starting with the historical explanation in Section~\ref{historical}, followed by the functional motivation in Section~\ref{functional}.  


\subsection{Historical explanations}\label{historical}\is{marked-S languages!historical explanations|(}

A prominent advocate of the historical explanation of marked"=S systems is \citet{Dixon:1979,Dixon:1994}.
He defines marked"=nominative languages purely on the basis of the contrast between overt and zero-coding \citep[76ff.]{Dixon:1994}. 
Yet, he also tries to give an explanation for the existence of these typologically rare languages.
Dixon\is{grammatical relations!subject|(} argues that marked"=nominative languages exist because ergative"=absolutive languages constitute a somewhat unsatisfactory case-system in neglecting the `universal concept' of subject and might eventually amend for this by extending the use of A-marking to S.

\begin{quote} The extension of `marked A case' can be explained in terms of the universal syntactic-semantic identification of A and S as `subject'
\citep[78]{Dixon:1979} \end{quote} 

Along\is{alignment!marked-absolutive|(} the same line of argumentation, Dixon denies the existence of marked"=absolutive languages.
Since, according to him, nominative"=accusative languages are more well-formed in this respect, there is no need for overt P-marking to extend its use to mark S.
There is thus no reason why marked-absolutive languages should emerge in the first place.\is{grammatical relations!subject|)}

\begin{quote} 
There is a more slender semantic link between O  and S, so that the fourth logical possibility---`marked O case' being extended to also cover S---appears not to occur.\footnote{Dixon's `O' corresponds to `P' in the terminology used here.} \citep[78]{Dixon:1979}  \end{quote}\is{alignment!marked-absolutive|)}

An ergative\is{case!individual forms!ergative} origin of the marked"=nominative coding-system has also been suspected by linguists confronted with individual languages of this type. 
\citet{Lietal:1977} analyze the Wappo\il{Wappo} marked"=nominative system as a recent innovation. 
They vaguely hint that the overt subject-marker might be a trace of an earlier ergative stage ``where the absolutive case was unmarked and the modern Wappo\il{Wappo} \emph{-i} was the ergative case-marker that became generalized into a subject-marker'' \citep[98]{Lietal:1977}. 
However, they elaborate an alternative pathway for the means of encoding grammatical relations in modern Wappo\il{Wappo}, which might be in conflict with the hypothetical ergative\is{case!individual forms!ergative} in pre-modern Wappo\il{Wappo}. 
Their main argument for the innovative status of the Wappo\il{Wappo} marked"=nominative is its absence from subordinate clauses and equational sentences. 
Both sentence types have a rigid SOV word order, while main clauses (of the non-equational type) are more flexible in the ordering of constituents. 
Since ``subordinate clauses are known to be more conservative than main clauses in preserving'' word order -- citing \citet{Lehmann:1974} and \citet{Vennemann:1975} on this -- they conclude that Wappo\il{Wappo} must be moving from a stage where grammatical relations were encoded by word order to a stage where this is done via case-marking \citep[100]{Lietal:1977}. 
So basically they propose a change from word order to case-marking, on the one hand, and a change within the case-marking system (namely from ergative"=absolutive to marked"=nominative), on the other hand. 
Of course, it would be possible that the two events took place sequentially. 
First, the word order-based system changed to an ergative case-marking system with a freer word order, which then in turn became a marked"=nominative system. 
Yet this is a highly speculative proposal, which cannot be backed up by any historical records of the language. 
The first records on the Wappo\il{Wappo} language date back to the early twentieth century and at this time the marked"=nominative system was already established, as the first grammar by \citet[131]{Radin:1929} shows. 
Since doubt is cast on the classification of Wappo\il{Wappo} being closely-related to the Yukian languages by \citet{Sawyer:1980}, a comparison with these languages to reconstruct earlier stages of Wappo\il{Wappo} will most likely not help in solving the riddle of the origins of the Wappo\il{Wappo} marked"=S system. 

The former-ergative analysis is the most widespread line of historical explanation for marked"=nominative languages, yet there are a number of other explanations that have been suggested in the literature. 
An overview of various possible historical scenarios for the rise of overt-nominative marking is provided by \citet[178]{Koenig:2008}.  
She proposes for example a passive agent marker as another possible source for an overt nominative marker. She suggests that Maa\il{Maa} could be a case of this scenario. 
This pro\-po\-sal is parallel to what has been suggested as the origin of ergative\is{case!individual forms!ergative} markers for many languages. \citet{Anderson:1977} lists Polynesian languages such as Tongan, Niuean and Samoan, Australian languages, such as Walpiri, as well as Indo-European languages of the Indic and Iranian subgroups, for which a passive origin of ergativity has been proposed.  
The parallel scenario has not been widely discussed for marked"=nominative languages so far -- not to mention the doubts which have been cast on this origin for ergative languages (cf. for example Hindi as discussed by \citealp{ButtKing:2004}).

While the theories mentioned so far all search for the origin of the marked"=nominative within a prior stage of the case-system, there are other hypotheses suggesting that overt nominative-marking might have originated from a different domain of grammar altogether. 
The first such proposal sees the nominative marker as a reanalyzed definiteness marker, while another suggest an origin as a topic-marker. 
The definiteness origin is proposed for the Northern Lwoo languages Anywa, P\"ari\il{P\"ari}, and Jur-Luwo by \citet[179]{Koenig:2008}. 
Note also that \citet{Reh:1996} still analyzes the form under discussion in Anywa as a definite subject and does not consider the system a fully-fledged case-system. 
An origin as a topic-marker is suggested for the marked"=nominative of East Cushitic by \citet{Tosco:1994}. 
He lists two reasons for this assumption. 
First, subject-marking  only occurs with definite subjects in some of the East Cushitic languages, a feature he associates with topicality. 
Secondly, Tosco notes that nominative case-marking is not found with focused subjects in many of the languages under investigation. 

One critical point that has been ignored by all proponents of the historical explanations of marked"=S alignment is the immense rarity of this system. 
In other words, if there are so many routes that lead to marked"=S alignment, why are there so few marked"=S languages around?
This point is especially problematic for the `extended ergative' theory put forward by \citet{Dixon:1994}, since this approach states a universal pressure for ergative"=absolutive languages to become marked"=nominative -- yet standard ergative"=absolutive languages are far more numerous than marked"=nominative languages.
\citet[312--313]{Maslova:2000} suggests two reasons why a linguistic structure might be rare on a worldwide scale. 
One reason is that something is rare because there are hardly any ways in which a given linguistic structure can arise -- a situation that does not seem to hold for marked"=S coding given all the historical scenarios discussed above.  
The other possible reason for cross-linguistic rarity is that even though a linguistic structure may arise through a number of pathways, there are still more pathways leading away from that structure. 
So once a system has come into existence, it will very quickly be lost because it changes into yet another structure.

One example for such a rise and quick demise of marked"=S alignment is Old French\il{French (Old)}. 
While the old Latin case-system was abandoned, traces of the Latin Nominative remained, which were in fact the only traces of overt nominal inflection on full NPs. 
Therefore nouns distinguish only two case-forms (\ref{OldFre}), the Nominative form  that is encoded with the suffix \emph{-s} in most cases and the zero-coded Oblique form comprising all non-subject functions \citep[182]{Detges:2009,Jespersen:1992}.

\begin{exe} \ex\label{OldFre}\langinfo{Old French}{}{\citealp[94]{Detges:2009}}
\begin{xlist}
\ex\label{OldFreNom}\gll li chien-s mort l' ome\\
\deter{}.\sbj{} dog-\nom{} bites \deter{}.\obl{} man.\obl{} \\
\glt `The dog bites the man.'
\ex\label{OldFreObl}\gll le chien mort li uem\\
\deter{}.\obl{} dog.\obl{} bites \deter{}.\nom{} man.\nom{}\\
\glt `It is the dog whom the man bites.'
\end{xlist}
\end{exe}

In contemporary French\il{French}, the Nominative endings have been eliminated, making the marked"=nominative stage a transitional episode during the transfer from case-marking to positional licensing of grammatical relations. 
Notably, this quick episode of the marked"=S coding-system in French did not come about by any of the historical sources proposed in the literature, but simply by morphophonological attrition.  
However, not all marked"=S systems appear to be this short-lived. 
Since this alignment system spreads over major branches of language families, as in, e.g. Cushitic or the Yuman languages, marked"=S appears to be a rather stable system in these genealogical groupings.\is{marked-S languages!historical explanations|)}

\subsection{Functional explanations}\label{functional}\is{marked-S languages!functional definition|(} 

The second proposal to account for the existence of marked"=S systems is based on the range of functions individual case-forms have in a language.
This approach reduces the impact of the formal marking of case-forms. As a consequence, a different sense of the term markedness is employed for marked"=S languages by \citet{Koenig:2006,Koenig:2008}. 
She distinguishes between what she calls Type 1 and Type 2 marked"=nominative languages. 
Type 1 languages overtly code the S+A relation, while using a zero-coded form for P. 
Type 2 languages on the other hand employ overtly coded forms for all core relations, but the form employed for P is functionally unmarked (i.e. it covers the wider range of functions). 
I will refer to the two types as formally (Type 1) and functionally (Type 2) marked"=S languages.

The formal and functional criteria for identifying marked"=S languages coincide in the majority of cases. 
In other words, most languages which overtly code the S-case will employ the zero-case in a wider range of functions than the S-case. 
And vice versa, the case which covers the widest range of functions in a language will typically receive the least amount of overt coding. 
However, there are some exceptions to both generalizations. 
As this study will show, the overtly marked S-case sometimes covers all the functions one would expect of a non-marked nominative -- Maidu\il{Maidu} \citep{Shipley:1964} is a prime example of this. 
And conversely, even if a non-S-case has a wider range of functions, it will not necessarily receive less overt marking than the S-case -- this situation is found  for example in Wolaytta\il{Wolaytta} \citep{Lamberti:1997} or Gamo\il{Gamo} \citep{Hompo:1990}. 

The\is{marked-S languages!formal definition|(} formal approach is the more traditional way of characterizing marked"=S systems and is based on the presence or absence of overt formal marking of the different case relations. 
One short-coming of this approach is that it exclusively focuses on the encoding of the S, A and P arguments. Other functions the case-forms might have in the language under investigation are neglected. 
Those other functions are, for example, the usage to mark attributive possessors, predicate nominals or subjects of passive clauses. 
These other functions are taken into consideration in the second approach of defining marked"=S systems -- the functional one. 
This approach takes other functions besides S, A and P marking into account when identifying which case is the marked one and which is the default case\is{case!default}. 
This definition coincides with a slightly different notion of marked"=S languages in which overt marking of the S relation and zero-coding of one transitive relation is not a prerequisite.\is{marked-S languages!formal definition|)} 
The functional definition of marked"=S languages also includes languages in which all of the core verbal arguments (S, A, and P) are equally marked in terms of overt morphology, but the form used for the grammatical relation including S is used in less functions than the case-form used for the other core argument (P with nominative"=accusative and A with ergative"=absolutive alignment). 

The functional view of marked"=S alignment is advocated by \citet{Koenig:2006}. 
It is also the predominant take on this system by most scholars working on the marked"=S languages of Eastern Africa -- although it is usually not explicitly phra\-sed. 
The special role attributed to the zero-coded form in these languages is for example mirrored in the label chosen for this case-form, i.e. `absolute' or `absolutive\is{case!individual forms!absolutive}', which recognizes its wider use than just encoding the P relation (cf. Section~\ref{label}).  

This functional approach is put to the test within this study by examining the range of functions the different case-forms cover within marked"=S languages. 
In her formulation of the functional approach K\"onig states quite openly that the accusative\is{case!individual forms!accusative} is ``used with the widest range of functions'' (\citeyear[138]{Koenig:2008}). 
However, she does not explicitly define how this widest range of functions is to be measured. 
It appears that she is counting the number of different functions a certain case-form has, without distinguishing how peripheral or central to the grammar a certain function is. 
Also, she does not clearly explain how she arrives at the list of functions she is using in her comparison.  
One could, for example, argue that subjects of mono- and ditransitive verbs constitute two separate functions.\footnote{\citet[403]{Bickel.align} discusses a language encoding the two functions in distinct ways, namely Gyarong\il{Gyarong} (Sino-Tibetan).}
Furthermore, certain verb classes\is{verb class} are known to employ non-standard case-forms for their subjects in many languages, e.g. so-called experiencer-subjects. 
Marking the subject arguments of these verb classes could arguably be seen as separate functions.
As a result, the number of function an individual case-form has will vary considerably for the same language depending on the initial set of functions one considers.

\is{marked-S languages!functional definition!strong versus weak hypothesis|(}Irrespective of this, there are two possible interpretations of the claim of functional markedness of the S-case in marked"=S languages, and correspondingly two hypotheses one could test. 
In what I call the \textsc{weak version} of the functional markedness hypothesis, the widest range of functions is simply measured by comparing the range of functions of the zero-case with the range of functions of the S-case in a given language. 
In the \textsc{strong version} of the hypothesis the range of functions of the zero-case is not only measured against the number of functions of the S-case, but also against the combined number of functions of every other case-form in the language. 
Of course, for languages with a two term case-system both hypotheses make the same predictions. 
However, as \citet{Koenig:2008} notes, quite a number of the marked"=S languages of Africa have a larger inventory of case-forms. 
Also the marked"=S languages of North-America have somewhat more complex case-systems.
In this study I will test both versions of the functional markedness hypothesis -- the weak one and
the strong one.\is{marked-S languages!functional definition!strong versus weak hypothesis|)}%
%\enlargethispage{\baselineskip}

Like the historical explanations of marked"=S coding, the functional approach does not directly address the cross-linguistic rarity of this alignment system.
However, there is a promising line of argumentation for the dispreference of marked"=S alignment within this approach.
In what \citet[91--93]{Mallinson:1981} label the `discriminatory theories' of case, one would expect that the S argument -- which does not need to be distinguished from any other argument -- will be encoded with the zero-coded case-form. 
Therefore the same form will be employed for any transitive argument aligning with S.
%one step missing here, probably frequncy of zero-form (therefor other functions)
Using overt morphology on the S argument -- though there is no need for discriminating it from some other argument and thus no need for overt marking -- is a dispreferred strategy and therefore should not be widely distributed among the world's languages.
For similar reasons a number of other functions of a noun will be encoded with the zero-coded form. 
When using a noun in the citation\is{citation form} form (or other isolated context), there is no need for distinguishing its argument role from some other argument. 

In addition to providing the more promising explanation for the rarity of mark\-ed"=S languages, the functional approach also scores better in explaining marked"=absolutive\is{alignment!marked-absolutive} languages. While for the historical approaches integration of the marked"=absolutive system into the explanation is either doubtful or excluded by definition -- as it is the case for the `extended ergative' theory -- no such restriction exists for the functional approach. 
The zero-case is the one with the widest range of functions and whether the S-case comprises an S+A or S+P relation is irrelevant.\is{marked-S languages!functional definition|)} 
          
%In marked"=nominative languages the nominative case has some special properties as compared to the accusative. In the formal -- more traditional -- approach this special property of the nominative is its overt morphophonemic marking, while the accusative case has no such marking. In the functional approach to marked nominative languages this special property of nominative case is its restricted use, whereas the accusative has the property of a default case which is used in a wide variety of constructions

%%%%%%%%%%%%%%%%%%%%%%%%%%
%%%%%% SECTION 1.6 %%%%%%%
%%%%%%%%%%%%%%%%%%%%%%%%%%

\section{Implications for formal approaches to case-marking}\label{theoretical}

As noted in Section~\ref{definition}, the unexpectedness of the marked"=S system has been expressed by \citet{Greenberg:1963}, based on cross-linguistic observations.
The general tendency for the nominative\is{case!individual forms!nominative} and absolutive\is{case!individual forms!absolutive} to be encoded with less overt material has also been acknowledged by some formal linguists.
The following quote by \citet{Chomsky:1993} expresses the same observation as Greenberg's generalization (and extends it to the domain of verbal agreement).\enlargethispage{2\baselineskip}
  
\begin{quote}
The `active' element (Agr$_{S}$ in nominative"=accusative languages and Agr$_{O}$ in ergative"=absolutive languages) typically assigns a less-marked Case to its Spec, which is higher on the extractability hierarchy, among other properties. It is natural to expect less-marked Case to be compensated (again, as a tendency) by more-marked agreement (richer overt agreement with nominative and absolutive than with accusative and ergative). 
\citep[10]{Chomsky:1993}
\end{quote}

Beyond this brief statement, marked"=S languages are of no further relevance for Chomsky's theory, which when it comes to case is mostly concerned with the underlying deep-structure relations. 
Not much room is dedicated to the actual surface case-forms which are generated in spell-out. 
However, there are other formally-oriented linguistic paradigms for which marked"=S languages are of great relevance, since their existence poses a great challenge to the mechanisms of case-assignment underlying these theories.

Already in early work on case-systems, the nominative had a special status among the case-forms of the paradigm. 
This observation was often expressed by noting that, strictly speaking, only nouns in the nominative case can be viewed as nouns, while all other forms were just `cases of a noun', i.e. they were not considered to be nouns themselves. 
\citet[24]{Sweet:1876} for instance held the view that all ``oblique cases are really attribute-words.''
Likewise, in modern linguistics a strict division is  often made between the nominative and all other cases. 
As a result of this, some scholars treat the nominative (at least if zero-coded) as an non-case altogether, as exemplified in the following quotation; similar formulations can be found in \citet{Aissen:1999,Aissen:2003} and \citet[566]{deHoop:2008}:
 
\begin{quote}
We assume that nominative (or absolutive) case is in fact a label for `no case': that is, we assume that the absence of special morphological marking indicates the absence of case. 
\citep[322]{deHoop:2005}
\end{quote}

Even without denying case status to it, the special status of the nominative case is undisputed by theories of case. 
This special status is often referred to as it being the {`default'} or {`elsewhere'} case\is{case!default}. 
The elsewhere case is not restricted in its usage by any conditions on its occurrence, unlike, for example, German\il{German} Dative subjects, which have to be licensed by certain mental state verbs.
Because of its principally unrestricted usability, the nominative is the case that occurs in the widest range of functions, which is exactly the property ascribed to the zero-accusative by the functional approach to marked"=nominative languages. 
As a consequence, some theories run into trouble when confronted with marked"=S languages, since the default\is{case!default} case nature of the nominative is hard-wired into their structure. 

In some theories, the default\is{case!default} status of the nominative is not only an underlying assumption, but is actually built into the theory via a set of features or some similar technical apparatus.
The cases within the paradigm are specified with respect to those abstract features. 
Notably, the default\is{case!default} case is then analyzed as the maximally underspecified case (i.e. the set of features characterizing it is the empty set).
Hence there are no restrictions on the occurrence of the default\is{case!default} case, which means it could theoretically occur in all contexts.
The default\is{case!default} case is, however, subject to the Elsewhere Principle\is{Elsewhere Principle} \citep{Kiparsky:1973}.
Thus if a given contexts meets the feature specification of another case-form, this more specific case-form will be picked over the default\is{case!default} case.
 
As\is{Lexical Decomposition Grammar|(} an example I will discuss Lexical Decomposition Grammar (henceforward LDG) in some detail and the difficulties arising due to the existence of marked"=S languages. 
LDG \citep{Wunderlich:1997,Stiebels:2002} assigns the following underspecified feature to the grammatical core cases:

\begin{itemize}
\item Dative: [+hr,+lr] 
\item Ergative: [+lr] 
\item Accusative: [+hr] 
\item Nominative/Absolutive  : [\quad]
\end{itemize}

Arguments are assigned case according to their theta-structure.
The feature [+hr] translates to `there is a higher role', which means in order for an argument to be assigned a case with this feature, there must be another argument that has a higher role. 
Conversely, the [+lr] feature requires an argument bearing a lower role necessary.
The lexical entry of a verb can override the features specified in theta-structure \citep{WunderlichLakaemper:2001}; however, for the moment we will neglect this. 
The following example illustrates the case-assignment in LDG.  In (\ref{LDGSF}) the theta-structure and semantic form of the verb `to see' are given.
The lambda abstractors in the theta-structure generalize over the argument variables of the verb -- {s} being the situation (or event) variable, which is not relevant for the argument structure -- increasing from left to right with respect to how deeply they are embedded into the semantic form of the verb.
For each argument (i.e. {x} and {y} in example (\ref{LDGSF})), there is a higher role if another argument is embedded less deeply into the semantic form. 
Conversely, if there is an argument that is embedded more deeply, a lower role exists.
 
\begin{exe}
\ex\label{LDGSF}
$ \underbrace{ \lambda x \quad \lambda y \quad \lambda s}_{\text{\normalsize\rm theta-structure}}$
\qquad $ \underbrace{\{ \text{see} (x,y)\} (s)}_{\text{\normalsize\rm semantic form}} $ 
\end{exe}

The argument positions in theta-structure are fully specified with respect to their [hr] and [lr] features, thus they can be assigned both positive or negative values for the respective features. 
Example (\ref{LDGTS}) demonstrates the mechanism of case-assignment to the arguments of a verb. 
In this process, a language assigns the case-forms  which are at disposal in its lexical case inventory to these fully specified argument positions. 
Contradictions between the feature specifications of argument position and case-marker are not tolerated by the mechanism. 
For a position for which the language finds no better matching feature specification in its case inventory, the maximally underspecified default\is{case!default} case will be picked. 
Nominative"=accusative languages assign accusative\is{case!individual forms!accusative} case to the {y} arguments since its specification as [+hr] fits into this argument slot. 
For the {x} argument no case with matching features is found. 
Hence, the default\is{case!default} nominative case is assigned. 
Conversely, ergative"=absolutive languages have a matching candidate for the [+lr] feature of the {x} arguments -- the ergative\is{case!individual forms!ergative} case -- but no other candidate than the default\is{case!default} case for the {y} argument.  

\begin{exe}
\ex\label{LDGTS}
\begin{tabbing}
nominative"=accusative \quad \= \nom{} \quad \= \kill
{} \> $\lambda x$\>$\lambda y$\\
{}\> -hr\>{+hr} \\
{}\> {+lr}\>-lr \\
nominative"=accusative\> \nom{} \>\acc{}	\\
ergative"=absolutive\> \erg{} \>\abs{} 
\end{tabbing}
\end{exe}

While the LDG approach neatly derives the case-assignment in standard no\-mi\-na\-tive-accusative and ergative"=absolutive languages, the system is not as suitable for marked"=S languages. 
The default\is{case!default} case is best described, not by the case functions it covers, but by stating that it is used in all contexts in which all other cases cannot be used. 
For the standard systems, this property is reflected in the LDG feature specification.  
In marked"=S languages, the role of a default\is{case!default} case must be ascribed to the zero-case, since this is the case with the elsewhere distribution. 
Thus, the feature-values LDG proposes for accusative and ergative case do not do justice to zero-accusatives or zero-ergatives found in marked"=S languages. 
Furthermore, if one adopts the LDG case feature specifications, one  would have to assume a correspondence between zero-exponence (no overt morphology) and non-zero feature sets for the zero-accusative [+hr] or zero-ergative [+lr]. 
And conversely, one would postulate a relation between overt exponence and zero-feature specification for the marked"=nominative or marked"=absolutive.  
That leaves one with a `NO form to meaning' relationship on the one hand and a `form to NO meaning' relationship on the other hand.
This is a most unsatisfying situation, which violates basic principles of morphological theory.\is{Lexical Decomposition Grammar|)} 
In the concluding part of this work, I will come back to this issue (Section~\ref{consequencesformal}). 
For now, I just note that in addition to being typologically rare, marked"=S languages do not seem to fit into the patterns that a number of formal theories offer for analyzing case-assignment.

%%%%%%%%%%%%%%%%%%%%%%%%%%%%%%%%%%%%%%%%%%%
%%%%%%%%%%%%%% SECTION 1.7 %%%%%%%%%%%%%%%%
%%%%%%%%%%%%%%%%%%%%%%%%%%%%%%%%%%%%%%%%%%%

\section{Domains of alignment}\label{domains}

\subsection{Beyond case-marking}\is{case-marking}
The discussion of marked"=S languages so far has exclusively focused on nominal case-marking\is{alignment!through case-marking}.
The labels \textit{nominative"=accusative} and \textit{ergative"=absolutive} are commonly employed to classify the system of case-marking on the noun phrase (i.e. de\-pen\-dent-marking). 
However, the terms are also applied more generally for any morphosyntactic device treating S like A or P, and thus including verbal indexing\is{alignment!through verbal indexing}\is{verbal indexing} (i.e. head-marking). 
In some instances, the terminology has even been expanded to word order\is{alignment!through word order} \citep{Buth:1981, Andersen:1988}. %
In addition, behavioral properties have been used to characterize the alignment system(s) of a languages \citep{Bickel.align}, thereby extending the term alignment beyond coding-properties.

I will discuss these other domains of alignment -- head-marking (Section~\ref{head}), word order (Section~\ref{order}) and behavioral properties (Section~\ref{behavior}) -- and clarify for all of these domains what the marked"=S equivalent would look like. 
All of these domains have some limitation with respect to the possibility of investigating the marked"=S system in them. 
For nominal case-marking, some restrictions exist as well. 
In the final section (Section~\ref{invest}), I will state these limitations and thus define the exact domain of this study on marked"=S languages, namely, nominal case-marking of full noun phrases.

\subsection{Head-marking}\label{head}\is{alignment!through verbal indexing|(}\is{verbal indexing|(}

For indexing the S+A (or S+P) arguments, overt morphological marking is the norm cross-linguistically (if a language chooses to employ head-marking devices at all); cf. the \citet{Chomsky:1993} quote at the beginning of Section~\ref{theoretical}.
So for the indexing-system of a language, it would be unusual to lack overt coding of the S+A (or S+P) relation, and instead only indexing the P (or respectively A) argument.
Thus, a system that overtly marks S arguments via verbal indexing is actually the expected, and most common system (for languages that encode their arguments on the verb at all). 
A system that would be comparable to marked"=S case-marking in terms of its unexpectedness accordingly would lack overt marking of S arguments on the verb, while indexing some other arguments. 
The head-marking counterpart of marked"=S would thus be more appropriately called `unmarked"=S'.    
\enlargethispage{\baselineskip}

Just like the marked"=S dependent-marking, its equivalent in head-marking appears to be rare typologically.
\citet{Miestamo:2008} lists Khoekhoe as the only language with verbal indexing for objects but not subjects, while there are three languages with a marked"=S case-system in his 50 languages world-wide sample. 
Drawing on the larger sample of the \emph{World Atlas of Language Structures} (WALS), when combining the data from the two chapters devoted to verbal person marking \citep{WALS100, WALS102}, there are 18 languages that have nominative"=accusative alignment in their verbal person marking while indexing only their P argument.
These languages stand against 192 languages marking either only the A or both A and P arguments.
In addition there are 3 languages with ergative"=absolutive alignment cross-refencing only their A argument, which are compeeting with 14 language marking either only P or both A and P arguments.\footnote{The languages with the unmarked"=S agreement system are \textdoublevertline Ani\il{\textdoublevertline Ani}, Anejom\il{Anejom}, Batak\il{Batak}, Ijo\il{Ijo}, Indonesian\il{Indonesian}, Kera\il{Kera}, Khoekhoe\il{Khoekhoe}, Kisi\il{Kisi,}, Mupun\il{Mupun}, Nakanai\il{Nakanai}, Noon\il{Noon}, Palikur\il{Palikur}, Panyjima\il{Panyjima}, Selknam\il{Selknam}, Sema\il{Sema}, Tiguk\il{Tiguk}, Warao\il{Warao}, and Yapese\il{Yapese}, for the nominative"=accusative alignment, and Atayal\il{Atayal}, Chamorro\il{Chamorro} and Nad\"eb\il{Nad\"eb}, with ergative"=absolutive alignment. The total number of languages shared between the two WALS maps is 378.} 

However, except for the fact that marked"=S case-marking and unmarked"=S indexing are both rare, there is no structural or logical reason to compare the two structures. 
Including both phenomena into this study would even lead to methodological restrictions.
In Chapter~\ref{method}, I will outline an approach to comparing marked"=S languages by means of a number of functions such as attributive possessor or subject of positive and negative existential constructions; a number of these functions cannot be studied in indexing-systems.
This is the case for those structures which are below clause level or extra-syntactic, namely, attributive possessors and the form of a noun used in citation or address. 
Furthermore, in some of the more complex constructions the comparison of case- and agreement-marking languages is also problematic. 
In nominal predications, for example, not all languages employ a construction that exhibits verbal agreement. 
On the one hand, there are languages in which no overt verbal element at all is employed in this context. 
Zero-copulas are in fact most common in nominal predications and only occur in other types of non-verbal predications when also found there \citep[62--65]{Stassen:1997}. 
Also, overt copulas are most likely to be absent in third person contexts \citep[65]{Stassen:1997}, which comprise clauses with full noun phrases -- the domain of this investigation.
On the other hand, if a language makes use of a copula in nominal predications, this copula might
not behave like other verbs in terms of agreement and other properties. \citet{Pustet:2003} notes
the tendency of copulas not to behave like verbs in terms of morphosyntax.\is{alignment!through
  verbal indexing|)}\is{verbal indexing|)}%
\enlargethispage{\baselineskip}

\subsection{Word order}\label{order}\is{alignment!through word order|(}\is{word order|(}  

When extending the notion of marked"=S to word order, first of all, a few considerations have to be made on how alignment systems can be translated into the ordering of constituents.
While word order is seen as an alternate means to distinguish arguments on a par with case-marking and verbal indexing, the notion of nominative"=accusative or ergative"=absolutive word order is not commonly found in grammars.
A reason for this might be that word order is thought to be on the nominative"=accusative basis almost without exceptions.\footnote{Cf. the debate on syntactic ergativity \citep{Anderson:1976,Anderson:1977, Dixon:1994} and whether such a phenomenon exists at all -- though this debate is not restricted to word order.}
However, there are a few examples of languages for which an ergative word order has been proposed.
This is the case in P\"ari\il{P\"ari} \citep{Andersen:1988} -- though only in main clauses -- and Luwo\il{Luwo} \citep{Buth:1981}, which have a SV+PVA word order.\footnote{In order to make the alignment systems that are to be identified through word order more clear, I will consistently use the S, A, and P labels for the arguments, rather than employing the traditional word order abbreviations such as SOV or SVO.} 

One complication in associating the different types of alignment with specific word orders is the inherently relational nature of this property.
The ordering of a specific argument can only be identified in relation to some other element.
At least three factors could be taken into account here.
First, there is the ordering of an argument with respect to another argument. 
This measure can obviously only be applied in clauses with more than one argument.
So this factor in itself is not helpful for identifying nominative"=accusative or ergative"=absolutive alignment in a language, since the S argument of intransitive clauses is not parallel to the A or P argument in preceding or following the other argument. %(Criterion I). 
Secondly, one can take into account the ordering of an argument with respect to the verb. %(Criterion II)
Two sub-criteria can be called upon here, strict ordering (i.e. precedes/follows the verb% -- IIa
) and direct adjacency to the verb. %(I Ib)
And finally one can classify the ordering of an argument with respect to clausal boundaries, %(Criterion III)
i.e. whether it occurs at initial position (or final or any other salient position one might be interested in investigating) of the clause.

In verb-medial languages, there is an overlap between these two last factors. 
The argument that precedes the verb will also typically be in clause-initial position and be directly adjacent to the verb (unless there are good reasons to assume an intervening clause structure position). 
For these types of languages, positioning with respect to the verb and with respect to clausal boundaries will identify the same type of alignment system: nominative"=accusative for the ordering of SV+AVP or VS+PVA and ergative"=absolutive for languages with SV+PVA or VS+AVP word order. 
If both transitive arguments are positioned on the same side of the verb, the two criteria identify different alignment systems. 
Taking the adjacency and positioning to the verb as the unit of measure, SV+APV languages are ergative"=absolutive, but with respect to the clause initial position, they are nominative"=accusative. 
Also there is the theoretical possibility -- though not attested and supposedly highly unlikely to be found -- of a language where both transitive arguments are found on the same side of the verb, but intransitive S occurs on the other side. 
Here in the ordering with respect to the verb, S does not align with either A or P, but in terms of verb-adjacency it behaves like whichever transitive argument is adjacent to the verb.

In conclusion, the definition of nominative"=accusative or ergative"=absolutive word order is somewhat problematic for verb-initial and verb-final languages. 
Only if a language is verb medial can a distinction based on word order be made between a nominative"=accusative (SV+AVP/VS+PVA) and an ergative"=absolutive basis to the word order (SV+PVA/VS+AVP).

Furthermore, the notion of marked versus unmarked ordering must be clarified for identifying the marked"=S equivalent in word order.
As noted earlier, when defining alignment systems, the ordering of the A and P argument with respect to each other is not of much use. However, previous research on the ordering of subject and object has revealed that subjects tend to precede objects in word order. 
This might be a good starting point to identify the marked types of word order. Like marked"=S case-marking, languages with objects preceding subjects are rare on a worldwide basis. This finding suggests that the straightforward equivalent of a marked"=nominative language would be a language in which the object precedes the subject in the canonical word order. 

The tendency of ordering A before P was already stated by \citeauthor{Greenberg:1963} as the first universal of his seminal paper on word order: 

\begin{quote}
In declarative sentences with nominal subject and object, the dominant order is almost always one in which the subject precedes the object. 
 \citep[61]{Greenberg:1963}
\end{quote}

This observation is, however, biased toward languages with no\-mi\-na\-tive-ac\-cu\-sa\-tive word order.
Though this is the cross-linguistic norm, there are some remarkable languages not conforming to the S+A vs. P word order, as discussed above.
Therefore, in ergative"=absolutive languages the `marked' status of the object-precedes-subject ordering might be questionable, since the term `subject' is not applicable for the A argument in a language exhibiting a syntactic S+P pivot by grouping those two arguments together in terms of word order.

Greenberg's observation on the ordering of A and P was confirmed by large scale studies. 
For example, \citet{WALS81} finds 1017 languages in his sample in which the subject precedes the object, while only 39 have the subject following the object (172 languages are listed as having no dominant order of S, O and V). 
With this figure, one has to take into account that -- as discussed above -- not all of these languages will allow for a clear classification of having either nominative"=accusative or ergative"=absolutive alignment in terms of word order. 
Only verb-medial languages allow for an unambiguous classification. Out of the 39 languages with O-S order in Dryer's sample, there are nine verb-medial ones (two of which -- P\"ari\il{P\"ari} and Mangarayi\il{Mangarayi} -- will be discussed in this study due to the marked"=S properties in their case-marking systems).\footnote{The remaining seven languages with marked"=S word order are: Asurin\'i\il{Asurin\'i}, Cubeo\il{Cubeo}, Hixkarayana\il{Hixkarayana}, Selknam\il{Selknam}, Tiriyo\il{Tiriyo}, and Ungarinji\il{Ungarinji}.  These languages could be considered marked"=S word order languages, unless they are revealed to be instances of ergative word order like P\"ari\il{P\"ari}.}  

%In the previous section, it has been demonstrated that it is already hard to compare head-marking with dependent marking. For positional marking (word order) the situation gets even worse. Marking via position in the clause usually only gives a very limited range of options, furthermore it can only be determined with respect to some point of reference (e.g. before/after the verb, at the beginning of the clause). In order to compare a set of contexts the same point of reference must be present in all contexts, radically limiting the possible points of referents. Still, in most cases it will be not decidable whether to elements occur in exactly the same position (this has already been an issue above, when excluding all non-verb-medial languages from the further discussion). Further, many languages -- even with a rather fixed word order -- allow for some flexibility in ordering the elements of a clause. Sometimes this will lead to differences in interpretation, but pinning down those differences is usually not possible for grammar writers without native competence of a language (and sometimes also for native speakers). 
As with verbal indexing, there are no good reasons to enlarge the scope of this study of marked"=S languages to this domain.
The two phenomena are different in nature, and as with head-marking, some of the contexts of interest for nominal case-marking have no equivalent.
In addition, the definition of alignment systems based on the ordering of constituents should be put on firmer theoretical ground, before attempting to do a unified study of any alignment system through all domains. 
As a consequence, word order as a means of classifying languages as marked"=S will not be used within this study. 
However, there will be some discussion of the word order of marked"=S languages in Chapter~\ref{emphaticS}, which deals with information structure.\is{alignment!through word order|)}\is{word order|)}  

%Also there are other factors which have to be considered when studying the ordering of subjects and objects. Information structure probably being the most central one. The relationship between subjects and topics has long been discussed in linguistics (the contributions in \citealp{Li:1976} are just a few to mention). Because of their topical nature subjects do often constitute given information. As old information precedes new information %
%\textcolor{red}{citation needed} there is a straightforward principle to account for the preferences of subjects to come before objects in the languages of the world. Though \citet[199ff.]{Lambrecht:1996} discusses arguments against an universal topic first principle, this does not mean that such a principle cannot be states for some -- or indeed a majority of -- languages. Since this study is meant to focus on alignment and not information structure and therefor marked"=S word order -- if one chooses to adapt this term for object before subject canonical word order -- do not seem to be the ideal domain to study this phenomenon. 

\subsection{Behavioral properties}\label{behavior}\is{alignment!behavioral factors|(} 

Traditionally, the overt coding of case and verbal agreement have featured prominently in studies of alignment systems. 
Additionally\is{clause-type!relative clause|(}, behavioral properties such as relativization, equi-NP deletion, conjunction reduction or control/raising are also possible factors in establishing the alignment systems of a language \citep{Bickel.align}. 

Studies of behavioral properties have shown that the typical subject arguments often allow for behavior that cannot be found with other arguments.
So one definition of behavioral marked"=S would be a language in which subjects (or S+P pivots in ergative"=absolutive languages) are more restricted in their behavior than non-subjects.
For the domain of relativization, for example, it has been shown that subjects are the most widely relativizable elements across languages. 
Further, languages that allow for other types of relativization must also allow subjects to be relativized \citep{Keenan:1977}. 
All proposed counterexamples to this so-called `Accessibility Hierarchy'\is{Accessibility Hierarchy}  have not touched upon the special status of subjects, but rather made amendments to the non-subject part of the hierarchy. 
So, there do not seem to be any cases of `marked"=S relativization'.\is{clause-type!relative clause|)}

It may well be that there are other interesting behavioral properties of marked"=S languages, but their investigation requires an extensive description of these topics in a grammar. Hardly any of the materials available on marked"=S languages provide such an in-depth discussion of these issues.\is{alignment!behavioral factors|)} 

\subsection{The domain of investigation}\label{invest}

For the reasons listed above, I concentrate on marked"=S as a phenomenon in the domain of nominal case. 
I adopt the broad definition of case\is{case!overt marking}  as given by \citet{Bickel.marking}, which includes all instances of morphological case-marking (affixes, stem change, tone\is{case-marking!via tone}, clitics) and also adpositional\is{case-marking!via adposition} marking rather than only case-marking via inflectional affixes.
Furthermore, it is case-marking of full noun phrases (NPs) rather than pronominals which is the focus of this survey. 
The reasons for this restriction  are the following:

First of all, formal zero vs. overt coding is usually easier to identify for full NPs. 
The pronominal system of a language often consists of two or more sets of pronouns for the individual cases which are not historically related to each other (or such a relation might be blurred through language change). 
In these cases, the identification of the zero-coded vs. the overtly coded form of a pronoun either in terms of non-affixed vs. affixed form, or underived vs. derived form, cannot be performed easily and uncontroversially.  

Secondly, not all languages make ready use of pronominal arguments. 
A large number of languages (so-called `pro-drop' languages) will not overtly realize pronominal arguments in many instances. 
Moreover, if the pronoun is expressed in one of these languages, then the pronominal element will bear some special discourse status, such as expressing a contrast (either against the expectations of the listener, or to highlight a change in participants). 
Such contrastive contexts are an important part of this study. 
However, they have to be compared to contexts with a neutral information structure, in which often no overt pronouns occur at all. 
Therefore, overt pronominals as the element most prone not to be neutral with respect to their information structure are not the ideal domain for this investigation.

Finally, the pronominal and the nominal systems of a language sometimes behave differently in terms of their alignment. 
Split alignments along the so-called Silverstein Hierarchy\is{Silverstein Hierarchy}, which have long been noted for ergative"=absolutive languages \citep{Silverstein:1976,Dixon:1994} are also found in marked"=S languages \citep{Handschuh:2008,Handschuh:2014}. 
A discussion of all alignment splits found with marked"=S languages -- including those on the Silverstein Hierarchy -- is presented in Section~\ref{splits}. 

Instead of lumping together the categories of pronominal and nominal case-marking, full NPs have been selected for this study.    
Unavoidably, some of the examples presented do contain pronouns, but this is only done when the description of the language makes it clear the behavior of pronouns and full NPs is identical for the feature under investigation. 
Wherever possible, examples are chosen where the relevant item is a full noun phrase.

%%%%%%%%%%%%%%%%%%%%%%%%%%%%%%%%%%%%
%%%%%%%%%% SECTION 1.8 %%%%%%%%%%%%%
%%%%%%%%%%%%%%%%%%%%%%%%%%%%%%%%%%%%

\section{Outlook}\label{outline}

In the remainder of this study, an in-depth investigation of marked"=S coding-systems will be provided as found in the case-marking system of full noun phra\-ses.
The next chapter will present the methodology of this investigation. 
This methodology is based on the notion of \textsc{micro-alignment}. 
That is the notion that the alignment system of a language can be established in a large number of contexts, and that in fact the alignments regularly differ between these contexts within one and the same language. This phenomenon is often referred to as \textit{split alignment}.
Part two (Chapter~\ref{nompred}--\ref{extrasyn}) comprises the discussion of the contexts selected for this investigation (these will be introduced in the next chapter) and demonstrates the strategies employed by marked"=S languages to encode these contexts.
Based on these data, I will investigate how uniform or diverse the marked"=S languages are. 
Special emphasis will be put on systematic patterns arising in term of genealogically\is{typology!sampling!genealogical grouping} or areally\is{typology!areal} defined groups of languages (Chapter~\ref{typology}).
Finally, I will conclude this survey in discussing the validity of the findings, the overall applicability of the methodology, the implications for theoretical linguistics, and propose further lines of investigation (Chapter~\ref{conclusion}).     


 

