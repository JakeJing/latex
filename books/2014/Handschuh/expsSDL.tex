\chapter{Emphatic subjects}\label{emphaticS}
%\setcounter{exx}{0}

%%%%%%%%%%%%%%%%%%%%%%%%%
%%%%% Section 5.1 %%%%%%%
%%%%%%%%%%%%%%%%%%%%%%%%%

\section{Introduction}\label{EmphIntro}


For the languages of East-Africa, it has been repeatedly observed that discourse function plays a crucial role for case-marking of subjects. 
\citet[240--271]{Koenig:2008} concludes that overt nominative markers in many of these languages are absent in pre-verbal position. 
More generally, she notes a tendency of all languages of North-Eastern Africa not to employ overt case-marking in this position. 
This tendency is referred to as the `no case before the verb' rule by K\"onig. 
The Nilotic languages, for example, are predominantly verb-initial, thus the canonical position for all arguments is post-verbal. 
Whenever a subject argument is fronted -- usually for discourse structure reasons -- it will occur in the zero-coded form.

\begin{exe}\ex\label{TurEmphEx} {Turkana\il{Turkana}} \citep[Nilotic; Kenya; ][390, 177]{Dimmendaal:1982}\nopagebreak
\begin{xlist}
\ex\gll\textipa{t\`Ok\`Ona\`{}} \textipa{n\`eg\`{}} \textipa{a:a}, \textipa{\textbf{N\`esi\`{}}} \textipa{e-los-\`I} \textipa{n\`a-wuy\r*{\`{e}}}\\
now here \topic{} he.\acc{} 3-go-\asp{} \dir{}-home\\
\glt `Now, HE goes home.'
\ex\gll\textipa{tO-rUk-\`O-U} \textipa{\textbf{Nes\`I}}  \textipa{k-\`IpUd-\`Ud} \textipa{a-m\`ana}\\
3-meet-\epen{}-\ventiv{} he.\nom{}  3-trample-\result{} field.\acc{}\\
\glt `He found the field in a trampled state.'
\end{xlist}
\end{exe}

Most grammars are vague on the exact function that this fronting of arguments fulfills. 
What seems to be common to all languages is that special emphasis is put on the fronted argument, hence, I refer to the context to be studied in this chapter as \textsc{emphatic subjects}\is{emphatic subject}.
\footnote{Remember that I use the term subject as a shorthand for the S argument of intransitive verbs plus whichever transitive argument is encoded in a parallel fashion in terms of the overt case-marking.}  

In this chapter, I will investigate the marking of discourse-prominent subjects in marked"=S languages. 
First, I give a very brief overview of the different patterns of interaction between the marking of discourse structure and case-marking, as well as a general overview (Section~\ref{discourse}).
Following this brief introduction, I will discuss the accounts offered for the absence of case-marking in emphatic contexts (Section~\ref{exposed}). 
Next, I will discuss overt case-marking exclusively found on emphatic subjects, another pattern of interaction of the domains of case-marking and discourse structure, which can be analyzed as a very special instance of marked"=S coding, though the languages in question are not typically included in the study of marked"=nominative languages (Section~\ref{focusNOM}).
Afterwards, I will summarize the different patterns and point out the research questions that are of interest for this study (Section~\ref{EmphQuest}). 
The subsequent sections provide detailed information on how the individual languages of the Nilo-Saharan (Section~\ref{EmphNilo}) and Afro-Asiatic  stocks (Section~\ref{EmphAfro}), as well as the Pacific (Section~\ref{EmphPac}) and North American areas (Section~\ref{EmphNA}) behave with respect to the interaction of case-marking and discourse structure. 
Finally, a summary of these data will be provided in Section~\ref{EmphSum}.


%%%%%%%%%%%%%%%%%%%%%%%%%
%%%%% Section 5.2 %%%%%%%
%%%%%%%%%%%%%%%%%%%%%%%%%

\section{Case-marking and discourse structure}\label{discourse}

Zero-coded emphatic subjects are not an exclusive feature of African marked"=S languages. 
A similar structure can be found in some languages of the Pacific region. 
Also in the Pacific region, another opposite type of discourse-structure sensitive marked"=S system exists: languages in which overt marking of the S argument is exclusively found in emphatic contexts. 
The discussion of this kind of marked"=S system is commonly subsumed under the phenomenon of optional ergativity \citep{OptionalErg}\is{case-marking!optional}, even if the optional ergative\is{case!individual forms!ergative} marker is found on intransitive S.
Further, case-marking does not distinguish between emphatic and non-emphatic contexts in a number of marked"=S languages. 
These languages are mostly found in North America but some African languages are of this type as well.

The\is{emphatic subject|(} main focus of this chapter will be on the two patterns that distinguish between emphatic and non-emphatic subject arguments in terms of case-marking. 
For both systems, different explanations have been proposed on how the respective system arose. 
The two systems and proposed explanations are discussed in the subsequent sections (\ref{exposed} and \ref{focusNOM}). 
First, I will introduce the basic concepts of information structure in this section.\is{emphatic subject|)} 

The term `information structure' has been coined by \citet{Halliday:1967}, but the study of this domain of grammar can be traced back to the classical works of Aristotle. 
Nowadays, information structure is often treated within the larger field of discourse analysis. 
It is concerned with the introduction and tracking of referents within a larger discourse and the formal means used for this purpose. 
The whole domain of information structure is a field in which little consensus on the basic concepts or the meaning of specific terms appear to exist \citep[261--276]{Payne:1997}. %Entire volumes could be filled investigating the different terminological schools and traditions applied to this domain of grammar.
In contrast, information structure is a field that is only rarely treated by linguists working on little described languages.
Possibly as a result of this, typological work on discourse structures is still rarely carried out \citep{Myhill:2001}.   
The following discussion is meant to introduce the basic concepts of the study of information structure as well as to define the terminology used in this chapter.

The two concepts `topic' and `focus' are the most widely used types of discourse relations in the literature. 
The terms `theme' and `rheme' are often used instead for these concepts.
\citet{Lambrecht:1994} for example dedicates a complete chapter of his book to each of the two concepts. 
\textsc{Topics}\is{topic|(} are generally understood to be the things that one is talking about, or formulated less vaguely, they have a high level of mental activation with the discourse participants and are repeatedly expressed as arguments within the discourse. 
As a result, topical elements are often expressed through very little overt material once they have been established. 
Pronominals are typical discourse representations of topics. 
If a language has the option to not overtly express an argument, topical elements are the prototypical candidates for this process (see also the discussion on the omission of arguments in Section~\ref{omitarg})\is{topic|)}. 
\textsc{Focused\is{focus|(} elements}, in contrast, are unexpected in the given context. 
They do not have to be mentioned in the previous discourse and typically have a low level of mental activation. 
Focused elements tend to be realized with more overt material (e.g. as full noun phrase rather than as pronoun). 
In more philosophical treatments, topics are often equated with the subject of a clause while focus is linked to the predicate.
The following English\il{English} examples are typical topic (\ref{EngTop}) and focus (\ref{EngFoc}) structures.

\begin{exe} \ex\label{EngTop} \textit{Speaking of \textbf{John}, he was involved in a car crash.} \end{exe}
\begin{exe} \ex\label{EngFoc} \textit{It was \textbf{John} (not Susan) who was involved in a car crash.} \end{exe}\is{focus|)}

The broad notions of topic and focus are often subdivided into subcategories, which might have quite different properties with respect to their linguistic expression. 
A\is{topic!contrastive|(} special kind of topic is the so-called `contrastive topic' \citep[291--296]{Lambrecht:1994}. 
This type of construction is used in cases where more than one possible discourse topic has been established and after referring to one of them, reference to another of these topical elements is made. 
This switch of topic is usually marked overtly, but usually no more overt material is used than is necessary to establish the reference. 
In languages that have gender-specific pronouns, for example, the switch between a male and a female topical participant is transparent through the use of the respective pronoun.\is{topic!contrastive|)}
In\is{topic!afterthought|(} addition, topics are sometimes classified with respect to their position in the clause. 
One cross-linguistic generalization that is often repeated is that ``old information precedes new information'' \citep[119]{Ward:2001}. 
However, this is just a tendency. 
Apart from the observation that topics (i.e. old information) can also be left out from overt realization since they are already known, a number of languages have a special topic construction, the so-called `afterthought topic'. 
In this construction the topical element, which has not been prominently realized in a proposition, is added after the proposition has been made as a sort of addition to the clause into which it is not syntactically integrated.\is{topic!afterthought|)}  

Focus constructions are often distinguished by the grammatical status of the element in focus \citep[226--235]{Lambrecht:1994}. 
The first type of focus construction is `predicate focus'. 
In this situation a topic (most likely a person) is established within the discourse and some additional information on this participant is given. 
`Argument\is{focus!constituent|(} focus' constructions, in contrast, are used if what happened is already known but there is some uncertainty or misunderstanding about the involved participant(s), as exemplified by (\ref{EngFoc}) above.\is{focus!constituent|)}
Furthermore\is{focus!sentence|(}, an entire sentence can consist of new information, in which case one speaks of `sentence focus'.
A different terminology for sentences like this is `thetic', which is contrasted with `categorial' sentences \citep{Sasse:1987}.
In addition, the term `contrastive focus' is also used for constructions in which the focused element is opposed to another element of the same syntactic category.\is{focus!sentence|)} 
All of the focus constructions introduced above can be used contrastively. 
This type of focus corrects an assumption that the listener had about an event (concerning the predicate, argument(s) or entire proposition respectively).

%%%%%%%%%%%%%%%%%%%%%%%%%
%%%%% Section 5.3 %%%%%%%
%%%%%%%%%%%%%%%%%%%%%%%%%

\section{Zero-coded emphatic subjects}\label{exposed}

As\is{emphatic subject|(} noted above, the absence of nominative case-marking in pre-verbal position is one of the signature features of the African marked"=nominative languages. 
However, this pattern is not found exclusively in this region of the world. A similar pattern is also found in some languages of the Pacific region. 

The following are some examples of languages in which the emphatic S is not marked for case in the same way as the non-emphatic S. 
The (a) example is always the one with the emphatic subject while in the (b) example no emphasis is put on the subject.
In the Nilotic languages Nandi\il{Nandi} (\ref{NanEmphExa}) and Turkana\il{Turkana} (\ref{TurEmphExa}) emphatic subjects are in the zero-coded accusative\is{case!individual forms!accusative} case and occur in pre-verbal position. 
In non-emphatic contexts, on the other hand, subjects are in the nominative case-form, which has a different tonal pattern and is derived from the accusative\is{case!individual forms!accusative} form of a noun.  
The Western Malayo-Polynesian language Nias\il{Nias} behaves in a similar fashion. When S or P arguments occur in the non-canonical pre-verbal position, they  are in the Unmutated form (\ref{NiaEmphExa}a) while in post-verbal position they would be in the Mutated form (\ref{NiaEmphExa}b). 
The fronting of an argument is a communicative means employed to express the importance in discourse of the respective argument.


\begin{exe}\ex\label{NanEmphExa} {Nandi\il{Nandi}} \citep[Nilotic; Kenya;][124, 125]{Creider:1989}\nopagebreak
\begin{xlist}
\ex\gll \textipa{\textbf{kipe:t}} \textipa{k\'o} \textipa{k\^e:r-\'ey} \textipa{la:kw\'e:t}\\
Kibet.\acc{} \partic{} see-\ipfv{} child.\acc{}\\
\glt `{KIBET} is looking at the child.'
\ex\gll\textipa{k\`e:r-\'ey} \textipa{\textbf{k\'Ipe:t}} \textipa{la:kw\'e:t}\\
see-\ipfv{} Kibet.\nom{} child.\acc{}\\
\glt `Kibet is looking at the child.'
\end{xlist} 
\end{exe} 

%\pagebreak
\begin{exe}\ex\label{TurEmphExa} {Turkana\il{Turkana}} \citep[Nilotic; Kenya;][82]{Dimmendaal:1982}\nopagebreak
\begin{xlist}
\ex\gll\textipa{\textbf{Na-atuk\`{}}} \textipa{\textbf{Na-arey\`{}}} \textipa{m\`ake\`{}} \textipa{e-yak\`a-sI} \textipa{a-yON\`{}}\\
\Non{}\_\neu{}.\pl{}-cow.\acc{} \Non{}\_\neu{}.\pl{}-two.\acc{} self 3-be-\pl{} 1\sg{}.\acc{}\\
\glt `Two cows is all I have.'
\ex\gll\textipa{a-yON\`{}} \textipa{e-yak\`a-sI} \textipa{\textbf{Na-\`at\`uk}} \textipa{\textbf{Na-\`ar\`ey}} \textipa{m\`ake\`{}}\\
1\sg{}.\acc{} 3-be-\pl{} \Non{}\_\neu{}.\pl{}-cow.\nom{} \Non\_\neu{}.\pl{}-two.\nom{} self\\
\glt `I only have two cows.'
\end{xlist} 
\end{exe}

\begin{exe}\ex\label{NiaEmphExa} {Nias\il{Nias}} \citep[Sundic; Indonesia;][262]{Brown:2001}\nopagebreak
\begin{xlist}\ex\gll
\textbf{si'o} h\"o'\"o ma+i-taru-'\"o ba dan\"o\\
stick \dist{} \prf{}=3\sg{}.\rls{}-plant-\transitiv{} \loc{} ground.\mut{}\\
\glt `That stick he planted in the ground.'
\ex\gll 
i-taru-'\"o \textbf{zi'o} h\"o'\"o ba dan\"o\\
\prf{}-3\sg{}.\rls{}.plant-\transitiv{} stick.\mut{} \dist{} \loc{} ground.\mut{}\\
\glt `He planted that stick in the ground.'
\end{xlist}\end{exe}\is{emphatic subject|)}

There are two types of explanation for this alternation in case-marking with emphatic subjects. 
The first explanation argues that the emphatic S argument is in a structural position in which it cannot be assigned the regular S-case. 
The second approach is only suitable for those languages that mark the emphatic S argument by some other device, e.g. a focus-marker. 
For the languages of this type the occurrence of the S-case-marker might simply be blocked by the presence of another marker on the S argument and not by its structural position. 

The\is{focus!cleft-analysis|(} first explanation -- i.e. the one claiming that emphatic subjects are outside of the domain in which they can be assigned S-case -- comes in a more specific and a more general version. 
The more specific variant analyzes the whole structure as a biclausal cleft-construction while the second analysis more generally states that the emphatic argument is outside the domain of case-assignment.\footnote{The more general analysis of the emphatic argument being outside the domain of case-assignment also captures the more specific cleft-analysis.}
I will first turn to the more specific version of the structural explanation, which I will refer to as the \textsc{cleft analysis}. 
It states that sentences with an emphatic subject have a structure similar the the one exemplified by the English\il{English} cleft-construction in (\ref{EngCleft}).

\begin{exe} \ex\label{EngCleft} \textit{It is John who lost his wallet.} \end{exe}

The whole structure of the clause with an emphatic subject argument is interpreted as actually consisting of two clauses.
The\is{nominal predication!predicate nominal|(} first clause, i.e. the cleft, only consists of the logical subject of the entire structure. 
However, it is not realized as a grammatical subject but as a predicate nominal. 
The second clause is a headless subject relative clause modifying the predicate nominal\is{nominal predication!predicate nominal}. 
This analysis predicts that since the logical subject does actually function as a predicate nominal, the emphatic S will have the same marking as a predicate nominal in the respective language. 
As has been shown in Chapter~\ref{nompred}, many marked"=S languages indeed employ the zero-coded form for predicate nominals. 
This analysis of emphatic subjects as biclausal structures is put forward by \citet{Koenig:2008} for African marked"=S languages. 
Also \citet[278--281]{Payne:1997} discusses cleft-constructions as a source for focus-constructions in general.

This line of argumentation can either be interpreted as a synchronic analysis or merely as the historical source of the modern construction. 
In either way, this analysis is only plausible if a language meets the following typological requirements (or met them at the point in time, when the emphatic S construction developed): 

\begin{enumerate}
\item The formal marking of predicate nominals and emphatic subjects must be the same.
\item The language must allow for nominal predications to lack an overt copula\is{copula!absence versus presence} (or an additional marker, that functions as a copula must be present in the construction).
\item The language must either allow for relative clauses to be formed without an overt relative marker, or such a marker that introduces relative clauses must be present in the constructions.  
\end{enumerate}\is{nominal predication!predicate nominal|)}

These requirements are easy to check as a synchronic claim. 
However, as a diachronic claim this check is not always possible.
In addition, there will be languages that meet only some of these criteria, that one would nevertheless want to analyze in a parallel fashion.

For some languages this analysis appears to be quite promising, since they meet all requirements. 
With regard to Tennet\il{Tennet}, \citet[261]{Randal:1998} strongly argues in favor of an analysis of emphatic statements like (\ref{TenEmphExamp1}) as structures consisting of a predicate nominal plus a headless relative clause. 
The so-called `associative marker' (\am{}) linking the predicate nominal to the relative clause is also used with other nominal modifiers such as adjectives. 
Randal also states that the same utterance can be made in the longer variant in (\ref{TenEmphExamp2}), making the nominal predication more transparent. 
However, this approach does not explain why in the fuller version of the nominal predication both arguments are in the Accusative\is{case!individual forms!accusative} case. 
From the description of nominal predication in Tennet\il{Tennet}, one would expect the subject argument of the nominal predication (i.e. `Lokuli') to be in the Nominative case.

\begin{exe}\ex {Tennet\il{Tennet}} \citep[Surmic; Sudan;][261]{Randal:1998}\nopagebreak
\begin{xlist}
\ex\label{TenEmphExamp1}\gll\textipa{\textbf{lok\'uli}} \textipa{c\'I} \textipa{\'a-r\'uh} \textipa{loh\^am}\\
Lokuli \am{} \ipfv{}-beat Loham\\
\glt `It is Lokuli who is beating Loham.' 
\ex\label{TenEmphExamp2}\gll\textipa{\textbf{lok\'uli}} \textipa{n\'en\'e} \textipa{c\'I} \textipa{\'a-r\'uh} \textipa{loh\^am}\\
Lokuli the\_one \am{} \ipfv{}-beat Loham\\
\glt `Lokuli is the one who is beating Loham' 
\end{xlist}
\end{exe}

A good argument for the status of the initial noun as a predicate nominal is provided by Arbore\il{Arbore}.
In Arbore\il{Arbore} there is a special case-form used only for predicate nominals\is{nominal predication!predicate nominal} -- the so-called `Predicative' case -- which is also found on emphatic subjects (\ref{ArbEmph1}a), while non-emphatic subjects receive standard Nominative case (\ref{ArbEmph1}b). 
Also there is a reduced amount of morphological marking found on the verb in the emphatic context. 
For instance, the so-called `pre-verbal selector' (\pvs{}) is missing, a feature also associated with verbs in relative clauses \citep[315]{Hayward:1984}. 

\pagebreak

\begin{exe}\ex\label{ArbEmph1} {Arbore\il{Arbore}} \citep[Eastern Cushitic; Ethiopia; ][113, 114]{Hayward:1984}\nopagebreak
\begin{xlist}\ex\gll \textipa{\textbf{farawa}} \textipa{z\'eHe}\\
horse.\pred{} died\\
\glt `(A) horse died.' (answer to the constituent question `What died?')
\ex\gll \textipa{\textbf{faraw\'e}} \textipa{\textglotstop\'I-y} \textipa{zaHate}\\
horse.\nom{} \pvs{}-3\sg{} die.3\sg{}.\fem{}\\
\glt `(A) horse died.'
\end{xlist} 
\end{exe}\is{focus!cleft-analysis|)}

A more general structural explanation for the lack of case-marking on emphatic S-arguments is provided by \citet[60]{Donohue.Brown:1999} based on Nias\il{Nias}. 
They state that ``when an argument receives a degree of pragmatic salience, and appears focused\is{focus} or topicalized\is{topic}, then it is beyond the scope of the case-marking system.''
The argument that emphatic subjects are outside of the domain in which they can be assigned case by the verb (or any other node that in a given syntactic theory would assign case to the subject argument) also comprises the cleft-construction analysis, since in a biclausal structure an element in the first clause (i.e. the cleft) is outside of the domain in which the verb of the second clause can assign any case to it. 
However, it is not necessary to assume that the verb and logical subject are in different clauses for this more general analysis. 
It is sufficient for the emphatic subject to be located on a higher level of projection. 
However, because a claim like this presents a very abstract explanation, it is hard to confirm or disprove.

At least for Tennet\il{Tennet}, there can be made a clear case that it is not simply the pre-verbal position that prohibits Nominative case-marking on an argument, since there are also pre-verbal arguments with Nominative case-marking, as example (\ref{TenNonEmphExamp3}) illustrates. 
Other languages may of course behave differently in this respect, and one could still argue that the logical subject in (\ref{TenPreVerb}) and (b) are located in different structural positions.

\begin{exe}\ex {Tennet\il{Tennet}} \citep[Surmic; Sudan; ][261]{Randal:1998}\nopagebreak
\begin{xlist}
\ex\label{TenPreVerb}\gll\textipa{lok\'uli} \textipa{c\'I} \textipa{\'a-r\'uh} \textipa{loh\^am}\\
Lokuli \am{} \ipfv{}-beat Loham\\
\glt `It is Lokuli who is beating Loham.' 
\ex\label{TenNonEmphExamp3}\gll\textipa{\'Ijja} \textipa{zin} \textipa{w\'ala-i} \textipa{\'I-k\'Iya}\\
and then \propnoun-\nom{} \ipfv{}-come\\
\glt `And then Crow came.' \end{xlist}
\end{exe}

Further\is{case-marking!absence of!morphophonologically conditioned|(}, there is a whole different line of argumentation to explain the lack of S-case-marking on emphatic subjects that could be used in some languages. 
Instead\is{emphatic subject|(} of disallowing S-case-marking for structural (i.e. syntactic) reasons, morphology seems to be the important factor in this scenario. 
In the languages to which this explanation applies, the emphatic status of the subject argument is not only encoded by its position in the clause but by a special marker of discourse structure. 
The occurrence of this marker apparently blocks other markers such as overt S-case-markers. 
However, these languages do not have zero-coded emphatic subjects in the same sense as the languages previously discussed since emphatic subjects are overtly marked, though not for their role as the subject argument of a clause.
The Savosavo\il{Savosavo} example in (\ref{SavEmphEx}) illustrates this pattern (also see the discussion of Savosavo\il{Savosavo} in Section \ref{EmphPac}).

\begin{exe}\ex\label{SavEmphEx} {Savosavo\il{Savosavo}} \citep[Solomons East Papuan; Solomon Islands; ][221]{Wegener:2008}\nopagebreak
\gll \textbf{pa} \textbf{poi=e} te lo mane=la\\
one thing=\emphat{} \emphat{} 3\sg{}.\mas{}.\gen{}/\deter{}.\sg{}.\mas{} side=\loc{}.\mas{}\\ 
\glt `One thing (is) at its/the side.'
\end{exe}\is{case-marking!absence of!morphophonologically conditioned|)}\is{emphatic subject|)}
 

%In Boraana\il{Oromo (Boraana)} Oromo for example a polyfunctional marker referred to as `linker' by \citet{Stroomer:1995} is found on emphatic subjects. 
% NB: Try to figure out this marker and its functions! Apart from this context the linker consisting of a long cliticized underspecified vowel also attaches to nominals in contexts where modifiers such as possessors follow it. 

%\medskip
%{Boraana\il{Oromo (Boraana)} Oromo} \citep[Cushitic; ][122f.]{Stroomer:1995}

%\begin{exe} \ex\label{BorEmph1}\gll Amm=oo jaldees-ii hin-dabs-an-ne \textbf{kinniis=aa} dabs-at-e\\
%But=\lin{}  baboon-\nom{} NEG-win-MIVO-NEG.\pst{}, bee=\lin{} win-MIVO-3M.\pst{}\\
%`But the baboon did not overcome the bees, it was the bees that won.'
%\end{exe} 

%Exposed subjects \citet{Gueldemann:2008}
%The standard notions of discourse  analysis such as `contrastive focus' or `discontinuous topic' cannot  

%%%%%%%%%%%%%%%%%%%%%%%%%
%%%%% Section 5.4 %%%%%%%
%%%%%%%%%%%%%%%%%%%%%%%%%

\section{Overtly coded emphatic subjects}\label{focusNOM}

The\is{case-marking!optional|(} phenomenon of optional case-marking and more specifically optional ergativity has gained recent prominence in linguistic work \citep[see][]{OptionalErg}. 
It has been noted that case-markers are sometimes dropped in syntactic context in which they normally would be expected in a language.
\footnote{The influence of optional case-marking on the overall frequency of individual case-forms have already been briefly discussed in Section \ref{omitcase}.} 
The\is{emphatic subject|(} conditions for the dropping of overt case-marking often relate to information structure. 
Many languages which show this optional type of case-marking only employ the overt marking when the relevant constituent is in focus\is{focus}, while the marker is usually omitted otherwise. 
However, often there are additional contexts in which the markers can occur.\is{emphatic subject|)}
Special reference has been made to the optional nature of ergative case-marking in particular. 
For some languages it is noted that the optional ergative\is{case!individual forms!ergative} case is sometimes also used for intransitive S arguments. 
Thus, the languages would be better described as having optional marked"=nominative case-marking. 
However, because of the strong association between overt marking of agents and the label `ergative' that has been put forward by \citet{Dixon:1979}, the term `optional ergative' has stuck. 
Another reason might be that the overt markers are only rarely found in intransitive clauses because in these contexts a need for disambiguation of the participants arises far less frequently.
It is in such contexts that optional ergative\is{case!individual forms!ergative} markers are often found in. 
This practice is for example expressed in the following quote:  

\begin{quote}
I have labeled the affix \emph{-ro} [\dots] as an ergative marker. It is true that it only occurs on subjects of transitive verbs. However, it does not occur on all subjects of transitive verbs [\dots]. As in many PNG languages, it seems to occur most commonly where there is potential ambiguity as to which noun phrase is the subject. \citep[22]{Clifton:1997}
\end{quote}

Despite this claim that in Kaki Ae\il{Kaki Ae} the relevant marker occurs only on transitive subjects, \citet{Clifton:1997} provides some examples in which the same marker occurs on intransitive subjects as will be demonstrated in the following.\is{case-marking!optional|)}

Languages that have a marked"=nominative system only in specific contexts, such as focusing\is{focus} or disambiguation, are especially common in the Pacific region.
In his survey on participant marking in the so-called Papuan languages (a cover term for non-Austronesian languages of Oceania),\citet{Whitehead:1981} lists Siroi\il{Siroi}, Waskia\il{Waskia}, Kunimaipa\il{Kunimaipa}, and Nabak\il{Nabak} as having an optional S+A marker in combination with zero-coded P arguments.
\footnote{Instead of the labels S and A, \citet{Whitehead:1981} uses the terms `actor' and `agent'.} 
Further, the descriptions  of Kaki Ae\il{Kaki Ae}, Eipo\il{Eipo} and Yawuru\il{Yawuru}  suggest these languages are also of this special type marked"=nominative languages that employ overt marking only for emphatic subjects.
The pattern is exemplified in the following. 
The Waskia\il{Waskia} sentences in (\ref{WasEmphEx1}) demonstrate the alternation between emphatic contexts, in which the marker \emph{ke} follows the subject (\ref{WasEmphEx1}a), and non-emphatic contexts, which lack this marker (\ref{WasEmphEx1}b). 
The Kaki Ae\il{Kaki Ae} clause chain in (\ref{KakEmphEx1}) demonstrates the marking of emphatic and non-emphatic subjects. 
While the mother is marked with the marker \emph{-ro} (glossed as Ergative\is{case!individual forms!ergative} by Clifton) the noun \emph{aua} `children' that is the subject in the following clauses does not receive this marking. 
Thus, the status of the mother is marked as a participant that has not been present in the previous discourse.
 
\begin{exe}\ex\label{WasEmphEx1} {Waskia\il{Waskia}} \citep[Kowan; Papua New Guinea, Karkar Island; ][37, 17]{Ross:1978}\nopagebreak
\begin{xlist}
\ex\gll \textbf{nu} \textbf{ke} {taleng duap}\\
3\sg{}.\pronoun{} \nom{} policeman\\
\glt `He is a policemen (i.e. not someone else)'
\ex\gll \textbf{aga} \textbf{bawa} {taleng duap}\\
my brother policeman\\
\glt `My brother is a policeman.'
\end{xlist}
\end{exe}

\begin{exe} \ex\label{KakEmphEx1} {Kaki Ae\il{Kaki Ae}} \citep[Eleman; Papua New Guinea; ][52]{Clifton:1997}\nopagebreak
\gll naora\textbf{-ro} loea-ra-kape naora\textbf{-ro} u-ra-ha luera-ma \emph{aua} erahe uriri-RDP-isani naora kai w\"a'\"\i-isani-pe ko''ara oporo hu'a fua-isani koi'ara \"e'a rea-vere katlain ekakau himiri fua-isani a-isani-pe ava-isani\\
mother-\erg{} return-\irr{}-and mother-\erg{} call-\irr{}-3\ssbj{} then-\loc{} child 3\pl{} run-CONT-and mother to go\_down-and-? another wood block carry-and another that 3\sg{}-\poss{} fishing\_line something many carry-and get-and-? go\_up-and\\
\glt `The mother returns, the mother calls, and the children run down to the mother, some carry blocks of wood, some carry fishing line and many other things, they get them and go up.'
\end{exe}

While this general pattern of marking (intransitive) subjects only in con\-tras\-tive or emphatic contexts is quite widespread in the Pacific region (also extending to some Australian languages), the system is analyzed quite differently by different linguists. 
For all languages for which I could get information on, this structure, the variation between absence and presence of the marker is influenced by information structure. 
The languages exhibiting the system described in this section employ markers that have both discourse structure and case-marking properties. 
The linguists working on the relevant languages vary in assigning the pattern to either the domain of grammatical relations or pragmatic discourse relations.
These two domains of grammatical marking are often difficult to tease apart \citep[276]{Payne:1997}. 
Furthermore, there is often a strong correlation between a certain discourse status with a certain syntactic role. 
Therefore, it is no surprise that historical relations between the two types of markers are pretty common.  
It has already been discussed that markers of discourse relations have been proposed as a source for marked"=S systems (cf. Section \ref{historical}). 
It thus might be the case that the respective markers in the Pacific languages just presented are currently in a transitional phase from one of the domains to the other.
Some authors note that the optional (or focal) nominative\is{case!individual forms!nominative} marker in these languages is cognate to an ergative\is{case!individual forms!ergative} marker in related languages. 
However, the direction of change cannot be clearly established on this basis, since the ergative stage of related languages could either be more conservative or more innovative than the pattern of the language that does not use the marker to unambiguously encode grammatical relations. 
For other languages, both directions of change have been argued for: from discourse marking to case-marking and vice versa.
\citet{Shibatani:1991} discusses the grammaticalization of marking of a discourse category into the marking of grammatical relations, taking the example of topics\is{topic} and subjects. 
A change from case-marking to discourse marking, on the other hand, has been suggested, e.g. for the Australian language Jingulu \citep{Pensalfini:1999}.

The in-depth investigation of the focal marked"=nominative type according to the parameters of this study is particularly difficult. 
For most languages only a few odd examples of subject arguments receiving overt marking are given. 
These examples are usually accompanied by a mere impressionistic explanation of the factors leading to the presence of the marker (if any). 
From these few examples it is not possible to deduce in which of the contexts investigated in this study (other than basic (in)transitive clauses) the marker could or could not occur, given the relevant argument is emphatic. 
Therefore, the languages of this type will only be discussed in this chapter of the study, due to the missing data on for example emphatic existential subjects.

%%%%%%%%%%%%%%%%%%%%%%%%%
%%%%% Section 5.5 %%%%%%%
%%%%%%%%%%%%%%%%%%%%%%%%%

\section{Research questions}\label{EmphQuest}

The\is{emphatic subject|(} subsequent sections will provide an overview of the marking of emphatic subjects in marked"=S languages.
The languages can be classified as using one of three patterns (or a combination of these patterns).
These patterns are the following: 

\begin{enumerate}
\item Emphatic subjects do not receive S-case-marking.
\item Only emphatic subjects receive a special marker for the S-case.
\item Subjects receive S-case-marking independent of their discourse status.
\end{enumerate}

For each language of the sample, I investigate which of the three patterns are found.
Pattern 1 and 2 have been discussed in greater detail in the two previous sections.
In this section, I will give examples of the all of these possibilities of encoding emphatic subjects. 

Pattern 1 is exemplified by the Boraana\il{Oromo (Boraana)} Oromo sentence in (\ref{BorEmphExa}). 
While the first clause demonstrates the prototypical marking of subjects via Nominative\is{case!individual forms!nominative} case \emph{jaldees-ii}, the subject of the second clause (\emph{kinniis}) is focused\is{focus} and does not receive the Nominative suffix. Instead the so-called `linker clitic'  follows the emphatic noun.

\pagebreak
\begin{exe} \ex\label{BorEmphExa} {Oromo (Boraana\il{Oromo (Boraana)})} \citep[Eastern Cushitic; Ethiopia; ][122, 123]{Stroomer:1995}\nopagebreak
\gll Amm=oo jaldees-ii hin-dabs-an-ne \textbf{kinniis=aa} dabs-at-e\\
But=\lin{}  baboon-\nom{} \Neg{}-win-\Mid{}-\Neg{}.\pst{}, bee=\lin{} win-\Mid{}-3.\mas{}.\pst{}\\
\glt `But the baboon did not overcome the bees, it was the bees that won.'
\end{exe}

% !!!!!!!!!!!!!Also, there is a so-called Focus-marker in Boraana\il{Oromo (Boraana)} Oromo, try to find out more about the following constructions!!!!!!!!!!!!!!!!!!!!!!!!!
%
%{Oromo (Boraana\il{Oromo (Boraana)})} \citep[74,~89]{Stroomer:1995}
%\begin{exe} \ex\label{BorPassFoc} 
%\begin{xlist}\ex\gll fooni yaa d'aab-am-ani\\
%meat \foc{} cook-\pass{}-3\pl{}.\pst{}\\
%`The meat has been cooked.'
%
%\ex\gll mana yaa jaar-am-e\\
%house \foc{} build-\pass{}-3\sg{}.\pst{}\\
%`The house has been build.'
%
%\ex\gll sangaa yaa k'al-am-e\\
%ox \foc{} slaughter-\pass{}-3\sg{}.\pst{}\\
%`The ox has been killed.'
%\end{xlist} 
%\end{exe}  
%
%Compare also the impersonal construction in (\ref{BorImpers}) and the active sentence in (\ref{BorKill})
%
%{Oromo (Boraana\il{Oromo (Boraana)})}
%\begin{exe} \ex\label{BorImpers}\gll sangaa yaa ijees-ani\\
%ox \foc{} kill-3\pl{}.\pst{}\\
%`They killed the ox'
%\end{exe} \begin{flushright}\citet[90]{Stroomer:1995}\end{flushright}
%
%\begin{exe} \ex\label{BorKill}\gll namic-aa looni ijees-e\\
%man=\lin{} cow kill-3\sg{}.\mas{}.\pst{}\\
%`The man killed the cow.'
%\end{exe} \begin{flushright}\citet[105]{Stroomer:1995}\end{flushright}

The proposed origin of structures like this as cleft-constructions has been discussed before. 
Data on the marking of predicate nominals and the possibility of having zero-copulas in nominals predications will be provided in Section \ref{EmphSum}.
%The first pattern -- no marking on emphatic subjects -- most commonly is interpreted as a clefting strategy. The fronted subject is a predicate nominal that is further modified by the following headless-relative clause. In this interpretation one is dealing with a bi-clausal structure. Furthermore the case-marking is determined by the role of the logical subject. Since the NP co-referential with the subject is in fact a predicate nominal in these constructions, case-marking is expected to be identical to predicate nominals. A detailed discussion of the marking of predicate nominals can be found in Chapter \ref{nompred}, in the summary section , I will repeat the data on the marking of predicate nominals for the languages discussed in this chapter. 
% Furthermore it has been suggested, that the source of this pattern is probably a cleft-structure. If this is the case the emphatic S is really a predicate nominal in this context. Therefore, I will compare the case-form found on the fronted subjects with the marking on predicate nominals as discussed in Chapter 3. Furthermore the language in question must allow for zero-copulas in nominal predication and allow for headless relatives for this analysis in order to be plausible. All these factors will be taken into account in the following sections if information on them is available. 
Also, many of the languages with zero-coded emphatic subjects have in common that while their canonical word-order\is{word-order!basic} is verb-initial, emphatic subjects (or other element on which special emphasis is put) are placed before the verb\is{word-order!in emphatic contexts}. 
Therefore, in this chapter's summary I will also note the basic word-order(s)\is{word-order!basic} of each language discussed here. 
Examples of this strategy have already been discussed in Section \ref{exposed} in some detail.

The second pattern of case-marking on emphatic subjects is only found among the languages of the Pacific region.
\footnote{Special forms only found with emphatic subjects are more widespread. 
\citet[158--163]{Bruil:2014} discusses the subject-marker \emph{-bi} in Ecuadorian Siona (Tucanoan). 
The marker is used with focused\is{focus} subjects, but there might be additional uses, such as the disambiguation of arguments. 
Similar systems can be found in other languages of the same area. 
However, Ecuadorian Siona\il{Ecuadorian Siona} also has overt case-markers for (some types of) objects \citep[163--169]{Bruil:2014}, and thus does not fall under the definition of a marked"=S language. 
Yet it is very likely that languages with a pattern similar to the one found in the Pacific languages discussed in this section can be found elswhere.}
In some of the languages of this region, emphatic subjects receive a special marker while morphological marking of grammatical relations is absent in other contexts.
This pattern is exemplified by Was\-kia. 
In this language the marker \emph{ke} follows after subject arguments that are focused\is{focus} among other functions (\ref{WasEmphEx}a) while non-focused counterparts of these sentences the subject NP does not receive case-marking (\ref{WasEmphEx}b).

%\pagebreak

\begin{exe}\ex\label{WasEmphEx} {Waskia\il{Waskia}} \citep[Kowan; Papua New Guinea, Karkar Island; ][37, 17]{Ross:1978}\nopagebreak
\begin{xlist}
\ex\gll nu \textbf{ke} {taleng duap}\\
3\sg{}.\pronoun{} \nom{} policeman\\
\glt `He is a policemen (i.e. not someone else)'
\ex\gll aga bawa {taleng duap}\\
my brother policeman\\
\glt `My brother is a policeman.'
\end{xlist}
\end{exe}

% S-emph=S

The final pattern of interaction between case-marking and discourse structure is the absence of any interaction between the two systems. 
In other words subject-like arguments are marked with the overt S-case irrespective of their discourse structure relation.

This pattern is found in a number of languages of my sample. 
In Wappo\il{Wappo}, focused\is{focus} as well as non-focused subjects receive the Nominative\is{case!individual forms!nominative} ending in \emph{-i}. 
When the subject is focused the case-marked noun is followed by the focus-marker\is{focus!overt marker} \emph{lakhuh} (\ref{WapFoc}a). 
In sentences with non-focused subjects, this marker is not found (\ref{WapFoc}b).

\begin{exe}\ex\label{WapFoc} {Wappo\il{Wappo}} (Wappo-Yukian; California; \citealt[79]{Thompsonetal:2006}, \citealt[92]{Lietal:1977})\nopagebreak
\begin{xlist}
\ex\gll\textipa{ce} \textipa{\v saw\textbf{-i}} \textipa{\textbf{lakhuh}} \textipa{nuh-khe\textglotstop}\\
that bread-\nom{} \foc{} steal-\pass{}\\
\glt `It's the bread that got stolen.'
\ex\gll\textipa{mayi\v s\textbf{-i}} \textipa{ma\v cu\textglotstop-khe\textglotstop}\\
corn-\nom{} ash\_roast-\pass{}\\
\glt `The corn has been ash-roasted.'
\end{xlist}
\end{exe}

Another language in which the case-marking is identical for emphatic and non-emphatic subjects is K'abeena\il{K'abeena} (Eastern Cushitic).
Unlike in Wappo\il{Wappo}, the discourse prominence of the subject (or other argument or adjunct) is not encoded by overt morphological marking. 
This is rather achieved by putting a noun phrase into the position immediately preceding the verb.

\begin{exe}\ex {K'abeena\il{K'abeena}} \citep[Eastern Cushitic; Ethiopia; ][327, 104]{Crass:2005}\nopagebreak
\begin{xlist}\ex \gll bokku \textbf{womb\textsuperscript{u}} \'{}ijaar\'anu-r\textsuperscript{a} hecc'i gordanna fiilan\textsuperscript{u}\\
house.\acc{} K'abeena\il{K'abeena}.\nom{} build.\ipfv{}.3\sg{}.\mas{}-\tmp{} precede.\cvb{}.3\sg{}.\mas{} tree\_trunk\_for\_wall.\acc{} split.\ipfv{}.3\sg{}.\mas{}\\
\glt `When the K'abeena\il{K'abeena} build a house, they first split the tree trunks for the wall.'%\\ original translation: 
\ex\gll kamaal\textsuperscript{i} 'adbaareen\textsuperscript{i} 'ama'nanu-ba\\
Kamal.\nom{} familiar.\loc{} believe.\ipfv{}.3\sg{}.\mas{}-\Neg{}\\
\glt `Kamal does not believe in familiar spirits.'\\
 original translation: `Kamal glaubt nicht an Schutzgeister.'
\end{xlist}
\end{exe}

The following sections provide an in-depth study of the marking of emphatic subjects in marked"=S languages organized by areal and genealogical grouping into the Nilo-Saharan (\ref{EmphNilo}), Afro-Asiatic (\ref{EmphAfro}) Pacific (\ref{EmphPac}) and North-West-Ame\-ri\-can languages (\ref{EmphNA}). 
In the final section, the data is summarized and combined with additional information on the marking of nominal predications and basic word-order for each language (\ref{EmphSum}).

%%%%%%%%%%%%%%%%%%%%%%%%%
%%%%% Section 5.6 %%%%%%%
%%%%%%%%%%%%%%%%%%%%%%%%%

\section{Nilo-Saharan}\label{EmphNilo}

Most of the Nilo-Saharan marked"=S languages have a verb-initial basic word-order\is{word-order!basic}. 
Fronting of an argument to pre-verbal position leads to the loss of any overt case-marking. \citet{Koenig:2008} uses the slogan `no case before the verb' to allude to this property of these languages.  
The marked"=S languages of the Nilo-Saharan stock are almost completely uniform with regard to this expression of emphatic S arguments. 
Minor variations can be found in the Agar dialect of Dinka\il{Dinka (Agar)}, which has a topic-initial\is{topic} rather then verb-initial word-order\is{word-order!basic}, according to \citet{Andersen:1991}. 
Furthermore, in Tennet\il{Tennet} two types of pre-verbal subjects can be found, one with (zero) Accusative\is{case!individual forms!accusative} case-marking and the other one with regular Nominative case.

First, I will present the prototypical Nilo-Saharan system in which emphatic S arguments occur in pre-verbal position and are in the zero-coded accusative\is{case!individual forms!accusative} case, while non-emphatic post-verbal S arguments receive overt nominative\is{case!individual forms!nominative} case-marking. 
This system is found in Datooga\il{Datooga} (\ref{DatEmph}), Turkana\il{Turkana} (\ref{TurEmph}) and Nandi\il{Nandi} (\ref{NanEmph}); the (a) examples demonstrate the emphatic construction while the (b) examples are non-emphatic contexts. Maa\il{Maa} behaves in the same fashion (\ref{MaaEmph}).
%\footnote{The Maa examples have been cited after \citet[667, 666]{Koenig:2006}, who does not give page references for these examples.}

\begin{exe}\ex\label{DatEmph} {Datooga\il{Datooga}} \citep[Nilotic; Tanzania; ][183]{Kiessling:2007}\nopagebreak
\begin{xlist} 
\ex\gll \textipa{{b\'uun\`ee}} \textipa{{s\'uurj\'a}} \textipa{\`aa} \textipa{n\`I-y\^IIn} \textipa{d\`ab\'I t\'a-\textltailn\`aw\'a} \textipa{N\`\ae\ae\textltailn\r*i}\\
people.\acc{} others.\acc{} \topic{} \sbj{}3.\prf{}-put\_on weapons.\acc{}.\cs{}-3\pl{}.\poss{} down\\
\glt `Other people had put down their weapons.'
\ex\gll \textipa{g\`a-b\`IIkt\'a} \textipa{\textbf{q\'aar\`eem\`aNg\`a}} \textipa{\textbf{s\`uurj\'a}} \textipa{q\`oo}\\
\sbj{}3-return youths.\nom{} others.\nom{} home\\
\glt `Other youths went home.'
\end{xlist}
\end{exe}

\pagebreak
\begin{exe}\ex\label{TurEmph}\begin{xlist} {Turkana\il{Turkana}} \citep[Nilotic; Kenya; ][82]{Dimmendaal:1982}\nopagebreak
\ex\gll\textipa{\textbf{Na-atuk\`{}}} \textipa{\textbf{Na-arey\`{}}} \textipa{m\`ake\`{}} \textipa{e-yak\`a-sI} \textipa{a-yON\`{}}\\
\Non\_{}\neu{}.\pl{}-cow.\acc{} \Non{}\_\neu{}.\pl{}-two.\acc{} self 3-be-\pl{} 1\sg{}.\acc{}\\
\glt `Two cows is all I have.'
\ex\gll\textipa{a-yON\`{}} \textipa{e-yak\`a-sI} \textipa{\textbf{Na-\`at\`uk}} \textipa{\textbf{Na-\`ar\`ey}} \textipa{m\`ake\`{}}\\
1\sg{}.\acc{} 3-be-\pl{} \Non{}\_\neu{}.\pl{}-cow.\nom{}  \Non{}\_\neu{}.\pl{}-two.\nom{} self\\
\glt `I only have two cows.'
\end{xlist} 
\end{exe}

\begin{exe}
\ex \label{MaaEmph} {Maa\il{Maa}} (Nilotic; Kenya; \citealp[667, 666]{Koenig:2006} after {Tucker \& Mpaayei 1955}\nocite{Tucker:1955})
\begin{xlist}
\ex\gll \textipa{en-t\'it\'o} \textipa{na-d\'Ol} \textipa{nIny\'E}\\
\fem{}.\sg{}-girl.\acc{} \relativ{}.\fem{}.\sg{}-see 3\sg{}.\acc{}\\
\glt `I is the who sees him.'
\ex\gll \textipa{\'E-d\'Ol-\'Ita} \textipa{Ol-\textbf{k\'It\'EN}} \textipa{en-k\'O\'it\'o\'I}\\
3\sg{}-see-\prog{} \mas{}\sg{}-ox.\nom{} \fem{}.\sg{}-road.\acc{}\\
\glt `The ox sees the road.'
\end{xlist}
\end{exe}

%Fronted subjects, as in focus constructions \citep[11f.]{Koenig:2006}, are in Accusative case. \citet[132]{Melcuk:1997} analyzes these as topics. 

%{Murle\il{Murle}}
%%
%\begin{exe}\ex
%\begin{xlist} 
%\ex\gll
%\ex\gll
%\end{xlist}
%\end{exe}
%
%The topic in a topic-comment-clause is in the accusative form \citep[113]{Arensen:1982}

\begin{exe}\ex\label{NanEmph} {Nandi\il{Nandi}} \citep[Nilotic; Kenya; ][124, 125]{Creider:1989}\nopagebreak
\begin{xlist}
\ex\gll \textipa{{kipe:t}} \textipa{k\'o} \textipa{k\^e:r-\'ey} \textipa{la:kw\'e:t}\\
Kibet \partic{} see-\ipfv{} child.\acc{}\\
\glt `{KIBET} is looking at the child.'
\ex\gll\textipa{k\`e:r-\'ey} \textipa{\textbf{k\'Ipe:t}} \textipa{la:kw\'e:t}\\
see-\ipfv{} Kibet.\nom{} child.\acc{}\\
\glt `Kibet is looking at the child.'
\end{xlist} 
\end{exe} 

In addition to the structure demonstrated in (\ref{NanEmph}) above, \citet[124--125]{Creider:1989} discuss a second type of topicalization for Nandi\il{Nandi}, namely topic final sentences. 
The structure demonstrated above is referred to as `topic fronting'. 
However, from \citeauthor{Creider:1989}'s (\citeyear[150]{Creider:1989}) description of the use of this topic-final construction, it seems clear that this is rather a focus\is{focus} construction in the terminology introduced in Section \ref{discourse}.
Unlike in the construction with the fronted S argument, S arguments in the topic-final structure keep their Nominative\is{case!individual forms!nominative} tonal\is{case-marking!via tone} shape (\ref{NanAftTop}).

\enlargethispage{\baselineskip}

\begin{exe} \ex\label{NanAftTop} {Nandi\il{Nandi}} \citep[150]{Creider:1989}\nopagebreak
\gll\textipa{k\`e:r=\'ey} \textipa{kipe:t} \textipa{\textbf{k\'Ipro:no}}\\
see-\ipfv{} Kibet.\acc{} Kiprono.\nom{}\\
\glt `Kiprono sees  Kibet.'
\end{exe}

In the Agar dialect of Dinka\il{Dinka (Agar)}, basically the same situation  is found as in the other Nilo-Saharan languages. 
Pre-verbal subjects do not receive Nominative case-marking (\ref{DinEmph}a, b), which they would receive in post-verbal position. 
The difference from the languages described previously is that there does not seem to be a verb-initial basic word-order in Dinka\il{Dinka (Agar)} (or at least not anymore).  
\citet{Andersen:1991} analyzes Agar Dinka\il{Dinka (Agar)} as a topic-first language\is{topic}. 
That means that whichever element occurs in clause-initial position is the topic, usually this is the S or A argument. 
Only if some other argument is the topic of the discourse, like the P argument occurring sentence-initially in example (\ref{DinEmph}c), the subject is marked with the overt Nominative case. 
In addition, verbal agreement\is{verbal indexing} is with the topic\is{topic} rather then the subject \citep{Andersen:1991}. 

\begin{exe}\ex\label{DinEmph} {Dinka\il{Dinka (Agar)} (Agar Dialect)} \citep[Nilotic; Sudan; ][272, 273]{Andersen:1991}\nopagebreak
\begin{xlist} 
\ex\gll \textipa{\textipa{l\"*{\'a}y}} \textipa{\~*{\`a}-ku\~*{\`a}aN}\\
animal.\acc{} \decl{}-swim\\
\glt `The animal is swimming.'
\ex\gll \textipa{\textbf{l\"*{\'a}y}} \textipa{\~*{\`a}-n\"*{\`a}k} \textipa{ra\~*{\`a}an}\\
animal.\acc{} \decl{}-kill person.\acc{}\\
\glt `The animal is killing the person.'
\ex\gll \textipa{ra\~*{\`a}an} \textipa{\~*{\`a}-n\"*{\`a}k} \textipa{\textbf{l\"*{\`a}y}}\\
person.\acc{} \decl{}-kill animal.\nom{}\\
\glt `The animal is killing THE PERSON.' %check translations and the case-forms are the same here
\end{xlist}
\end{exe}

A different variation of the Nilo-Saharan pattern `no case before the verb' is found in Tennet\il{Tennet}.
This language distinguishes between two different S-initial emphatic structures. 
The first construction behaves like  the examples discussed before, as the fronted subject is in the zero-coded Accusative\is{case!individual forms!accusative} case (\ref{TenEmph}a).
\citet{Randal:1998} explicitly states that this construction is an instance of clefting and the fronted subject is part of a nominal predication.
The other construction used to put emphasis on an argument also involves fronting of this argument before the verb. 
However, this construction is not a cleft, as can be seen by the lack of the Associative Marker (\am{}), which among other function introduces relative clauses. 
Also, the Nominative\is{case!individual forms!nominative}  case-marking is retained if the subject is fronted using this construction (\ref{TenEmph}b).

\enlargethispage{\baselineskip}
\begin{exe}\ex\label{TenEmph} {Tennet\il{Tennet}} \citep[Surmic; Sudan; ][261]{Randal:1998}\nopagebreak
\begin{xlist}
\ex\gll\textipa{\textbf{lok\'uli}} \textipa{c\'I} \textipa{\'a-r\'uh} \textipa{loh\^am}\\
Lokuli.\acc{} \am{} \ipfv{}-beat Loham.\acc{}\\
\glt `It is Lokuli who is beating Loham.' 
\ex\gll\textipa{\'Ijja} \textipa{zin} \textipa{\textbf{w\'ala-i}} \textipa{\'I-k\'Iya}\\
and then \propnoun-\nom{} \ipfv{}-come\\
\glt `And then Crow came.' 
\end{xlist}
\end{exe}

Table~\ref{OverviewEmphNilo} summarizes the marking of emphatic subjects in the Nilo-Saharan marked"=S languages.
All languages use overt Nominative case-marking for non-emphatic subject arguments.
For emphatic subjects, all languages have at least one construction that marks this argument with the zero-coded Accusative\is{case!individual forms!accusative} case.
Tennet\il{Tennet} and Nandi\il{Nandi} have different constructions that can be employed to encode emphatic subjects, so that these arguments are either in the zero-coded Accusative\is{case!individual forms!accusative} or in the overtly coded Nominative case. 
Further, the marking of the predicate nominal (Chapter\~ref{nompred}) coincides with the predominant pattern of marking emphatic subjects for all languages. This supports the cleft-analysis of these structures to some extent.  


\begin{table}[h]
\begin{center}
\caption{Marking of emphatic S arguments in the Nilo-Saharan marked"=S languages}\label{OverviewEmphNilo}%\\
\begin{tabular}{lccc}
\hline \hline
\bfseries language&\bfseries non-emphatic S &\bfseries emphatic S &\bfseries predicate nominal \\
\hline
Datooga\il{Datooga}&\textbf{\nom{}}&\acc{}&\acc{}\\
%\hdashline
Dinka\il{Dinka (Agar)} (Agar)&\textbf{\nom{}}&\acc{}&{-}\\
%\hdashline
Nandi\il{Nandi}&\textbf{\nom{}}&\acc{}/\textbf{\nom{}}&\acc{}\\
%\hdashline
Tennet\il{Tennet}&\textbf{\nom{}}&\acc{}/\textbf{\nom{}}&\acc{}\\
%\hdashline
Turkana\il{Turkana}&\textbf{\nom{}}&\acc{}&\acc{}\\
\hline \hline
\end{tabular}
\end{center}
\end{table}


%%%%%%%%%%%%%%%%%%%%%%%%%
%%%%% Section 5.7 %%%%%%%
%%%%%%%%%%%%%%%%%%%%%%%%%

\section{Afro-Asiatic}\label{EmphAfro}

In the Afro-Asiatic languages, there is no uniform pattern for emphatic contexts. 
There are some languages that use cleft-structures for focusing\is{focus}, and thus use the same case-form on the emphatic S as on predicate nominals, whether this is the zero-coded form or not. 
In other languages, however, emphatic subjects use the nominative case-form, differently from the case-marking used for predicate nominals.

A cleft-strategy for emphatic subjects is used in Boraana\il{Oromo (Boraana)} and Harar\il{Oromo (Harar)}a Oromo as well as in Arbore\il{Arbore}. 
%For focussing? a constituent the emphatic marker \emph{-uma} is suffixed to a noun \citep[35]{Stroomer:1995} (maybe non-subject topics) %Boraana\il{Oromo (Boraana)}
In Boraana\il{Oromo (Boraana)} Oromo, the so-called `linker' (functioning as a Genitive, among other uses) attaches to focused\is{focus} constituents. 
The range of functions of this marker is pretty wide; one of them is to introduce relative clauses. 
This makes a cleft-analysis of this structure very plausible. 
Subjects which are focused via the linker precede their cleft-sentence and are zero-coded for case (\ref{BorEmph}). 
Harar\il{Oromo (Harar)} Oromo behaves similarly: emphatic A arguments do not receive Nominative\is{case!individual forms!nominative}  case-marking. 
Instead they receive some other marking, which consists of the lengthening of their final vowel and attaching the (non-obligatory) suffix \emph{{'}-t\'uu}.\footnote{The diacritic before the affix indicates a tonal change of the stem.} 
In this construction the agreement on the verb is invariably third person masculine \citep[108]{Owens:1985}. 
This indicates that the logical subject does not function as the syntactic subject in these contexts and is possibly located outside of the clause containing the verb.  %Harar\il{Oromo (Harar)}

In\is{case!individual forms!predicative|(} Arbore\il{Arbore}, emphatic S arguments are in the Predicative case (\ref{ArbEmph}a). 
In this language, it is thus clear that those elements are predicate nominals. 
This in turn strengthens the cleft hypothesis, as already noted in Section~\ref{exposed}. 
Also note that the verb does not agree with the subject in this construction \citep[113--114]{Hayward:1984}, indicating that those clauses are probably not simply derived from their counterparts with unmarked information structure (\ref{ArbEmph}b).

\begin{exe} \ex\label{BorEmph} {Oromo (Boraana\il{Oromo (Boraana)})} \citep[Eastern Cushitic; Ethiopia; ][122, 123]{Stroomer:1995}\nopagebreak
\gll Amm=oo jaldees-ii hin-dabs-an-ne \textbf{kinniis=aa} dabs-at-e\\
But=\lin{}  baboon-\nom{} \Neg{}-win-\Mid{}-\Neg{}.\pst{}, bee=\lin{} win-\Mid{}-3\mas{}.\pst{}\\
\glt `But the baboon did not overcome the bees, it was the bees that won.'
\end{exe} 

\begin{exe} \ex\label{HarEmph} {Oromo (Harar\il{Oromo (Harar)})} \citep[Eastern Cushitic; Ethiopia; ][108]{Owens:1985}\nopagebreak
\gll\textipa{makiin\'aa} \textipa{tiyy\'a-a} \textipa{d\'iim-tuu}\\
car my.\foc{} red-\fem{}\\
\glt `It is my car that is red.' \end{exe} 

\begin{exe}\ex\label{ArbEmph} {Arbore\il{Arbore}} \citep[Eastern Cushitic; Ethiopia; ][113, 114]{Hayward:1984}\nopagebreak
\begin{xlist}\ex\gll \textipa{\textbf{farawa}} \textipa{z\'eHe}\\
horse.\pred{} died\\
\glt `(A) horse died.' (answer to the constituent question `What died?')
\ex\gll \textipa{\textbf{faraw\'e}} \textipa{\textglotstop\'i-y} \textipa{zaHate}\\
horse.\nom{} \pvs{}-3\sg{} die.3\sg{}.\fem{}\\
\glt `(A) horse died.'
\end{xlist} 
\end{exe}\is{case!individual forms!predicative|)}

%\medskip
No difference in case-marking of emphatic and non-emphatic subjects seems to exist in Gamo\il{Gamo} and K'abeena\il{K'abeena}. 
The details of information-structure marking in Gamo\il{Gamo} are interpreted differently by different scholars. 
The main problem is probably the different use of terminology, which is not uncommon in the domain of information structure \citep[262]{Payne:1997}.
What \citet[359--360]{Hompo:1990} refers to as `focused elements' are moved to sentence initial position. 
This position is also the canonical position of the subject. 
Sentence-initial as well as non-sentence-initial subjects (with other constituents in `focus') are marked with the Nominative case-marker.
In contrast, \citet[222]{Taylor:1994} claims that in standard SOV order it is the object that is focused\is{focus}, and that altering the order to OSV results in subject focus. 
He also finds no case-marking alternations between the different word-orders\is{word-order!in emphatic contexts} (\ref{GamEmph}). 
Given that Hompo analyses the clause initial position as the canonical subject position, her focused elements can probably be understood as discourse topics\is{topic}, an analysis that would be compatible with Taylor's analysis.   %{Gamo\il{Gamo}}

The situation in K'abeena\il{K'abeena} resembles the one described for Gamo\il{Gamo} by Taylor. 
Focused arguments immediately precede the verb, where they receive the same case-marking as in unfocused position \citep[327]{Crass:2005}.
There is also the emphatic suffix \emph{-n\textsuperscript{u}} \citep[256]{Crass:2005}, though the interaction between this suffix and the word-order alternations is not discussed by Crass. 
The data shows that both subjects immediately precede the verb (\ref{KabEmph}a) and S arguments marked with the emphatic affix (\ref{KabEmph}b) are in the Nominative case. %{K'abeena\il{K'abeena}}

\begin{exe}\ex\label{GamEmph} {Gamo\il{Gamo}} \citep[Omotic; Ethiopia; ][222]{Taylor:1994}\nopagebreak
\begin{xlist} 
\ex\gll para \v c'abo\textbf{-i} yides\\
horse.\acc{} Chabo-\nom{} see.\prf{}.3\sg{}.\mas{}\\
\glt `It was Chabo who saw a horse.'
\ex\gll \v c'abo\textbf{-i} para yides\\
Chabo-\nom{} horse.\acc{} see.\prf{}.3\sg{}.\mas{}\\
\glt `Chabo saw a horse.'
\end{xlist}
\end{exe}

\enlargethispage{2\baselineskip}
\begin{exe}\ex\label{KabEmph} {K'abeena\il{K'abeena}} \citep[Eastern Cushitic; Ethiopia; ][327, 256]{Crass:2005}\nopagebreak
\begin{xlist} 
\ex\gll diini-'ne wakk'eeccu \textbf{k'aricc\textsuperscript{u}} mazaaran\textsuperscript{u}\\
religion.\gen{}.1\pl{}.\poss{} path.\acc{} god.\nom{} prepare.\ipfv{}.3\sg{}.\mas{}\\
\glt `It is God who prepares the path of our religion.' \\
original translation: `Den Weg unserer Religion bereitet Gott.'
\ex\gll \textbf{'\'ani-'n\textsuperscript{u}} gorru 'ataalaamm\textsuperscript{i} \\
1\sg{}.\nom{}-\emphat{} hunger.\acc{} be\_able.\ipfv{}.1\sg{} \\
\glt `When it comes to me, I can cope with hunger well.' \\
original translation: Was mich betrifft, ich kann Hunger gut ertragen.'
\end{xlist}
\end{exe}

An overview of discourse-motivated marking in the Afro-Asiatic marked"=S languages is presented in Table~\ref{OverviewEmphAfro}.
Gamo\il{Gamo} and K'abeena\il{K'abeena} do not use any case-marking in the emphatic context that diverges from non-emphatic subjects. 
They employ the regular Nominative case, and thus a different case-form than with predicate nominals. 
Another form of overt marking of emphatic subjects is found in Arbore\il{Arbore}. 
Instead of using the Nominative  case, the Predicative case is employed. 
This case is otherwise used to encode predicate nominals, and thus Arbore\il{Arbore} is a good example of the cleft-strategy to encode emphatic subjects. 
Boraana\il{Oromo (Boraana)} and Harar\il{Oromo (Harar)} Oromo use Accusative\is{case!individual forms!accusative} nouns to encode emphatic subjects. 
In both languages the relevant construction attaches additional material to the emphatic noun. 
In the Boraana\il{Oromo (Boraana)} dialect, this material (the so-called `linker') is used to connect nouns to relative clauses that modify them, among other functions. 
This supports the cleft analysis of this structure.

\begin{table}[t,h]
\begin{center}
\caption{Marking of emphatic S arguments in the Afro-Asiatic marked"=S languages}\label{OverviewEmphAfro}%\\
\begin{tabular}{lccc}
\hline \hline
\bfseries language&\bfseries non-emphatic S &\bfseries emphatic S &\bfseries predicate nominal \\
\hline
Arbore\il{Arbore}&\textbf{\nom{}}&\textbf{\pred{}}&\textbf{\pred{}}\\
%\hdashline
Gamo\il{Gamo}&\textbf{\nom{}}&\textbf{\nom{}}&\acc{}\\
%\hdashline
K'abeena\il{K'abeena}&\textbf{\nom{}}&\textbf{\nom{}}&\acc{}\\
%\hdashline
Oromo (Boraana\il{Oromo (Boraana)})&\textbf{\nom{}}&\acc{}+{\textbf{\lin{}}}&\acc{}\\
%\hdashline
Oromo (Harar\il{Oromo (Harar)})&\textbf{\nom{}}&\foc{}&\acc{}\\
\hline \hline
\end{tabular}
\end{center}
\end{table}


%%%%%%%%%%%%%%%%%%%%%%%%%
%%%%% Section 5.8 %%%%%%%
%%%%%%%%%%%%%%%%%%%%%%%%%

\section{Pacific}\label{EmphPac}

The languages of the Pacific region exhibit the most diversity in the interaction between case-marking and discourse structure in my sample. 
Aji\"e\il{Aji\"e} and Nias\il{Nias} behave similarly to the Nilo-Saharan languages. 
In both languages, discourse-prominent arguments are fronted to pre-verbal position and do not receive case-marking. 
In Savosavo\il{Savosavo}, the situation is a bit more complex. 
Subjects marked with the so-called `emphatic marker' do not receive their usual Nominative case-marking and also occur in clause initial position. 
However, there are also instances in which Nominative case-marking is found on subjects that have the same discourse status properties as the emphatically-marked subjects but that lack the emphatic marker. 
Further, there are a number of languages in this area that only mark subjects that are in some prominent discourse relation with an overt marker. 
These systems are usually not treated as proper case-marking systems, but most grammar writers acknowledge that the relevant marker is found with subject arguments only (or at least predominantly).

Nias\il{Nias} basic word-order\is{word-order!basic} is verb-initial. 
To put special emphasis on an argument, it can be fronted to pre-verbal position -- this construction is analyzed as encoding both topic\is{topic} or focus\is{focus} function by \citet[60]{Donohue.Brown:1999}. 
In this position, all arguments take the Unmutated nominal forms. 
Thus, if the fronted argument corresponds to an argument that would be in the Mutated form of a noun in a basic clause, the case-marking will be dropped in emphatic contexts (\ref{NiaEmph}). 
Aji\"e\il{Aji\"e} is also verb-initial in its basic word-order\is{word-order!basic}.  
The preposition \emph{na} marks S and A arguments (\ref{AjiEmph}). 
This marker does not appear on S or A arguments in pre-verbal position.\footnote{Claire Moyse-Faurie\aimention{Moyse-Faurie, Claire} (p.c. at the \emph{Syntax of the World's Languages} conference in Berlin on 25.09.2008) analyzes the marker glossed as `pause' by Lichtenberk as a focus-marker\is{focus!overt marker} that she regards as obligatory in this context.} 

\begin{exe}\ex\label{NiaEmph} {Nias\il{Nias}} \citep[Sundic; Indonesia; ][262]{Brown:2001}\nopagebreak
\begin{xlist}\ex\gll
\textbf{si'o} h\"o'\"o ma+i-taru-'\"o ba dan\"o\\
stick \dist{} \prf{}=3\sg{}.\rls{}-plant-\transitiv{} \loc{} ground.\mut{}\\
\glt `That stick he planted in the ground.'
\ex\gll 
i-taru-'\"o \textbf{zi'o} h\"o'\"o ba dan\"o\\
\prf{}-3\sg{}.\rls{}.plant-\transitiv{} stick.\mut{} \dist{} \loc{} ground.\mut{}\\
\glt `He planted that stick in the ground.'
\end{xlist}\end{exe}

\begin{exe}\ex\label{AjiEmph} {Aji\"e\il{Aji\"e}} \citetext{Oceanic; New Caledonia; \citealt[111]{Lichtenberk:1978} after \citealt[313]{Fontinelle:1976}}\nopagebreak
\begin{xlist}
\ex\gll na kuru \textbf{na} \textbf{tawa}\\
3\sg{} sleep \nom{} dog\\
\glt `The dog sleeps.'
\ex\gll \textbf{tawa} (we) na kuru\\
dog (`pause') 3\sg{} sleep\\
\glt `As for the dog, it sleeps.'
\end{xlist}
\end{exe} 


%\begin{exe}\ex\label{Aji\"e\il{Aji\"e}mph2}
%\begin{xlist}\begin{multicols}{2}
%\ex\gll na kuru \textbf{na} \textbf{tawa}\\
%3\sg{} sleep \nom{} dog\\
%`The dog sleeps.'
%\columnbreak
%\ex\gll \textbf{tawa} (we) na kuru\\
%dog (`pause') 3\sg{} sleep\\
%`As for the dog, it sleeps.'
%\end{multicols}
%\end{xlist}
%\begin{flushright}\citet[111]{Lichtenberk:1978} from \citet[313]{Fontinelle:1976}\end{flushright}
%\end{exe} 

%\citet[313]{Fontinelle:1976} gives the second example in the following form:

%\begin{exe} \ex\gll\textipa{tawa} (\textipa{rr\'e}), (\textipa{wE}) \textipa{na} \textipa{kuru}\\
%dog (\dem{}) (pause\_mat\'erialis\'ee\_ou\_non) \nom{} sleep\\
%`As for the dog, it sleeps.'
%\end{exe}

%She states that`La pause, qui isole le premier terme (celui-ci est d'ailleurs d\'etermin\'e la plupart du temps par un d\'emonstratif), correspond \`a une mise en valeur et remplace ;'indicateur de sujet.' (ibid.)


In Savosavo\il{Savosavo}, on the other hand, the basic word-order\is{word-order!basic} is SOV when constituents are realized as full NPs, a situation that, however, seldom occurs in naturalistic data \citep[199--200]{Wegener:2008}. 
The emphatic marker \emph{=e} (and its set of allomorphs used when cliticizing to pronouns) is used very often in Savosavo\il{Savosavo}.
\footnote{\citet[221]{Wegener:2008} states that it is the second most common morpheme in her data.}
The element marked with the emphatic enclitic is fronted.
The exact function of this marker does not seem to correspond to any of the categories usually distinguished in the linguistic analysis of information structure.
\citet[228--229]{Wegener:2008} describes the marker as having to do with information structure, though it does not occur exclusively on either focused\is{focus} elements or topic\is{topic} expressions. Furthermore, perfectly grammatical sentences with elements that would be analyzed as corresponding to one of these functions without the emphatic marker also occur. 

The marker \emph{=e} is found in non-verbal as well as in verbal clauses. 
In non-verbal clauses, it attaches either to the subject (\ref{SavEmphNVer}a) or predicate (\ref{SavEmphNVer}b). 
When attaching to the subject of the clause, Nominative case-marking does not appear on this argument (\ref{SavEmphNVer}a). 
In verbal clauses, the emphatic marker can also attach to the subject argument (though this seldom occurs) and like in non-verbal clauses, the Nominative-marker does not occur in this case (\ref{SavEmphVerb}a). 
Nominative case appears to be blocked by the emphatic marker, possibly due to morphological restrictions\is{case-marking!absence of!morphophonologically conditioned}. 

\begin{exe}\ex\label{SavEmphNVer} {Savosavo\il{Savosavo}} \citep[Solomons East Papuan; Solomon Islands; ][221, 222]{Wegener:2008}\nopagebreak
\begin{xlist}\ex\gll \textbf{pa} \textbf{poi=e} te lo mane=la\\
one thing=\emphat{} \emphat{} 3\sg{}.\mas{}.\gen{}/\deter{}.\sg{}.\mas{} side=\loc{}.\mas{}\\ 
\glt `One thing (is) at its/the side.'
\ex\gll apoi ata=e te lo \textbf{keva=na}\\
because here=\emphat{} \emphat{} \deter{}.\sg{}.\mas{} path=\nom{}\\
\glt `Because here (is) the road.'
\end{xlist}
\end{exe}
\begin{exe}\ex\label{SavEmphVerb} {Savosavo\il{Savosavo}} \citep[225, 144]{Wegener:2008}\nopagebreak
\begin{xlist}\ex\gll \textbf{ave=ve} gazu te livu-li Australia l-au bo-i\\
1\pl{}.\excl{}=\emphat{}.1\pl{}.\excl{} ripe\_coconut \emphat{} carry-3\sg{}.\mas{}.\obj{} Australia 3\sg{}.\mas{}.\obj{}-take go-\fin{}\\
\glt `We shipped ripe coconuts to Australia.' %[225]
%\enlargethispage{\baselineskip}
\pagebreak
\ex\gll \textbf{Jeffi=na}, baigho=e lo-va ela sua ko adaki nyuba=ka sua pa ghanaghana=na\\
Jeff=\nom{} \Neg{}.exist=\emphat{} 3\sg{}.\mas{}=\gen{}.\mas{} one \att{}.\sg{}.\mas{} \deter{}.\sg{}.\fem{} woman child=\loc{}.\fem{} \att{}.\sg{}.\mas{} one thought.\nom{}\\
\glt `Jeff, he didn't have any thought whatsoever about/because of the woman.' %[144]
\end{xlist} 
\end{exe}

However, there are instances when an emphatic subject -- even in clauses with the emphatic marker -- receives Nominative case-marking, as in the following two constructions.
Often, the subject is repeated as a pronoun at the end  of a clause. 
In these cases, the Nominative case occurs on this final pronoun even if the preceding noun phrase referring to the subject argument does not receive case-marking due to the occurrence of the emphatic marker (\ref{SavEmphDoubSbj}).
Also, if the full NP referring to the subject argument of a clause is added as an afterthought\is{topic!afterthought} topic, while the clause internal subject referent is realized as a subject enclitic, the postposed subject is marked with Nominative\is{case!individual forms!nominative}  case (\ref{SavEmphAft}).

\begin{exe}\ex {Savosavo\il{Savosavo}} \citep[224, 143]{Wegener:2008}\nopagebreak
\begin{xlist}
\ex\label{SavEmphDoubSbj}\gll ai to edo Fiji sua \textbf{mapa=lo=e} to boboragha mapa=lo=e \textbf{to=na}\\
this \deter{}.\du{} two Fiji \att{} person=\du{}=\emphat{} \deter{}.\du{} black person=\du{}=\emphat{} 3\du{}=\nom{}\\
\glt `These two Fijians, they (were) black people.'
\ex\label{SavEmphAft}\gll zu sesepi=la=ti=lo te alu kozi-zu,  lo \textbf{mapa=na}\\
and Sesepi=\loc{}.\mas{}=\prox{}=3\sg{}.\mas{}.\nom{} \emphat{} stand face.\pst{}.\ipfv{} \deter{}.\sg{}.\mas{} person=\nom{}\\
\glt `and he stands facing close to Sesepi, the man.'%[143]
\end{xlist} 
\end{exe}


Also in the languages of the Pacific region, a quite different pattern is found, namely, nominative case-marking only with emphatic arguments.
This pattern is exemplified by Waskia\il{Waskia}, for which \citet[36]{Ross:1978} notes that
``the subject-marker \emph{ke} is intimately related to topicalisation.''
The following examples demonstrate the usage of this marker and its absence in non-emphatic contexts on the same grammatical relations.
\footnote{At least the first two contexts -- answers to constituent questions and correction of wrong assumptions -- can be considered clear examples of focus\is{focus} constructions, despite Ross' classification as `topicalisation'.}
Subject arguments are marked by \emph{ke} if they are answers to constituent questions (\ref{WasFOCCount}) or if the speaker wants to correct a wrong assumption about the subjects of nominal predications (\ref{WasFOCNomPred}) or the S argument of any verb (\ref{WasFOCQ}).
%In the non-focused counterparts of these sentences the subject NP does not receive case-marking with the marker \emph{ke} (\ref{WasNFOC}).

\begin{exe}\ex\label{WasFOCCount} {Waskia\il{Waskia}} \citep[Kowan; Papua New Guinea, Karkar Island; ][37, 31]{Ross:1978}\nopagebreak
\begin{xlist}
\ex\gll \textbf{aweri} \textbf{ke} bamban tagiram ? \quad -- gagi ke\\
who \nom{} fish caught {} \quad {} Gagi \nom{}\\
\glt `Who caught the fish? \quad -- Gagi (did)
\ex\gll \textbf{gagi} kasili arigam\\
Gagi snake saw\\
\glt `Gagi saw the snake.' %[31]
\end{xlist}
\end{exe}

%{Waskia\il{Waskia}} \citep[17,~11]{Ross:1978}  
\begin{exe}\ex\label{WasFOCNomPred} {Waskia\il{Waskia}} \citep[17, 11]{Ross:1978}\nopagebreak
\begin{xlist}
\ex\gll \textbf{nu} \textbf{ke} {taleng duap}\\
3\sg{}.\pronoun{} \nom{} policeman\\
\glt `He is a a policemen (i.e. not someone else)'
\ex\gll aga \textbf{bawa} {taleng duap}\\
my brother policeman\\
\glt `My brother is a policeman.'
%\ex\gll mukolase ane ito ingam i urat biter-iki ito taun se namer-iki\\
%tomorrow I \conj{} garden in work do-\fut{}.1\sg{} \conj{} town to go-\fut{}.1\sg{}\\
%`Tomorrow I shall either work in the garden or go to town.'
\end{xlist}
\end{exe}

\begin{exe}\ex\label{WasFOCQ} {Waskia\il{Waskia}} \citep[37, 11]{Ross:1978}\nopagebreak
\begin{xlist}
\ex\gll mela, \textbf{gagi} \textbf{ke} madang urat biteso\\
\Neg{} Gagi \nom{} Madang work does\\
\glt `No it is Gagi who works in Madang. %[37]
\ex\gll \textbf{gagi} madang sule se bage-so, ayi ?\\
Gagi Madang school at stay-\prs{}.3\sg{} \question{}\\
\glt `Gagi is at school in Madang, isn't he?' %[11]
\end{xlist}
\end{exe}

Ross' discussion of Waskia\il{Waskia} is the only instance in which the emphatic subject-marker is explicitly treated as Nominative case-marking.
\footnote{\citet{Ross:1978} uses the term `subject-marker', including both transitive and intransitive subjects and thus the domain of a typical nominative case-marker.}
Other authors treating similar markers as instances of case rather than information structure markers usually analyze the marker as an ergative\is{case!individual forms!ergative} .
As noted in Section~\ref{focusNOM}, one reason for this might be the more frequent occurrence of this marker on transitive subjects than on intransitive ones. 
Also, in some languages, this marker appears on transitive subjects in different sorts of contexts, not only emphatic ones, while intransitive subjects receive this marker exclusively in emphatic contexts. 
If the marker in question serves to disambiguate argument structure as well as in contrasting functions, the absence of the marker in the first context would be expected for intransitive subjects. 
Two languages exhibiting this sort of system are Kaki Ae\il{Kaki Ae} and Yawuru.
\footnote{Following \citet{Lynch:1998}, I include languages of Australia into the group of Pacific languages.} 
The Kaki Ae\il{Kaki Ae} example (\ref{KakEmphEx2}) has already been discussed above. 
The marker \emph{-ro} marks the S argument in the first two clauses, which is newly introduced in the discourse, while other subjects remain zero-coded.
The Australian language Yawuru\il{Yawuru} has a similar structure. 
It employs the (optional) Ergative\is{case!individual forms!ergative} marker \emph{-ni} in so-called contrastive uses for encoding intransitive subjects as well such as in (\ref{YawEmphEx}a), while in other contexts S arguments cannot be marked with it (\ref{YawEmphEx}b).
 
\begin{exe} \ex\label{KakEmphEx2} {Kaki Ae\il{Kaki Ae}} \citep[Eleman; Papua New Guinea; ][52]{Clifton:1997}\nopagebreak
\gll naora\textbf{-ro} loea-ra-kape naora\textbf{-ro} u-ra-ha luera-ma \emph{aua} erahe uriri-\rdp{}% check original source
-isani naora kai w\"a'\"\i-isani-pe ko''ara oporo hu'a fua-isani koi'ara \"e'a rea-vere katlain ekakau himiri fua-isani a-isani-pe ava-isani\\
mother-\erg{} return-\irr{}-and mother-\erg{} call-\irr{}-3sS then-\loc{} child 3\pl{} run-\rdp-and mother to go\_down-and-? another wood block carry-and another that 3\sg{}-\poss{} fishing\_line something many carry-and get-and-? go\_up-and\\
\glt `The mother returns, the mother calls, and the children run down to the mother, some carry blocks of wood, some carry fishing line and many other things, they get them and go up.'
\end{exe}

\begin{exe}
\ex\label{YawEmphEx} {Yawuru}Yawuru \citep[Australian, Nyulnyulan; Western Australia; ][254]{Hosokawa:1991}\nopagebreak
\begin{xlist}
\ex\gll ngayu\textbf{-ni} nga-nga-nda mulukula-gadya, dyuyu\textbf{-ni} buru-bardu kari mi-na-bi-nda\\
1-\erg{} 1-\aux{}-\pfv{} work-\intens{} 2-\erg{} time-still grog.\abs{} 2-\transitiv{}-drink-\pfv{}\\
\glt `I was working hard while you were drinking.'
\ex\gll ngayu(*-ni) mulkula-gadya-nga-nga-rn\\
1.\abs{}(*-\erg{}) work-\intens{}-1-\aux{}-\ipfv{}\\
\glt `I'm working (hard).'
\end{xlist}
\end{exe}

Other authors, like \citet{Fabian:1998} for Nabak\il{Nabak}, mainly discuss the discourse structure functions of similar markers, while its predominant or even exclusive appearance with subject arguments is not paid much attention.
He labels the Nabak\il{Nabak} marker \emph{-\textipa{aN}} as `focus'.
The marker is mainly used to (re-)introduce participants to the discourse. 
The use of this marker is demonstrated in (\ref{NabEmphEx}a), while absence on regular subject arguments is shown in (\ref{NabEmphEx}b).
The marker is supposedly cognate to ergative\is{case!individual forms!ergative}  markers in related languages.

\begin{exe}\ex\label{NabEmphEx} {Nabak\il{Nabak}} \citep[Finisterre-Huon; Papua New Guinea; ][80, 95]{Fabian:1998}\nopagebreak
\begin{xlist}
\ex\gll\textipa{\textbf{tam-aN}} \textipa{gaki-ye}\\
dog-\foc{} die-3\sg{}.\pstrem{}\\
\glt `The dog died.'
\ex\gll \textipa{\textbf{bo}} \textipa{\textbf{ke}} \textipa{da-en} \textipa{met-ge}\\
pig \dem{} over\_there-\loc{} go-3\sg{}.\pstrem{}\\
\glt `That pig went over there.'
\end{xlist}
\end{exe}


The Eipo\il{Eipo} postposition \emph{arye} appears to have a different diachronic origin. 
This marker is also used as a semantic case, and is used to encode instrumental, allative\is{case!individual forms!allative} and related meanings. 
However, it is also used to mark the subject noun phrase especially if the subjects are used contrastively. 

\begin{exe}
\ex {Eipo\il{Eipo}} \citep[Trans-New-Guinea; Indonesia; ][169]{Heeschen:1998}\nopagebreak
\begin{xlist}
\ex\gll el ninye sik \textbf{do} \textbf{arye} a-motokwe nirye ba-lam-uk\\
he man their ancestor \sbj{} here-mountain all go-\hab{}-3\sg{}.\pstrem{}\\
\glt `Man's ancestor used to go to all mountains.'
\ex\gll ninye \textbf{na-arye} kweb-reib-se\\
man 1-\sbj{} create-put-1\sg{}.\pstrem{}\\
\glt `It was me who created man.'
\end{xlist}
\end{exe}


A summary of the data from the Pacific languages is given below (Table~\ref{OverviewEmphPac}).
Two distinct patterns are found in these languages. 
Nias\il{Nias}, Aji\"e\il{Aji\"e} and Savosavo\il{Savosavo} do not use the overt S-case-marker in emphatic contexts. 
In Savosavo\il{Savosavo}, blocking of the marker through the emphatic clitic could be analyzed as the reason for this. 
Similar to the Nilo-Saharan languages, in Nias\il{Nias} and Aji\"e\il{Aji\"e} the emphatic subject appears in a position preceding the otherwise initial verb.

The second pattern found in the Pacific is not usually included in the discussion of marked"=S languages. 
It is exclusively found with languages of this region.
The languages in question exhibit marked"=S properties only with emphatic subjects. 
In non-emphatic contexts, they either use no marking of the case relations S, A and P at all (Eipo\il{Eipo}, Nabak\il{Nabak}, Waskia\il{Waskia}) or have an ergative"=absolutive alignment with an overt ergative\is{case!individual forms!ergative} case (Kaki Ae\il{Kaki Ae}, Yawuru), this marker is extended to S contexts for emphatic subjects. 
For the languages that have neutral alignment in non-emphatic contexts (i.e. they encode S, A and P identically), the origins and properties of the overt S-case-marker found on emphatic subjects vary to some extent. 
At least two different sources have to be considered. 
In Eipo\il{Eipo} the marker is used as a Allative\is{case!individual forms!allative} case in other contexts, while for Nabak\il{Nabak} and Waskia\il{Waskia} it is considered to be cognate to an ergative\is{case!individual forms!ergative} marker in related languages.  

\begin{table}[h,t,b]
\begin{center}
\caption{Marking of emphatic S arguments in the languages of the Pacific area}\label{OverviewEmphPac}%\\
\begin{tabular}{lccc}
\hline \hline 
\bfseries language&\bfseries non-emphatic S &\bfseries emphatic S &\bfseries predicate nominal\\
\hline
Aji\"e\il{Aji\"e}&\nom{}&\acc{}&\acc{}\\
%\hdashline
Eipo\il{Eipo}&bare noun&\loc{}&bare noun\\
%\hdashline
Kaki Ae\il{Kaki Ae}&\abs{}&\erg{}&\abs{}\\
%\hdashline
Nabak\il{Nabak}&bare noun&\foc{}&bare noun\\
%\hdashline
Nias\il{Nias}&\mut{}&unmutated&unmutated\\
%\hdashline
Savosavo\il{Savosavo}&\nom{}&\acc{} + \emphat{}&\acc{}\\
%\hdashline
Waskia\il{Waskia}&bare noun&\foc{}&bare noun\\
%\hdashline
Yawuru\il{Yawuru}&\abs{}&\erg{}&\abs{}\\
\hline \hline
\end{tabular}
\end{center}
\end{table} 

%%%%%%%%%%%%%%%%%%%%%%%%%
%%%%% Section 5.9 %%%%%%%
%%%%%%%%%%%%%%%%%%%%%%%%%

\section{North America}\label{EmphNA}

The North American marked"=S languages show no remarkable patterns with respect to the marking of discourse prominent arguments.
For most languages, the discussion of discourse structure is very sparse. 
The reason for this might be that apart from intonation and possibly word-order\is{word-order}, there are no dedicated devices to mark the discourse properties of the participants, as \citet[276]{Munro:1976} notes for Mojave\il{Mojave}. 
The languages that do have information on these constructions mark emphatic S arguments in the same way as non-emphatic S.

In Wappo\il{Wappo}, special morphology is used to put emphasis on an argument. 
If the focus-marker\is{focus!overt marker} \emph{lakhuh} is attached to the S argument of a clause, the Nominative\is{case!individual forms!nominative}  case-marking remains on this argument (\ref{WapEmph}).
Similarly in Maidu\il{Maidu}, the emphasis marker \emph{-{\textglotstop}as} can follow every element of a sentence except the verb, and the emphasized element is sentence-initial, case-marking stays invariant (\ref{MaiEmph}).  

\begin{exe}\ex\label{WapEmph} {Wappo\il{Wappo}} (Wappo\il{Wappo}-Yukian; California; \citealt[79]{Thompsonetal:2006}, \citealt[92]{Lietal:1977})\nopagebreak
%\begin{xlist}\ex
\gll\textipa{ce} \textipa{\textbf{\v saw-i}} \textipa{lakhuh} \textipa{nuh-khe\textglotstop}\\
that bread-\nom{} \foc{} steal-\pass{}\\
\glt `It's the bread that got stolen.'
%\ex\gll\textipa{mayi\v s-i} \textipa{ma\v cu\textglotstop-khe\textglotstop}\\
%corn-\nom{} ash\_roast-\pass{}\\
%`The corn has been ash-roasted.'
%\end{xlist}
\end{exe}

\begin{exe} \ex\label{MaiEmph} {Maidu\il{Maidu}} \citep[Maidu\il{Maidu}an; California; ][711]{Dixon:1911}\nopagebreak
 \gll \textbf{s\"u-m} has nik do'kan\\
dog-\nom{} \emphat{} 1\sg{}.\acc{} bite\\
\glt `The dog bit me.' 
\end{exe}

For the Yuman languages, not much information on discourse structure marking is provided. 
The Mojave\il{Mojave} situation is probably prototypical for the whole language family. 
\citet[276]{Munro:1976} states that she ``has not found any evidence that Mojave\il{Mojave} has any syntactic devices for indicating topic, other than changes in stress or (possibly) word order.''

The only special discourse structure elements that are discussed for any Yuman language are afterthought\is{topic!afterthought} topics (i.e. right-dislocated arguments).
S arguments in this position bear the same nominative\is{case!individual forms!nominative}  case-marking as elsewhere in Jamul\il{Jamul Tiipay} Tiipay (\ref{JamEmph}) and Yavapai\il{Yavapai} (\ref{YavEmph}), the two languages for which the context is explicitly discussed.

\begin{exe} \ex\label{JamEmph} {Jamul\il{Jamul Tiipay} Tiipay} \citep[Yuman; California, ][334]{Miller:1990}\nopagebreak
\gll\textipa puu mesheyaay raw-ch yu \textbf{xu'maay-pe-ch}\\
that\_one be\_afraid \ipfv{}-\ssbj{} be boy-\dem{}-\nom{}\\
\glt `He was afraid of that (bull), the orphan boy (was).' 
\end{exe}

\begin{exe}\ex\label{YavEmph} {Yavapai\il{Yavapai}} \citep[Yuman; Arizona; ][139]{Kendall:1976}\nopagebreak
\begin{xlist}
\ex\gll \textipa{\textglotstop\~na} \textbf{\textipa{\textglotstop-tal-c}} \textipa{yu} \textipa{\textglotstop\~na} \textbf{\textipa{\textglotstop-cita-c}} \textipa{yu-e:-k} \textipa{ke} \textipa{qalyev-c-m} \textipa{\textglotstop-u:} \textipa{\textglotstop-om-km} \textbf{\textipa{\textglotstop\~na-c}}\\
1 1-father-\nom{} be 1 1-mother-\nom{} be-\conj{}-ego \Neg{} unhappy-\pl{}-\allo{} 1-see 1-not-inc 1-\nom{}\\
\glt `My father and mother, never unhappy do I see them, I.'

\ex\gll\textipa{\~nvat} \textipa{\textglotstop mo-\textglotstop han} \textipa{\textglotstop-tkay-c-k\~n} \textbf{\textipa{\~na-c-c}}\\
goat sheep-good 1-mix-\pl{}-\compl{} 1-\pl{}-\nom{}\\
\glt `(We) mixed the sheep and the goats.'
\end{xlist}\end{exe}

Table~\ref{OverviewEmphNA} summarizes the data for the languages of North-America.
They behave quite unremarkably concerning the marking of emphatic subjects.
In the Yuman languages, no special marking of discourse prominence in subjects is found on a segmental level and these element receive the regular nominative\is{case!individual forms!nominative}  case.
Maidu\il{Maidu} behaves in a parallel fashion, marking emphatic subjects identically to non-emphatic subjects with Nominative\is{case!individual forms!nominative}  case.
Only Wappo\il{Wappo} uses special morphology to mark focused elements. 
This marking is, however, not restricted to subjects and combines with the regular case-marking, i.e. Nominative case for subjects.
Coincidentally, the marking of emphatic subjects and predicate nominals is the same for the marked"=S languages of North America except Wappo\il{Wappo}.
However, the emphatic structures do not show any cleft-like properties otherwise (e.g. fronting of the subject).

\begin{table}[bht]
\begin{center}
\caption{Marking of emphatic S arguments in the marked"=S languages of North America}\label{OverviewEmphNA}%\\
\begin{tabular}{lccc}
\hline \hline
\bfseries language&\bfseries non-emphatic S &\bfseries emphatic S &\bfseries predicate nominal \\
\hline
Jamul\il{Jamul Tiipay} Tiipay&\textbf{(\nom{})}&\textbf{(\nom{})}&\acc{}\\
%\hdashline
Mojave\il{Mojave}&\textbf{\nom{}}&\textbf{\nom{}}&\textbf{\nom{}}\\
%\hdashline
Yavapai\il{Yavapai}&\textbf{\nom{}}&\textbf{\nom{}}&\textbf{\nom{}}\\
%\hdashline
Wappo\il{Wappo}&\textbf{\nom{}}&\textbf{\nom{}} + \foc{}&\acc{}\\
%\hdashline
Maidu\il{Maidu}&\textbf{\nom{}}&\textbf{\nom{}}&\textbf{\nom{}}\\
\hline \hline
\end{tabular}
\end{center}
\end{table}

%%%%%%%%%%%%%%%%%%%%%%%%%
%%%%% Section 5.10 %%%%%%
%%%%%%%%%%%%%%%%%%%%%%%%%

\section{Summary}\label{EmphSum}

In this section, the data on emphatic subjects in marked"=S languages are summarized. 
Table~\ref{OverviewEmph} provides an overview of the different systems of marking emphatic subjects in the languages discussed in this chapter.
First of all, for each language the table indicates how it marks emphatic and non-emphatic subject arguments. 
The table lists the case-form a noun appears in as well as any additional markers occurring on the noun in the given context.
Further on, the basic word-order (BWO)\is{word-order!basic} and the word-order in emphatic contexts\is{word-order!in emphatic contexts} (emphatic WO) is given. 
The table also summarizes the case-form a nominal predicate receives in the given language (this data is discussed in more detail in Chapter 3).
And finally, I indicate whether a language allows for zero copulas\is{copula!absence versus presence} with nominal predications.

An interesting generalization is that all languages of the sample with verb-initial word-order\is{word-order} in non-emphatic clauses front emphatic subjects (and other emphatic elements)\is{word-order!in emphatic contexts}. 
The tendency that languages with a dominant VSO-order allow for an alternative SVO-order has also been observed by \citet{Greenberg:1963} as his Universal 6. 
In addition, the overt S-case-marking found on post-verbal subjects is not found in this pre-verbal position in all these languages. 
This pattern holds for the Nilo-Saharan languages Datooga\il{Datooga}, Nandi\il{Nandi}, Tennet\il{Tennet} and Turkana\il{Turkana} as well as for the Polynesian languages Nias\il{Nias} and Aji\"e\il{Aji\"e}. 
Also, the zero-coded form of emphatic subjects is identical to the form of predicate nominals\is{nominal predication!predicate nominal} in these languages, making an analysis of these structures as a cleft-construction likely. Additional support for the cleft hypothesis comes from the fact that all these languages generally allow zero-copulas. 
Three other languages for which the cleft analysis of emphatic subjects might work out are the Eastern Cushitic languages Arbore\il{Arbore}, Boraana\il{Oromo (Boraana)} Oromo, and Harar\il{Oromo (Harar)} Oromo. 
In Arbore\il{Arbore}, the emphatic subjects are overtly coded, though not with the Nominative case. 
Instead, the Predicative case is used indicating the status of this element as predicate nominal.\is{emphatic subject|)}  

\begin{table}
\caption{Overview on the marking of emphatic S arguments}\label{OverviewEmph}%\\
\begin{sideways}
\resizebox{\textheight}{!}{
\begin{tabular}{lcccccc}
\hline \hline
%\begin{longtable}{|l|c|c|c|c|c|c|}
\bfseries language&\bfseries non-emphatic S &\bfseries emphatic S &\bfseries predicate nominal&\bfseries zero copula &\bfseries BWO &\bfseries emph. WO \\
\hline
%\endfirsthead
%\bfseries language&\bfseries emphatic S &\bfseries non-emphatic S&\bfseries Basic Word-order &\bfseries emphatic Word-order &\bfseries Predicate Nominal&\bfseries zero copula\\
%\endhead
%\hline \multicolum{3}{r}{\emph{Continued on next page}}
%\endfoot
%\hline
%\endlastfoot
Aji\"e\il{Aji\"e}&\nom{}&\acc{}&\acc{}&possible&VOS&SVO\\
%\hdashline
Arbore\il{Arbore}&\nom{}&\pred&\pred&possible&S(O)V&SV\\
%\hdashline
Datooga\il{Datooga}&\nom{}&\acc{}&\acc{}&possible&VSO&SVO\\
%\hdashline
Dinka\il{Dinka (Agar)} (Agar)&\nom{} (non-topic)&\acc{}&{-}&{-}&Topic initial (SVO)&Topic initial\\
%\hdashline
Eipo\il{Eipo}&bare noun&\loc{}&bare noun&always&SOV&SOV\\
%\hdashline
Gamo\il{Gamo}&\nom{}&\nom{}&\acc{}&possible&SVO&SVO\\
%\hdashline
Jamul\il{Jamul Tiipay} Tiipay&(\nom{})&(\nom{})&\acc{}&always&SOV&?/OVS\\
%\hdashline
K'abeena\il{K'abeena}&\nom{}&\nom{}&\acc{}&no&SOV&OSV\\
%\hdashline
Kaki Ae\il{Kaki Ae}&\abs{}&\erg{}&\abs{}&always&SOV&SOV\\
%\hdashline
Maidu\il{Maidu}&\nom{}&\nom{}&\nom{}&no(?)&SOV&SOV\\
%\hdashline
Mojave\il{Mojave}&\nom{}&\nom{}&\nom{}&possible&SOV&SOV\\
%\hdashline
Nabak\il{Nabak}&bare noun&\foc{}&bare noun&always&SOV&SOV\\
%\hdashline
Nandi\il{Nandi}&\nom{}&\acc{}/\nom{}&\acc{}&always&VSO&SVO/VOS\\
%\hdashline
Nias\il{Nias}&\abs{}&\erg{}&\erg{}&always&VOS&SVO\\
%\hdashline
Oromo (Boraana\il{Oromo (Boraana)})&\nom{}&\acc{}+(\gen{}) ?&\acc{}&possible&SOV&SV\\
%\hdashline
Oromo (Harar\il{Oromo (Harar)})&\nom{}&\foc{}&\acc{}&possible&SOV&SOV\\
%\hdashline
Savosavo\il{Savosavo}&\nom{}&\acc{} + \emphat{}&\acc{}&always&SOV&S\\
%\hdashline
Tennet\il{Tennet}&\nom{}&\acc{}/\nom{}&\acc{}&only equat.&VSO&SVO\\
%\hdashline
Turkana\il{Turkana}&\nom{}&\acc{}&\acc{}&possible&VSO&SVO\\
%\hdashline
Wappo\il{Wappo}&\nom{}&\nom{} + \foc{}&\acc{}&only future&SOV&SOV\\
%\hdashline
Waskia\il{Waskia}&bare noun&\foc{}&bare noun&most&SOV&SOV/OVS\\
%\hdashline
Yavapai\il{Yavapai}&\nom{}&\nom{}&\nom{}&no ?&SOV&SOV/OVS\\
%\hdashline
Yawuru\il{Yawuru}&\abs{}&\erg{}&\abs{}&possible&SVO/OVS&SV\\
\hline \hline
\end{tabular}
}
\end{sideways}
\end{table}




