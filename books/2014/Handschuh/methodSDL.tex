\chapter{Redefining alignment}\label{method}

\section{Introduction}
The functional\is{marked-S languages!functional definition} approach to marked"=S coding \citep{Koenig:2006,Koenig:2008} claims that the zero-case will have a wider distribution than the S-case in every language. 
However, measuring the distribution of case-forms in such a way as to be comparable across languages is not a trivial enterprise. 
A number of factors have to be taken into account. 
In this chapter, I discuss these different factors and develop a methodology for measuring these distributions.  

Traditionally, typological work on alignment systems has always considered languages in a more finely-grained manner than simply to state that language X has, for example, no\-mi\-na\-tive-accusative alignment. 
Rather than classifying languages as a whole as belonging to type A or B, a large part of the typological literature has been focused on the investigation of the alignment in specific domains of the grammar. 
Central to this take on alignment are languages with a so-called split alignment-system\is{alignment!splits} -- i.e. languages employing different alignments in different parts of their grammar. 
This study follows the spirit of such approaches, which are discussed in more detail in Section~\ref{grammar-based}. 

The methodology of this study is outlined in the subsequent sections. 
Several meta-linguistic contexts are defined for which the case-realization in the individual marked"=S languages %through specific constructions (and utimatively the case-forms employed in said constructions) 
is investigated (Section~\ref{methods}). 
These contexts constitute possible split-up points for alignment, such that a language will use one case-form to encode an S-like argument in one context but a different case-form in the next context.
The contexts that turned out to be most interesting are discussed in depth in separate chapters. 
These contexts are introduced in Section ~\ref{contexts}. Further, there are various types of splits found only in a small number of languages in my sample -- in most cases only in a single language. These idiosyncratic splits will be presented in Section~\ref{splits}.

Finally, I will look at the usage-based factors that influence the distribution of zero-case and S-case (Section~\ref{usage-based}). 
These factors have not been dealt with in previous studies on marked"=S alignment.  
However, without taking the actual overall usage of the case-form into account -- as measured, for example,  through textual frequencies -- any claim about the distribution of S-case and zero-case can only be of preliminary nature. 
The lack of well-designed corpora for marked"=S languages prevents me from doing a quantitative analysis of usage-based factors. 
Therefore, I will discuss the influence of such factors only from a qualitative point of view. 

%%%%%%%%%%%%%%%%%%%%%%%%%%%%%%%%%%%%%%%%
%%%%%%%%%%    SECTION 2.1   %%%%%%%%%%%%
%%%%%%%%%%%%%%%%%%%%%%%%%%%%%%%%%%%%%%%%

\section{Split alignment}\label{grammar-based}\is{alignment!splits|(}

The standard method for identifying alignment systems is to compare the morphosyntactic treatment of the S argument of intransitive verbs with the A and P arguments of transitive verbs. 
In the previous chapter, it has been discussed how on this basis the basic types of nominative"=accusative and ergative"=absolutive alignment are distinguished (Section~\ref{coding}). However, it has long been noted that the classification of a languages as a whole as being ergative"=absolutive or nominative"=accusative is problematic.   

\begin{quote}
\dots it is rather misleading to speak of ergative languages, as opposed to no\-mi\-na\-tive-ac\-cu\-sa\-tive languages, since we have seen that it is possible for one phenomenon in a language to be controlled on an ergative"=absolutive basis while another phenomenon in the same language is controlled on a no\-mi\-na\-tive-accusative basis. Thus one should ask rather to what extent a language is ergative"=absolutive or nominative"=accusative, or, more specifically, which constructions in a particular language operate on the one basis and which on the other.
\citep[350--351]{Comrie:1978}
\end{quote}

The prototypical kind of languages not having a uniform alignment throughout are languages exhibiting so-called `split ergativity'. 
As the terminology suggests, this phenomenon has most prominently been studied for languages of a basically ergative type \citep[cf.][]{Silverstein:1976, Dixon:1994}. 
As a more general term, I will use the term \textsc{split alignment system} for all cases in which two different alignment systems are employed in two domains of a grammar. 
Differences in coding will also be subsumed in this analysis of splits, since, strictly speaking, marked"=nominative and marked"=absolutive systems are coding variants of the basic nominative"=accusative and ergative"=absolutive alignment types.

Several observations have been made about split ergative languages. 
First, splits in the alignment system seem to occur in a limited set of grammatical domains \citep{Dixon:1994}. Most frequent are splits along the line of some kind of a nominal prominence hierarchy and splits based on temporal or aspectual information %used to be: type
of the clause. 
Second
\is{Silverstein Hierarchy|(}, the ergative pattern is in most cases found on the same side of those splits, namely in the more salient part of the nominal hierarchy \citep{Silverstein:1976} and in the past tense/perfective aspect \citep{Malchukov.tam}. 
The view that one side of a hierarchy is uniformly associated with the same type of alignment system across languages has recently been challenged \citep{Bickel.hierarchy}. %it is still widely regarded as textbook knowledge in linguistics.
Third, it has been observed that hardly any overlap is found between those two splits, i.e. when a language has a split along the nominal hierarchy, it will not have a tense/aspect split, and vice versa \citep{Trask:1979}. 

Examples of the NP split ergative system can be found in many Australian languages. 
In Dyirbal \citep{Dixon:1972} Accusative\is{case!individual forms!accusative} case (which is overtly coded)  is distinguished from (zero-coded) Nominative for first and second person pronouns. 
All other nominals (i.e. third person pronouns, proper names and common nouns) show a distinction between (overtly coded) Ergative\is{case!individual forms!ergative} forms and (zero-coded) Absolutives\is{case!individual forms!absolutive}.\is{Silverstein Hierarchy|)} 
Splits based on tense, aspect, or modality of the clause are attested in the Indo-Aryan languages and in Georgian (see \citet{Malchukov.tam} for a discussion of splits of this type). 
In Section~\ref{splits}, I will discuss the extent to which the split systems in marked"=S languages behave in the same way as classical split-ergative languages.\is{alignment!splits|)}

\section{Micro-alignment}\label{methods}\is{micro-alignment approach|(}

The goal of this study is to provide an in-depth view of the marked"=S system that goes beyond the encoding of the primitive  S, A, and P. 
Instead, I will survey a variety of grammatical domains with respect to which case-forms are employed. 
Ultimately, this study aims to test the claim that in marked"=S languages the zero-coded form of a noun has a wider range of functions than the overtly coded S-case \citep{Koenig:2006}. 
So, instead of determining alignment by considering three different possible occurrences of case-marking (the well-known S/A/P trinity of alignment), I will look at fourteen possible different contexts in which case-marking can occur.
The focus is on areas of grammar that are coded by the S-case in typical languages of the standard nominative"=accusative or ergative"=absolutive kind.   
The contexts define the possible split-up points for alignment.
The larger the number of contexts one considers, the more room there is for cross-linguistic variation.
Compared with broad classifications of languages as nominative"=accusative or ergative"=absolutive, this approach studies alignment on a microscopic level.
Hence, I call this take on alignment \textsc{micro-alignment}. 
Various current typological approaches are attending to ever finer-grained distinctions between languages.
For example, \citet{Bickel:2007} proposes a multivariate approach to language typology that aims at more precisely quantifying how different individual languages are. 
This approach is employed for the domain of grammatical relations by \citet{Witzlack:2010}.
A fine-grained classification of the linguistic structures under investigation is vital for such quantitative work.

%The method chosen by the proponents of the functional approach to marked"=S systems is to look at the list of linguistic contexts the two case-forms occur in. Moreover this is also the way the phenomenon is dealt with in grammatical descriptions of the languages in question. For practical reasons the main focus of this study will be on these issues as well, however, there is a range of other factors which should be taken into account when testing the claim of the wider range of uses of the zero-case in marked"=S languages. But even when limiting the scope of ones survey to the discription of a language at outlined in a descriptive grammar, 

When\is{typology!tertium comparationis|(} doing language typology, one has to consider the important issue how the data from different languages can be compared at all.  
The need for a \textsc{tertium comparationis} has been highlighted in many works.
\citet[28--19]{Seiler:2000} criticizes the practice of choosing an individual language as the unit of measure to compare other languages against.  
Furthermore, he agrees with earlier scholars, such as \citet{Heger:1990}, that the \textit{tertium comparationis} should also not be taken from beyond the domain of language activity.
\citet[185]{Wierzbicka:1995} argues that meaning is the only possible source a \textit{tertium comparationis} can be derived from since linguistic form and structure differ among individual languages.
\citet[13--14]{Croft:2003} discusses the common practice in typological research of choosing a semantic definition of the domain to be investigated (at least for typological research that is concerned with morphosyntactic structures). 
He notes that the traditional notion of semantics is too narrow to subsume all relevant aspects and thus includes pragmatic structures into the domain from which means of comparison can be drawn.
A similar stance is taken by \citet{Haspelmath.comp:2010}, who notes that comparison across languages should not be done on the basis of grammatical categories, since these are of language-specific nature. 
He suggests that one should rather resort to `comparative concepts', which are not language-specific but specifically defined as a cross-linguistic means of comparison.\is{typology!tertium comparationis|)} 

The methodology underlying this study is visualized in Figure~\ref{MethodGraph}. 
The\is{language-specific level|(} distinction between language-specific categories and comparative concepts is captured on the horizontal axis -- the left side being dedicated to language-specific categories, while metalinguistic comparative concepts are to be found on the right side of the figure. 
The term \textit{surface} on top of the left-hand side of the figure is not to be understood in opposition to any deep structure level. 
Case-forms or constructions that one wants to postulate for a given language must be identifiable on the surface level.\is{language-specific level|)} 
However\is{language-independent level|(}, whether they are directly mapped from the language-independent \textit{conceptual level} or from a language-specific deep structure representation (which in turn receives its information from the conceptual level) is not relevant here. 
For this approach I assume at least these two levels, the conceptual level, which is employed to make comparisons across languages possible, and the surface level, which reflects the observable data from a language. 
An additional language-specific level that comprises an underlying representation could be integrated if need be\is{language-independent level|)}. 

\begin{figure}[h,t,b] \centering \fbox{
\begin{picture}(265,180)

\put(25,85){\makebox(70,20){\textbf{Case}}}
\put(25,35){\makebox(70,20){\textbf{Construction}}}

\put(165,85){\makebox(70,20){\textbf{Role}}}
\put(165,35){\makebox(70,20){\textbf{Context}}}

\closecurve(15,70, 20,20, 60,15, 100,20, 105,70, 100,120, 60,125, 20,120)
\closecurve(155,70, 160,20, 200,15, 240,20, 245,70, 240,120, 200,125, 160,120)

\multiput(117,40)(0,50){2}{\line(1,0){26}}
\multiput(117,50)(0,50){2}{\line(1,0){26}}

\multiput(117,40)(0,50){2}{\line(0,-1){5}}
\multiput(117,50)(0,50){2}{\line(0,1){5}}

\multiput(107,45)(0,50){2}{\line(1,1){10}}
\multiput(107,45)(0,50){2}{\line(1,-1){10}}

\multiput(143,40)(0,50){2}{\line(0,-1){5}}
\multiput(143,50)(0,50){2}{\line(0,1){5}}

\multiput(153,45)(0,50){2}{\line(-1,1){10}}
\multiput(153,45)(0,50){2}{\line(-1,-1){10}}

%\put(115,60){\makebox(30,20){maps on}}

\multiput(50,60)(140,0){2}{\makebox(20,20){$\in $}}

\put(35,145){\makebox(50,20)[t]{surface level}}
\put(35,125){\makebox(50,20)[t]{(language-specific)}}

\put(175,145){\makebox(50,20)[t]{conceptual level}}
\put(175,125){\makebox(50,20)[t]{(language-independent)}}

\end{picture}}

\caption{Cross-linguistic comparison of case-forms based on the contexts of use}\label{MethodGraph} \end{figure}

Another aspect depicted in the figure is the level of granularity of the elements considered; granularity increases from top to bottom. 
The left side of the graph consists of two elements: \textsc{case} and \textsc{construction}. 
Case\is{case} refers to a specific case in a given language, e.g. the German Dative, the Latin Ablative or the Finnish Partitive. 
A\is{construction|(} construction in the sense used here is a linguistic entity roughly corresponding to a clause. 
The notion of construction is defined more narrowly than in Construction Grammar \citep[18]{Goldberg:2006} where constructions go ``all the way down'' and up. 
While in Construction Grammar individual case-forms are considered to be constructions as well, in my definition, constructions are larger entities. 
A typical construction in my view is a predication and thus takes at least one argument. 
Note, however, that not all contexts I discuss in this work easily fall under this definition, namely, the extra-syntactic forms of citation and address.
A use of the notion construction similar to mine can be found in descriptive grammars, where labels such as `{copula construction}', `{existential construction}' or `{locational construction}' are used for language-specific ways of expressing a certain meaning.
For example, the most commonly used existential construction in English\il{English} is the `there is an X' construction.
Nominals marked in a given case-form constitute a part of a construction, i.e. cases are elements of constructions\is{construction|)}. 

On the right side of Figure~\ref{MethodGraph} the topmost concept are \textsc{roles}, which are elements of the larger meta-linguistic \textsc{contexts}. 
The notion of role is quite familiar from works such as \citeauthor{Fillmore:1969}'s \citeyear{Fillmore:1969} `case roles' or the `se\-man\-tic/\-the\-ma\-tic/\-$\theta$-' roles of other schools of linguistics. 
No consensual term has yet been established for what I have labeled context here. % Used by Cysouw and Wälchli 
Labels such as `meaning', `function' or `sense' could be employed as well.
This level of representation is meant to represent some larger chunk of meaning that can have varying levels of abstractness. 
Semantic forms along the lines of \citet{Dowty:1991} are a way to envisage this level of representation, as in: $\exists e [kissing(e)$ \& $Agent-of(John,e)$ \& $Patient-of(Mary,e)]$. 
% check Dowty representation
However, these can be paired with additional contextual information like the discourse properties of a given role within the specific contextual occurrence (like: ``mono-transitive verb whose A  is a contrastive topic'').      

The double-headed arrows connecting the language-independent conceptual level with the language-specific surface level represent the relation between the two sides in a given language. The correspondences between the two sides can be manifold (one-to-one, many-to-one, one-to-many or many-to-many). 
A given context might be expressed by only a single construction in a language, but for another context (or in another language) there can be multiple constructions encoding the same meaning. 
Conversely, a language may use a construction to encode a whole array of contexts or it might have a construction that is exclusively used for encoding a specific context. 
\footnote{During the final stage of production, Eitan Grossman pointed out to me that a very similar idea is presented in \citet{Frajzyngier:2003}. 
I was not aware of this work previously and the production schedule does not permit me to review this work here in any detail. }% Eitan: Explaining Language Structure through Systems Interaction,  By Zygmunt Frajzyngier, Erin Shay 2003, John Benjamins, Amsterdam

\begin{figure}[h] \centering \fbox{
\begin{picture}(330,280)

\put(15,257){\makebox(85,10)[tr]{English}}
\put(15,245){\makebox(85,10)[rt]{constructions}}

\put(15,210){\makebox(85,10)[tr]{As for \textbf{me}, \dots}}
\put(15,180){\makebox(85,10)[tr]{It's \textbf{me}}}
\put(15,150){\makebox(85,10)[tr]{\textbf{I} am old}}
\put(15,120){\makebox(85,10)[tr]{\textbf{I} am here}}
\put(15,90){\makebox(85,10)[tr]{\textbf{I} am cold}}
\put(15,30){\makebox(85,10)[tr]{\textbf{I} see her/}}
\put(15,15){\makebox(85,10)[tr]{She sees \textbf{me}}}

\put(105,165){\line(-1,0){60}}
\put(105,165){\line(0,-1){80}}

\put(45,85){\line(1,0){60}}
\put(45,85){\line(0,1){80}}

\put(140,215){\line(-1,0){30}} %A
\put(140,185){\line(-1,0){30}} %B
\put(140,155){\line(-1,-1){30}} %C
\put(140,125){\line(-1,0){30}} %D
\put(140,95){\line(-1,1){30}} %E
\put(140,28){\line(-1,0){30}} %F


\put(130,251){\makebox(50,10)[tl]{\textbf{Context}}}

\put(145,210){\makebox(50,10)[tl]{\textbf{A}}}
\put(145,180){\makebox(50,10)[tl]{\textbf{B}}}
\put(145,150){\makebox(50,10)[tl]{\textbf{C}}}
\put(145,120){\makebox(50,10)[tl]{\textbf{D}}}
\put(145,90){\makebox(50,10)[tl]{\textbf{E}}}
\put(145,23){\makebox(50,10)[tl]{\textbf{F}}}


\put(157,215){\line(1,0){30}}
\put(157,185){\line(1,-1){30}}
\put(157,155){\line(1,0){30}}
\put(157,125){\line(1,1){30}}
\put(157,95){\line(1,0){30}}
\put(157,95){\line(1,-1){30}}
\put(157,28){\line(1,0){30}}


\put(200,257){\makebox(50,10)[tl]{German}}
\put(200,245){\makebox(50,10)[lt]{constructions}}

\put(200,210){\makebox(50,10)[tl]{Was \textbf{mich} betrifft, \dots}}
\put(200,180){\makebox(50,10)[tl]{\textbf{Ich} bin's}}
\put(200,150){\makebox(50,10)[tl]{\textbf{Ich} bin alt}}
\put(200,120){\makebox(50,10)[tl]{\textbf{Ich} bin hier}}
\put(200,90){\makebox(50,10)[tl]{\textbf{Mir} ist kalt}}
\put(200,60){\makebox(50,10)[tl]{\textbf{Mich} friert}}
\put(200,30){\makebox(50,10)[tl]{\textbf{Ich} sehe sie/}}
\put(200,15){\makebox(50,10)[tl]{Sie sieht \textbf{mich}}}

\put(195,115){\line(0,1){80}}
\put(195,115){\line(1,0){70}}
\put(265,195){\line(0,-1){80}}
\put(265,195){\line(-1,0){70}}

\end{picture}}
\caption{Mapping between contexts and constructions in English\il{English} and German\il{German}}\label{contextvsconstruction}
\end{figure}

In Figure~\ref{contextvsconstruction} the mapping between a number of contexts to individual constructions in English\il{English} and German\il{German} is illustrated. English\il{English} has a specific construction for contexts A, B, and F while C, D, and E are all encoded by the same construction. 
German\il{German} on the other hand uses one construction for contexts B, C, and D, while individual constructions are employed for A, E and F. For E even two different constructions are used.

English\il{English} and German\il{German} already differ notably with respect to the mapping of contexts and constructions. 
The difference between the languages is also apparent when considering the case-forms employed in the constructions. 
English\il{English} uses Accusative\is{case!individual forms!accusative} case for the only role in context A and B as well as for the patient role in context F.
All other roles are in the Nominative case. 
In German\il{German}, the Accusative\is{case!individual forms!accusative} is used for the only role in context A and in one of the constructions to express context E. 
In the other construction expressing E, this role is encoded by Dative\is{case!individual forms!dative} case.
The patient role in context F is encoded in the Accusative\is{case!individual forms!accusative} again, and every other role in the contexts listed is in the Nominative\is{micro-alignment approach|)}.  

\section{Contexts of investigation}\label{contexts}

For this study, I have selected a number of contexts which contain roles typically encoded by the unmarked nominative/absolutive case in non-marked"=S languages.
Of course there is variation in the encoding of different roles between the languages of the standard nominative"=accusative as well. 
However, in a small test sample of non-marked"=S languages all the roles studied here have indeed been encoded with the nominative case in the majority of the languages.
\footnote{The test sample has not been of any representative size, however, the languages have been chosen in order to represent different genera. 
The following languages have been included: German\il{German}, Finnish\il{Finnish}, Turkish\il{Turkish}, Japanese\il{Japanese}, Maori\il{Maori} and Kanuri\il{Kanuri}.} 
For each context one or more constructions will be discussed that are used to express the context in each language. 
Special attention is given to the case-marking employed in these constructions.
The full discussion of all data will be presented in the following five chapters. 
Here I will introduce the contexts which have been selected for this survey. 
A more detailed discussion is provided in the chapters dedicated to a certain context. 

For each language data on the encoding of prototypical transitive and intransitive clauses have been collected (i.e. the traditional marking of S, A and P). 
In addition, the following contexts have been investigated: 

\begin{itemize}
\item subject of nominal predication
\item predicate nominal\is{nominal predication!predicate nominal}
\item subject of positive existential predication\is{existential predication}
\item subject of negative existential predication\is{locational predication}
\item subject of locational predication
\item emphatic subject\is{emphatic subject}\footnote{The term \textit{emphatic}, which is used in the same way in many grammars, refers to an argument that receives a certain amount of highlighting in the given context. Typical examples of this are focused arguments and contrastive topics\is{topic}.}
\item subject of dependent\is{clause-type!dependent} clauses (more precisely: relative\is{clause-type!relative clause}, adverbial\is{clause-type!adverbial clause} and complement\is{clause-type!complement clause} clauses)
%\item subject of valency-increasing operations introducing a proto-Agent argument %maybe remove
\item subject of valency-decreasing operations\is{valency-decreasing construction} 
\item attributive possessor\is{possession!attributive}
\item noun in citation form\is{citation form}
\item noun used for addressing\is{terms of address} someone
\end{itemize}

The term \textit{subject}\is{grammatical relations!subject} used here is short hand for the argument that would be encoded with a nominative case in an ordinary nominative"=accusative language. 
For each context a more specific definition is provided in the chapter it is treated in. 
Subjects of nominal predication and nominal predicates are discussed in Chapter~\ref{nompred}. 
Chapter~\ref{existpred} deals with subjects of positive and negative existential and locational predications. 
Subjects marked as having a specific role in the discourse (referred to as \textsc{emphatic subjects}) are dealt with in Chapter~\ref{emphaticS}.  
Subjects of %valency-increasing (causatives), 
valency-decreasing (passives, antipassives) and subordinate clauses are subsumed under non-basic clause-types, discussed in Chapter~\ref{nonbasic}. 
And finally a number of contexts that are below clause level (attributive possessors) or extra-syntactic altogether (citation, address) are investigated in Chapter~\ref{extrasyn}.

Unfortunately, the information on the encoding of all contexts is not available for every language of the sample. 
I have included data as far as it is available, with the result that some languages will only be discussed for a small number of contexts. 
Those languages described in less detail are not very useful for the typological and statistical analysis. 
However, since the data are still informative for the more descriptive discussion, I have nonetheless included them. 
An in-depth statistical analysis will be presented in Chapter~\ref{typology} for a smaller sample of languages for which I have sufficient data.


%%%%%%%%%%%%%%%%%%%%%%%

\section{Incidental splits}\label{splits}\is{alignment!splits|(}

\subsection{Contexts versus other splits}

The contexts listed in the previous section are set up in order to investigate different types of alignment (and also different coding systems) that may exist in a language. 
As noted already, not all splits encountered in marked"=S languages -- let alone the world's languages -- are mirrored in this set of contexts I have picked for more detailed investigation. 
This section discusses the residual splits that will not be taken into account in the chapters to follow.

While it would in principle be possible to define the contexts listed in Section~\ref{contexts} above in such a way that all splits are covered by them, this would radically increase the number of contexts surveyed. 
Therefore I decided to take only those splits into account that regularly show up in the marked"=S languages of my sample. 
It is of course a matter of debate how to assess whether something shows up regularly. 
I have made this decision on a somewhat impressionistic basis rather than taking any hard arithmetic criterion, because of the limited amount of information that could be gathered from grammars for some contexts. 
Otherwise, a number of splits  would have had to be discarded even though they show up quite frequently in the small number of languages.
Another factor leading to the inclusion of a context is the applicability of a domain across the sample (e.g. excluding splits between genders, which are not applicable for languages without a grammatical category of gender).
%And of course only those domains of grammar are taken into account in which marked"=S alignment exists for the later survey of marked"=S languages.

In the following three subsections, I will discuss all the types of split marking found in marked"=S languages that are glossed over in the remainder of this study. 
These are splits based on the semantics of the case-marked noun phrase (Section~\ref{splitsemnoun}), the semantics of the verb (Section~\ref{splitsemverb}), and splits conditioned by morphophonological properties of the noun or noun phrase (Section~\ref{morphophon}).

\subsection{Splits based on the semantics of the noun phrase}\label{splitsemnoun}\is{Silverstein Hierarchy|(}

The classical examples of split ergative languages discussed by \citet{Silverstein:1976} are splits between pronoun and full NPs or within the different persons of the pronominal system. 
This type of split is also found in languages of the marked"=S type, for example in Oirata\il{Oirata}.
This language marks first and second person pronouns functioning as S or A arguments of the verb by the Nominative suffix \emph{-te} (\ref{Oir12}).
Third person referents (whether expressed by demonstratives -- proper 3rd person pronouns do not exist -- or full NPs) receive no case-marking in either S, A or P function (\ref{OirFullN}). 
Thus there is a split marked"=S system with marked"=S coding (i.e. a subtype of nominative"=accusative alignment) for first and second person and neutral alignment for all elements lower on the referential hierarchy.


\begin{exe}\ex\label{Oir12}\langinfo{Oirata}{Timor-Alor-Pantar; Maluku, Timor}{\citealt[66]{Donohue.Brown:1999}}\nopagebreak[4]
\begin{xlist}\ex\gll in\textbf{-te} ee asi\\
1\pl{}.\excl{}-\nom{} 2\sg{}.\polite{} see\\
\glt `We saw you.'

\ex\gll ee\textbf{-te} in asi-ho\\
2\sg.\polite{}-\nom{} 1\pl.\excl{} see-\Neg{}\\
\glt `You didn't see us.'

\ex\gll an\textbf{-te} ete na'a ippa\\
1\sg{}-\nom{} tree \obl{} fall\\
\glt `I fell out of the tree.'

\end{xlist}\end{exe}

\begin{exe}\ex\label{OirFullN}\langinfo{Oirata}{}{\citealp[67]{Donohue.Brown:1999}}
\begin{xlist}
\ex\gll \textbf{maaro} mede-n kopete-he\\
person eat-\relativ{} black-\Neg{}\\
\glt `The person who is eating isn't black.'

\ex\gll \textbf{ihar} ani asi-le mara\\
dog 1\sg{}.\acc{} see-\ssbj{} go\\
\glt `The dog$_i$ saw me$_j$ and $\emptyset _i$ left.'
\end{xlist}\end{exe}

Other marked"=S splits on the nominal hierarchy can be found in K'abeena\il{K'abeena} of the Cushitic language family \citep{Crass:2005} and the Nilotic language Datooga\il{Datooga} \citep{Kiessling:2007}. 
A more detailed discussion of marked"=S splits along the nominal hierarchy and their implications for the theoretical analysis of split marking in general can be found in \citet{Handschuh:2008, Handschuh:2014}.\is{Silverstein Hierarchy|)} 

Another domain of nominal semantics that has been noted to affect the alignment system is in the gender system of a language (or of nominal inflection classes in general). 
The Neuter nouns of several Indo-European language (like German\il{German} and Russian\il{Russian}) are known for notoriously conflating their Nominative\is{case!individual forms!nominative} and Accusative\is{case!individual forms!accusative} case-forms. 
Looking at these systems from a split alignment perspective one could describe them as having nominative"=accusative alignment for some nouns (e.g. those with masculine gender) and neutral alignment for another class of nouns (neuter nouns).  
Gender-based splits in marked"=S languages which look similar to the Indo-European situation can be found in some Cushitic languages \citep{Sasse:1984}. 
This is for example the case in Qafar\il{Qafar} \citep{Hayward:1998} where only masculine nouns have marked forms for S+A function versus zero-coded P function. 
All other genders do not distinguish these two cases.

Another -- more complex -- instance of gender-based splits is exhibited by the Australian language Mangarayi\il{Mangarayi} \citep{Merlan:1989}. 
In each of the three genders (Masculine, Feminine and Neuter), a different alignment or coding system is employed. 
While Feminine nouns use a standard type of nominative"=accusative system (\ref{MangF}), Masculine nouns are of the marked"=nominative type (\ref{MangM}) and Neuter nouns are ergative"=absolutive (\ref{MangN}).
\footnote{Of particular historical interest is the fact that the Ergative\is{case!individual forms!ergative} marker of Neuter nouns is the same form as the Nominative found with Masculine nouns.
 There seems to be a clear diachronic relation between these two alignment systems. 
However, the exact historical scenario (either the Ergative extended its domain to Masculine nouns, or the Nominative ceased to be used for S arguments that were Neuters) has not been established so far. 
Also, the tendency for Neuter nouns in some contexts to receive overt case-marking even as S arguments can be either analysed as an innovation or as a remainder of an older system.}   

 \begin{exe} \ex\label{MangF}\langinfo{Mangarayi}{Australian; Northern-Territory}{\citealp[59, 61, 64]{Merlan:1989}}
 \begin{xlist} \ex\label{MangFS} \gll \textipa{\textbf{Na\:la-gadugu}} $\emptyset$\textipa{-ya-\textbardotlessj}\\
 \nom{}.\fem{}-woman 3\sg{}-go-\pstpunc{}\\
\glt `The woman went.'
  \ex\label{MangFA} \gll \textipa{buy\textglotstop} \textipa{\textltailn an-wu-na} \textipa{\textbf{Na\:la-bugbug}} \textipa{\textbf{Na\:la-}}X?\\
 show 3\sg{}$>$2\sg{}-\aux{}-\pstpunc{} \nom{}.\fem{}-old\_woman \nom{}.\fem{}-X\\
\glt `Did old woman X (name deleted) show you?'
  \ex\label{MangFP} \gll \textipa{\textbf{Nan-gu\:dugu}} \textipa{buy\textglotstop} \textipa{wu\:la-wu-na} \textipa{Nani}\\
 \acc{}.\fem{}-woman show 3\pl{}$>$3\sg{}-\aux{}-\pstpunc{} language(\neu{})\\
\glt `They taught the woman language.'
 \end{xlist}
 \end{exe}

%\pagebreak

\begin{exe} \ex\label{MangM}\langinfo{Mangarayi}{}{\citealp[59, 61, 63]{Merlan:1989}}
 \begin{xlist} \ex\label{MangMS}  \gll \textipa{\textbf{\:na-malam}} $\emptyset$\textipa{-gala+wu-yi-ni} \textipa{\:na-landi-yan}\\
 \nom{}.\mas{}-man 3\sg{}-hang-\Mid{}-\pstcont{} \loc{}.\neu{}-tree-\loc{}\\
\glt `The man was hanging in the tree.'
 
 \ex\label{MangMA} \gll \textipa{\textbf{\:na-muyg}} \textipa{Nan-da\:lag}\\
 \nom{}.\mas{}-dog 3\sg{}$>$1\sg{}-bite.\pstpunc{}\\
\glt `The dog bit me.' 
 
 \ex\label{MangMP} \gll \textipa{\textbf{malam}} \textipa{Na-\:da\:ra+wu-b}\\
 man(\mas{}) 1\sg{}$>$3\sg{}-find-\pstpunc{}\\
\glt `I found the man.'
 \end{xlist}
 \end{exe}
  
\begin{exe} \ex\label{MangN}\langinfo{Mangarayi}{}{\citealp[59, 61]{Merlan:1989}}
\begin{xlist} 
\ex\raggedright\label{MangNS} \gll \textipa{wumbawa} \textipa{\textbf{\:landi}} \textipa{\textbardotlessj ir} $\emptyset$\textipa{-\textbardotlessj aygi-ni} \textipa{wuburgba} \textipa{\:na-budal-an}\\
one tree(\neu{}) stand 3\sg{}-\aux{}-\pstcont{} halfway \loc{}.\neu{}-billabong-\loc{}\\
\glt `One tree is standing in the middle of the billabong.'

\ex\label{MangNAP} \gll \textipa{\textbf{\:na-gunbur}} \textipa{Nan-gawa-\textbardotlessj} \textipa{\textbf{\textbardotlessj ib}-Nanju}\\
\erg{}.\neu{}-dust 3\sg{}$>$1\sg{}-bury-\pstpunc{}  eye(\neu{})-mine\\
\glt `Dust buried (i.e. blew into) my eye.'
\end{xlist}
\end{exe}

\subsection{Splits based on the semantics of the verb}\label{splitsemverb}\is{verb class!splits base on|(}

Another domain in which languages might have multiple alignments in different categories of their grammar is verbal semantics. 
Splits in the domain of intransitive subjects which are based on the semantics of the verb are often viewed as a form of split ergativity \citep[70--83]{Dixon:1994}.
This phenomenon is also known under the name of stative-active, split-S/fluid-S or semantic alignment.
More generally, it has been recognized that most languages have different alignment patterns that are found with specific classes of verbs or even individual verbs -- be they intransitive, monotransitive or ditransitive. 
In her survey of the world-wide distribution of stative-active languages, \citet{Nichols:2008}, for example, uses a quantitative approach to identifying alignment systems. 
Only if a certain proportion of verbs in a language uses the same alignment pattern does she refer to this language as being of that alignment type.

In Nias\il{Nias}, an Austronesian language of Indonesia, some types of verbs show specific alignment patterns. 
Mental state verbs take both of their arguments -- the one in experiencer role as well as the one in stimulus role -- in the Mutated (i.e. absolutive\is{case!individual forms!absolutive}) form of a noun (\ref{NiasMent}). 
Change of state verbs exhibit another special case frame: the participant undergoing the change is in the Mutated form while the target of change is in the Unmutated (i.e. zero-coded) form of a noun (\ref{NiasChange}). 
Since most other marked"=S languages do not exhibit any differences in alignment between different semantic classes of verbs, this domain -- although a very interesting one -- will not be treated further in this study. 
For Nias\il{Nias}, verbs of the types exemplified in (\ref{NiasBasic}) are used for the further typological comparison.


\begin{exe}
\ex \langinfo{Nias}{Sundic, Western Malayo Polynesian, Austronesian; Sumatra, Indonesia}{\citealp[345,591]{Brown:2001}}
\begin{xlist}
\ex\label{NiasMent}\gll a-ta'u \textbf{mba'e} \textbf{n-}ono matua\\
\stat{}-fear monkey.\mut{} \mut{}-child male\\
\glt `The monkey is afraid of the boy.'

\ex\label{NiasChange}\gll  tobali \textbf{n-}idan\"o \textbf{es}\\
become \mut{}-water ice\\
\glt `The water changed into ice.'
\end{xlist}
\end{exe}

\begin{exe}\ex\label{NiasBasic}\langinfo{Nias}{}{\citealp[343, 208]{Brown:2001}}
\begin{xlist}\ex\gll aukhu \textbf{n-}idan\"o\\
\stat{}.hot \mut{}-water\\
\glt `The water is hot.'
\ex\gll asese la-tandraig\"o va-nan\"o \textbf{go\ss i} \textbf{Balanda}  ba Dan\"o Niha\\
often 3\pl{}.\rls{}-try \nmlz{}.\mut{}-plant tuber.\mut{} Dutch \loc{} land.\mut{} person.\mut{}\\
\glt `The Dutch have often tried to plant potatoes in Nias\il{Nias}.'
\end{xlist}
\end{exe}

Another factor which falls under the heading  of verbal semantics involves splits based on the tense/mood/aspect properties of the clause. 
Though these splits are commonly observed for ergative"=absolutive languages, no straightforward example of a TAM-based split has been found for marked"=S languages. 
\citet{Urban:1985} proposes an analysis of Shokleng (G\^e) that suggests marked"=nominative coding for stative aspect and ergative"=absolutive alignment for dynamic aspect, but his data are rather controversial. 
In particular, the question whether the elements discussed by him should be considered case-marking at all -- or rather as some kind of resumptive pronouns -- remains to be answered conclusively.\is{verb class!splits base on|)}  

\subsection{Splits based on morphophonological factors}\label{morphophon}\is{case-marking!absence of!morphophonologically conditioned|(}

A final factor that can lead to the absence of case-marking in a predictable context is morphophonology -- though this is usually not viewed as a form of split case-marking. 
If the segment(s) that a case-marker consists of are deleted in a certain phonological environment, any host (noun or other case-marked element) meeting these requirements will be lacking this case-marking even in a context where it usually would be assigned this case by the construction it is used in. 
This situation holds in Cocopa\il{Cocopa} where according to \citet[104]{Crawford:1966} the subject-marker \emph{-c} ``is not usually attached to a noun ending in more than one consonant or in /\textsubdot{t}/.'' 
Also\is{case-marking!via nominal mutation|(}  in Nias\il{Nias} the process of nominal mutation is not visible on all nouns. 
Nominal mutation is straightforward with vowel initial nouns and those beginning in a voiceless obstruent. 
Other segments do not (or do not always) undergo this process \citep[69]{Brown:2001}. 
Considering that voiceless consonants become voiced through the process of nominal mutation, the most likely explanation for such a `split' is a morphophonological one, namely that those segments cannot receive any more voicing and thus do not undergo any visible transformation between the Unmutated and Mutated form\is{case-marking!via nominal mutation|)}. 
I will ignore such `apparent' splits in the remainder of this study.\is{case-marking!absence of!morphophonologically conditioned|)}\is{alignment!splits|)} 

 
%%%%%%%%%%%%%%%%%%%%%%%%%%%%%%%%%%%%%%%%
%%%%%%%%%%    SECTION 2.6   %%%%%%%%%%%%
%%%%%%%%%%%%%%%%%%%%%%%%%%%%%%%%%%%%%%%%

\section{Usage-based factors}\label{usage-based}

\subsection{Frequency}\is{frequency|(}

After investigating the ways in which the grammar of a language can influence the overall use of the different case-forms, I will now turn to language usage. 
In studies of marked"=S languages, the main focus has been on grammar-based factors, and the present study is no exception.
However, usage-based factors can strongly influence the distribution of the case-forms in actual language data.

For the present study, `usage-based factors' basically is equated to `textual frequency of the individual case-forms'. 
This factor is strongly influenced by the possibility of a language to omit overt arguments (and the use the language makes of this possibility) and the optionality or non-optionality of the overt case-markers.  
These two aspects of usage frequency will be discussed in more detail in Section~\ref{omitarg} and Section~\ref{omitcase} respectively.  
First, however, I will address the topic of textual frequencies and how it is relevant for the present study from a more general point of view.

\citet[38]{Zipf:1935} prominently noted ``that the length of a word tends to bear an inverse relationship to its relative frequency.'' 
While this observation refers to a language's vocabulary in its totality, it can also be applied to the paradigmatic structure of individual words, such as the different case-forms of a noun. 
An observation which more specifically addresses the relation between frequency and the length of morphological forms (of the same word) was made by \citet{Fenk:2001}  -- among numerous other frequency effects she postulates:

\begin{quote}
So we may say that relatively independent of its degree of markedness, that which is more frequent because of its natural salience and/or cultural importance: [\dots] is encoded in shorter morphological form \citep[435]{Fenk:2001}
\end{quote}

Translated into the domain of case-marking this means that a case-form which does not employ overt morphological marking should be more frequent than case-forms which bear overt coding. 
For marked"=S languages this can be broken down to the formula: zero-case is more frequent than S-case.
This prediction in principle goes in the same direction as the functional approach to marked"=S, though it proposes a completely different direction for research. 
While a case-form might be used in a wider number of functions throughout the grammar (because it covers a larger set of roles and/or appears in a larger number of constructions), this does not have to be reflected in any kind of usage frequency effect. 
A case-form that appears in a large number of marginal constructions might still be significantly less frequent than a case-form that is employed in the most widely used construction. So the survey of the grammar of a language and the contexts where case-forms are used does not necessarily give any insights into usage frequencies. 

In order to get informative results on the usage frequencies of individual case-forms one would need extensive corpora with data from a wide variety of different genres (narratives, spoken discourse etc.). 
Statistically meaningful comparison across languages can only be achieved when the types of data used in the analysis are comparable across languages. 
Otherwise the results cannot be interpreted. If, for example, one compares languages A and B and the data for language A comprise naturalistic examples from spoken discourse but language B is only represented through elicited narratives, one runs into severe problems.
%\footnote{For a basic phenomenon such as the case-marking system of a language, the differences in frequency between different genera may not be as dramatic as for other parts of the grammar. However, comparing elicited example sentences with longer narratives or even natural discourse data will most likely still reveal a significant difference in the number of overtly realized noun phrases, for example.} 
In this constellation any differences arising between languages A and B could either be due to a differences between the languages studied or due to the different types of data. 
For a discussion on how representativeness in corpora can be achieved see \citet{Biber:1990,Biber:1993} and \citet[13--21]{McEnery:2006}.  %Sections A2, A8, B1

Unfortunately the situation is such that for most of the languages from my sample data of the nature described above are not accessible or do not exist at all. 
Setting up corpora for twenty or so languages -- for most of which quite an amount of data would have to be gathered in the first place -- is certainly beyond the scope of this work. Therefore, as regrettable as it is, a frequency-based study of marked"=S systems is precluded from this study.

As just pointed out, it is absolutely necessary to have extensive, reliable and balanced data from actual language use in order to make any strong claims about the distribution of certain forms in a language. 
However, this does not mean that the usage data of a language are completely detached from its grammar. 
The grammar of a language can specify a number of parameters which will strongly influence language usage. 
The parameters in question here are the ones which determine what can and cannot be left out (and under which circumstances) in a language. 
With regard to marked"=S languages, and more specifically with regard to the range of usage of the zero- and S-case-forms, this boils down to two factors, which are discussed in the following sections. First, can core arguments be omitted? And second, is the use of the overt markers of core arguments optional in the language? 
If a language allows for any of these possibilities, the question arises, how frequently speakers make use of these possibilities, and what are the factors that influence the choice between omission and occurrence of the marker.\is{frequency|)} 

\subsection{Omission of arguments}\label{omitarg}\is{argument!omission of|(}

The tendency to leave out arguments that are required by a verb's semantic profile has long been noted for a number of languages of otherwise completely different typological profiles. \citet[131--132]{Gilligan:1987} finds that in his genealogically balanced sample of 100 languages around 80\,\% allow the omission of topical\is{topic} subjects. 
In addition, at least one third of the languages allow for non-topical subjects to be omitted as well. 
This phenomenon is often referred to as `pro-drop', a term that is prevalently used for the phenomenon of subject omission -- especially by linguists of a more formal persuasion -- but has been extended to the domain of object omission by at least some scholars \citep[e.g.][]{Rizzi:1986}. 

Many languages allow for the omission of overt NPs if they can be understood from the context, and indeed speakers of such languages make wide use of this.
Since subjects are typically highly topical\is{topic}, and subject NPs are especially prone to lacking overt realization. 
This suggests that the actual textual frequency of overt subjects (and thus S-case-marked NPs) will be lower than the frequency of overt objects. 
For the reasons listed above, a corpus analysis supporting the fact that subject NPs are omitted significantly more often than object NPs in marked"=S languages will have to wait until representative corpora for the languages studied will be available. 
At the current stage, I can only give an impressionistic evaluation of the preferences in dropping overt arguments in speech for the languages of my sample. 

Most of the languages under investigation here allow for the possibility of omitting arguments in actual speech. 
This is especially obvious where collected texts are available. 
In cases in which the author of a grammar used mainly naturalistic data for illustration instead of elicited examples, this has even led to the situation that hardly any examples could be found of a given construction to illustrate the nominal case-marker for the purpose of this study. 
Transitive subjects expressed by overt nominals were the hardest to find throughout all languages surveyed. 
This again hints at a lower frequency of S-case-marked forms, at least in the languages of the marked"=nominative type. 
For Nias\il{Nias}, overt transitive subjects were particularly hard to find.
This should lead to a lower figure for the zero-coded Unmutated (Ergative\is{case!individual forms!ergative}) in textual counts, resulting in a situation in which the overtly coded Mutated form of the noun (the absolutive\is{case!individual forms!absolutive}) could actually be the most frequent case-form in a corpus.\is{argument!omission of|)}

 
\subsection{Optional case-marking}\label{omitcase}\is{case-marking!optional|(}\is{case-marking!absence of}

A\is{case-marking!optional|(} final factor influencing the frequency of each case-form is the optionality of case-marking.
In some situations, an overt marker is employed only occasionally to mark the subject relation, while in other instances the marker is absent from an NP in the very same role. 
This is the case for quite a number of marked"=S languages. 
These will briefly be discussed in this section. 
The reasons usually listed for this behavior are often related to need to distinguish between different participants and their relevant roles in a given situation. 
However, these explanations are rather tentative, for the most part.  

In the description of the Australian language Malakmalak\il{Malakmalak}, it is noted that there is an optional Nominative suffix.
\citet[112]{Birk:1976} describes the distribution of the marker with the following words ``[it] can be suffixed to transitive or intransitive subject, but not to transitive object.'' 
The case-suffix is only employed when it cannot be distinguished otherwise, if an argument is the subject or object of the verb (i.e. if they are of the same person and gender, otherwise verbal indexing gives clues for identification). 
The need for disambiguating between participants does not appear to arise very frequently in Malakmalak since the examples in the grammar hardly provide any instances of the Nominative case-form. 
Two of the few examples is given in (\ref{MalEx}).

%\pagebreak

\begin{exe}\ex\label{MalEx}\langinfo{Malakmalak}{Australian; Northern Territory}{\citealp[113]{Birk:1976}}
\begin{xlist}
\ex\gll\textipa{alalk} \textipa{\textbf{yikpi-waN}} \textipa{yin\super{y}a} \textipa{ta\v r} \textipa{yimin\super{y}n\"o}\\
child little.\sg{}.\mas{}-\nom{} man bite 3\sg{}.\mas{}.\sbj{}.\punc{}.3\sg{}.\mas{}.\obj{}\\
\glt `The little boy bites/bit the man.'
\ex\gll\textipa{yin\super{y}a} \textipa{alalk} \textipa{\textbf{yikpi-waN}} \textipa{ta\v r} \textipa{yimin\super{y}n\"o}\\
man child little.\sg{}.\mas{}-\nom{} bite 3\sg{}.\mas{}.\sbj{}.\punc{}.3\sg{}.\mas{}.\obj{}\\
\glt `The little boy bites/bit the man.'
\end{xlist} 
\end{exe}

This phenomenon is often discussed under the title `optional ergativity', even though a language might permit this optional ergative\is{case!individual forms!ergative} marker to occur on intransitive subjects as well. 
The phenomenon appears to be particularly widespread in Australia and the non-Austronesian languages of Oceania (commonly referred to as Papuan). 
The absence or presence of the overt marker is often linked to the discourse structure of a given utterance (for a discussion of optional ergativity see \cite{OptionalErg} and the other papers in the special issue of \emph{Lingua} dedicated to this very topic). 
A more detailed discussion of the interaction of these kinds of information will be provided in Chapter~\ref{emphaticS}.\is{case-marking!optional|)} 

A similar situation seems to hold in some Yuman languages of North-West America, though the Nominative marker seems to be used more often in these languages. 
\citet[19]{Munro:1976} notes that in Mojave\il{Mojave} ``[o]ccasionally, when the context is clear, the subject case-marker may be omitted, particularly in fast speech, and with intransitive verbs.'' 
In the closely-related language Jamul\il{Jamul Tiipay} Tiipay, the Nominative marker \emph{-ch} is optional with most noun phrases. 
According to \citet[160]{Miller:2001}, it ``appears obligatorily on lexical demonstratives and on the interrogative/indefinite word \emph{me'a} `where?, somewhere''' and is almost exceptionless ``on noun phrases marked with the demonstrative clitic \emph{-pu}'', but in other context the case-marker is optional. 
Also, some African languages allow for the omission of S-case-marking, as for example has been noted for the Cushitic language Boraana\il{Oromo (Boraana)} Oromo \citep[93]{Stroomer:1995}.
   
The discussions of the mechanisms triggering the presence of the case-marker are very sparse, if present at all, especially in the languages of Australia and Oceania. 
This makes it very difficult to include their data into this present study.  
Usually, the discussion is restricted to the presentation of a few odd examples. 
In the rest of the grammatical description, the phenomenon is not treated in any more detail, so that for a given construction it is usually not clear whether it would allow for the presence of the respective case-marker on either of its arguments. 
Any judgments, whether a given role in encoded by the zero-case only, or if marking with the S-case is also possible, would have to be based on negative evidence. 
Eyeballing texts from the languages in which the S-case-marker is optional hints that they only rarely make use of the overt-S marker, so that for textual frequency one has to expect a clear dominance of the zero-case.\is{case-marking!optional|)}

\section{Summary}

In this chapter, the methodological basis of this study of marked"=S systems was presented.
This methodology draws heavily upon the notion of split alignment systems, which has been a central aspect in the research on morphosyntactic alignment in past decades.
In my approach, the idea of different alignments existing in different domains of a grammar is taken one step further.
The alignment systems of marked"=S languages are investigated at a micro-level by surveying a set of very specific contexts. 
By looking at all these contexts, the claim that the zero-case in marked"=S languages has the widest distribution is to be tested.

In the final section, I have discussed another factor influencing the distribution of case-forms in a language, namely textual frequency.
I have argued that a corpus analysis would provide the ultimate measure for which case-form has the widest distribution in a given language.
For marked"=S languages dealt with here, no such corpora exist at present.
\footnote{One exception to this is Savosavo\il{Savosavo}, on which a large amount of corpus work has been done in recent years, e.g. \citet{Haig:2011}. 
However, this work was only published after the completion of the original research for this book.}  
However, coming up with actual figures on the usage of the two case-forms for at least a subset of the languages from my sample is a very desirable enterprise for future studies.





